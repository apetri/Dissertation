%%%%%%%%%%%%%%%%%%%%%%%%%%%%%%%%%%%%%%%%%%%%%%%%%%%%%%%%%%%%%%%%%%%%%%%%%%%

\chapter{Numerical Weak Lensing}
\lhead[\fancyplain{}{\thepage}]{\fancyplain{}{\rightmark}}
 \thispagestyle{plain}
\setlength{\parindent}{10mm}

\label{chp:3}
In this Chapter we outline the numerical methods we used to produce simulated WL convergence and shear images. 

%%%%%%%%%%%%%%%%%%%%%%%%%%%%%%%%%%%%%%%%%%%%%%%%%%%%%%%%%%%%%%%%%%%%%%%%%%%

\section{Cosmological simulations}
The matter density contrast $\delta$ evolves according to equation (\ref{eq:1:delta-lin}) at linear stage. For WL studies, however, $\delta$ becomes too big at the redshifts of interest ($z\sim 1$), the linear approximation breaks down and an exact solution of (\ref{eq:1:b-monopole}), (\ref{eq:1:b-dipole}) and (\ref{eq:1:einstein-full}) is required. A popular approach to solve the Boltzmann equation (\ref{eq:1:boltzmann}) for collisionless dark matter is the $N$--body method, in which phase space is discretized using a large but finite number $N_p$ of particle tracers (see \citep{gadget2}). The particles are placed in a cubical periodic box of comoving size $L_b$ and are given initial conditions that correspond to the linear density contrast at high $z$. The system is then evolved with a Hamiltonian that mimics Newtonian gravitational interactions ($L_b\ll c/H$), described by the potential $\Phi$.  

\subsection{Initial conditions}
The starting point of an $N$--body simulation is a configuration of particles positions $\{\bb{x}_i\}$ and velocities $\{\bb{v}_i\}$ at some high redshift $z_{\rm in}\gg 1$, which mimic the linear density contrast $\delta$. Suppose that the particles are initially are arranged in a ``glass'' pattern and are given positions $\{\bb{x}_i^g\}$ that correspond to a uniform density profile ($\delta=0$). We seek to displace each particle by a small amount $\bb{d}(\bb{x}_i^g)$, so that the new density matches an input $\delta$ profile, which we can vary. Because mass is conserved by the displacement transformation we can write

\begin{equation}
\label{eq:3:masscons}
\rho_m d^3 x^g = \rho_m(1+\delta)d^3x
\end{equation} 
%
Which relates the density contrast $\delta$ to the Jacobian of the displacement transformation 

\begin{equation}
\label{eq:3:delta-displ}
1+\delta = \left\vert\mathds{1}_{3\times 3}+\frac{\partial \bb{d}}{\partial \bb{x}}\right\vert^{-1}
\end{equation}
%
Using the identity

\begin{equation}
\label{eq:3:det-identity}
\vert\mathds{1}+\lambda\bb{M}\vert = 1 + \lambda\Tr\bb{M} + O(\lambda^2)
\end{equation}
%
for generic $\bb{M}$ and real $\lambda$, and noting that at high redshift we expect $\delta$ and $\bb{d}$ to be small, we immediately obtain a linear relation between the density contrast and the displacement field 

\begin{equation}
\label{eq:3:delta-displ-real}
\delta = -\nabla\cdot\bb{d}
\end{equation}
%
A possible solution to equation (\ref{eq:3:delta-displ-real}) is easily expressed in Fourier space if we assume the displacement field to be longitudinal (which is a good assumption since the peculiar velocity field is irrotational)

\begin{equation}
\label{eq:3:delta-displ-fourier}
\tilde{\bb{d}}(\bb{k}) = \frac{i\bb{k}}{k^2}\tilde{\delta}(\bb{k})
\end{equation}
%
Equation (\ref{eq:3:delta-displ-fourier}) takes the name of Zel'dovich approximation (see \citep{ZeldovichWhite} for a review), which essentially states that the displacement field is the gradient of the local gravitational potential. The Fourier coefficients $\tilde{\delta}(\bb{k})$ can be randomly drawn from $\mathcal{N}(0,P_\delta(k,z_{\rm in}))$. The linear matter power spectrum $P_\delta(k,z_{\rm in})$ can be computed analytically with Einstein--Boltzmann codes such as \ttt{CAMB} \citep{CAMB}. We assign the initial peculiar velocities $\bb{v}=\dot{\bb{d}}$ in the context of the Zel'dovich approximation using the time derivative of $\delta$. Since we limit ourselves to dark matter density perturbations, we can assume self--similar linear growth, described by the linear growth factor $D(z)$ that appears in (\ref{eq:1:growth-diff}). In order to imprint baryonic physics in the initial conditions, we adopt a hybrid approach, in which we use \ttt{CAMB} to compute the linear matter power spectrum $P^{\rm lin}_\delta(\bb{k},0)$ at the present time, with baryons included, and scale it back to $z_{\rm in}=100$ using the linear growth factor

\begin{equation}
\label{eq:3:hybrid-power}
P_\delta(\bb{k},z_{\rm in}) = P^{\rm lin}_\delta(\bb{k},0)\left(\frac{D(z_{\rm in})}{D(0)}\right)^2 
\end{equation}
%
The initial conditions with baryonic imprinted physics is then evolved with dark matter only collisionless dynamics. A random realization of $\tilde{\delta}$ can then be drawn from $P_\delta(\bb{k},z_{\rm in})$, and the peculiar velocities can be assigned as 

\begin{equation}
\label{eq:3:peculiar-initv}
\tilde{\bb{v}}(\bb{k}) = \frac{i\bb{k}}{k^2}\tilde{\delta}(\bb{k})\left(\frac{\dot{z}}{D(z)}\frac{dD(z)}{dz}\right)_{z=z_{\rm in}}
\end{equation}  
%
In order to generate random realizations of the initial conditions from the linear power spectrum $P^{\rm lin}_\delta(\bb{k},0)$, we used the \ttt{N-GenIC} software add--on to \ttt{Gadget2} \citep{gadget2}.   

\subsection{Time evolution}
Once generated, the initial conditions specified by equations (\ref{eq:3:delta-displ-fourier}) and (\ref{eq:3:peculiar-initv}) must be evolved in time from $z=z_{\rm in}$ until the present $z=0$. Since we consider collisionless dark matter only, which interacts only via gravity, the Hamiltonian $\mathcal{H}$ of the particle system (ignoring PN corrections, which is reasonable if $L_b\ll c/H$) can be written as 

\begin{equation}
\label{eq:3:collisionless-ham}
\mathcal{H} = \sum_{i=1}^{N_p} \frac{\bb{p}_i^2}{2m_i a(t)^2} + \frac{1}{2}\sum_{i\neq j}m_im_j\varphi(\bb{x}_i-\bb{x}_j)
\end{equation} 
%
We denoted as $m_i$ the particle masses, as $\bb{p}_i$ the particle momenta conjugated to the comoving coordinates $\bb{x}_i$ and as $\varphi$ the pair interaction potential (which comes from gravity) per unit mass. If one imposes periodic boundary conditions on the simulation box, the interaction potential satisfies the Poisson equation

\begin{equation}
\label{eq:3:part-poisson}
\nabla^2\varphi(\bb{x}) = \frac{4\pi G}{a}\left(\sum_{\bb{n}\in\mathds{Z}^3}\delta_{r_s}(\bb{x}-\bb{n}L_b)-\frac{1}{L_b^3}\right) 
\end{equation}
%
Where $\delta_{r_s}$ is the Dirac delta function $\delta^D$ convolved with a softening kernel of scale $r_s$. This softening is introduced because the $N$--body particles are in reality extended, collisionless, objects and the Newtonian interaction potential is smoothed out on interaction scales smaller than $r_s$. In our simulations $r_s$ has been fixed to $r_s\approx 10\,{\rm kpc}/h$. Note that the summation term can be dropped if we restrict $\bb{x}$ to be inside the box, but is important to enforce the periodic boundary conditions. We can relate $\varphi$ to the gravitational potential in (\ref{eq:1:poisson}) by    

\begin{equation}
\label{eq:3:macro-potential}
\Phi(\bb{x},t) = -\frac{1}{c^2}\sum_{i=1}^{N_p} m_i\varphi(\bb{x}-\bb{x}_i(t))
\end{equation} 
%
We can observe that, inside the simulation box, (\ref{eq:3:macro-potential}) leads to 

\begin{equation}
\label{eq:3:macro-potential-poisson}
\nabla^2\Phi(\bb{x},t) = -\frac{4\pi Ga^2}{c^2}\left(\sum_{i=1}^{N_p}m_i\delta_{r_s}\left(a(\bb{x}-\bb{x}_i(t))\right)-\frac{1}{a^3L_b^3}\sum_{i=1}^{N_p}m_i\right)
\end{equation}
%
Note that (\ref{eq:3:macro-potential-poisson}) is essentially the discretized version of (\ref{eq:1:poisson}) for a system with $N_p$ particles, where gravitational forces are softened on scales below $r_s$. The Hamiltonian equations of motion derived from (\ref{eq:3:collisionless-ham}) can be numerically integrated to yield a trajectory $\bb{x}(t)$ for each particle. To preserve the Hamiltonian nature of the time evolution, \citep{gadget2} suggest adopting a Kick-Drift-Kick (KDK) numerical integration scheme. The drift step updates the particle coordinates using their momenta, while the kick updates their momenta using the local force field. The force field calculation requires the solution of (\ref{eq:3:part-poisson}) and a summation over all particle pairs, which leads to an $O(N_p^2)$ time complexity. In the limit of collisionless dynamics, approximate force field calculations can be carried over at a significantly lower complexity using the hybrid Tree Particle Mesh (TreePM) approach. The details of the force field calculation, time integration and TreePM implementation can be found in the \ttt{Gadget2} paper \citep{gadget2}. We used the publicly available version of the \ttt{Gadget2} code to perform the $N$--body simulations on which our WL simulations are based. We stored the $N$--body simulation outputs $\{\bb{x}_i(t)\}$ at a discrete set of time steps $\{t_k\}$, with the goal of using them to estimate the potential $\Phi$ necessary to reconstruct the WL quantities $\kappa,\pmb{\gamma}$. The numerical details of these WL simulations are contained in the next section.  

%%%%%%%%%%%%%%%%%%%%%%%%%%%%%%%%%%%%%%%%%%%%%%%%%%%%%%%%%%%%%%%%%%%%%%%%%%%

\section{The multi--lens--plane algorithm}

\subsection{Geodesic solver}
The goal of this section is to review the algorithm used to solve the light geodesic equation (\ref{eq:2:geodesic-fo-3}), and that allows to compute the integrals for $\pbeta$ (\ref{eq:2:geosol-beta}) and $\bb{A}$ (\ref{eq:2:geosol-jac}) efficiently and in a numerically stable fashion. In the remainder of the Chapter we will consider a source placed at a fixed longitudinal comoving distance $\chi_s$. The source appears to be positioned at an angle $\pt$ on the sky, but its real position is at angle $\pbeta(\chi_s,\pt)$, which can be calculated by solving (\ref{eq:2:geodesic-fo-3}). Numerical integration of (\ref{eq:2:geosol-beta}) can be done by dividing the interval $\chi\in[0,\chi_s]$ in $N_l$ equally spaced steps, each of size $\Delta = \chi_s/N_l$ and using a first order explicit method

\begin{equation}
\label{eq:3:int-fo}
\int_0^{\chi_s} f(\chi)d\chi = \Delta\sum_{i=1}^{N_l}f(\chi_k) + O\left(\frac{1}{N_l}\right)
\end{equation} 

\begin{equation}
\label{eq:3:int-steps}
\chi_k = k\Delta
\end{equation}  
%
Where $f$ is a generic function of $\chi$ and can be either $\pbeta$ or $\bb{A}$. Before applying the numerical integration method to (\ref{eq:2:geodesic-fo-3}), it is convenient to recast the geodesic equation into an equation for $\pbeta=\xp/\chi$ 

\begin{equation}
\label{eq:3:geodesic-fo-beta-1}
\frac{d^2}{d\chi^2}(\chi\pbeta(\chi)) = \frac{2}{\chi}\nabla_{\pbeta}\Phi(\chi,\pbeta(\chi))
\end{equation} 
%
We promoted the $\xp$ dependency of $\Phi$ to a $\beta$ dependency using $\Phi(\chi,\xp=\chi\pbeta)\rightarrow\Phi(\chi,\pbeta)$. Equation (\ref{eq:3:geodesic-fo-beta-1}) is equivalent to 

\begin{equation}
\label{eq:3:geodesic-fo-beta-2}
\frac{d^2\pbeta(\chi)}{d\chi^2} + \frac{2}{\chi}\frac{d\pbeta(\chi)}{d\chi} - \frac{2}{\chi^2}\nabla_{\pbeta}\Phi(\chi,\pbeta(\chi)) = 0
\end{equation}
%
Now let us focus on an intermediate discrete step $k$ and introduce the compact notation 
\begin{equation}
\label{eq:3:compactnotation}
\begin{matrix}
f_k\equiv f(\chi_k) & ; & f'_k\equiv\left.\frac{df}{d\chi}\right\vert_{\chi=\chi_k} & ; & f''_k\equiv\left.\frac{d^2f}{d\chi^2}\right\vert_{\chi=\chi_k}
\end{matrix}
\end{equation}  
%
and define
\begin{equation}
\label{eq:3:alphak}
\palpha_k = \frac{2\Delta}{\chi_k}\nabla_{\pbeta}\Phi(\chi_k,\pbeta_k)
\end{equation}
%
Using the first order finite difference approximations for the $\pbeta$ derivatives

\begin{equation}
\label{eq:3:finitediff}
\begin{matrix}
\pbeta'_k = \frac{\pbeta_{k+1}-\pbeta_{k-1}}{2\Delta} + O(\Delta^2) & ; & \pbeta''_k = \frac{\pbeta_{k+1}+\pbeta_{k-1}-2\pbeta_k}{\Delta^2} + O(\Delta^2)
\end{matrix}
\end{equation}
%
we can write equation (\ref{eq:3:geodesic-fo-beta-2}) as 
\begin{equation}
\label{eq:3:geodesic-fo-beta-disc}
\frac{\pbeta_{k+1}+\pbeta_{k-1}-2\pbeta_k}{\Delta^2} + \frac{\pbeta_{k+1}-\pbeta_{k-1}}{\chi_k\Delta} - \frac{\palpha_k}{\chi_k\Delta} = 0
\end{equation} 
%
Once we solve (\ref{eq:3:geodesic-fo-beta-disc}) for $\pbeta_{k+1}$, we immediately find

\begin{equation}
\label{eq:3:betakp1}
\pbeta_{k+1} = \frac{2\pbeta_k\chi_k-(\chi_k-\Delta)\pbeta_{k-1}+\Delta\palpha_k}{\chi_k+\Delta}
\end{equation}  
%
The expression (\ref{eq:3:betakp1}) has a simple physical interpretation if we look at the scheme in Figure \ref{fig:3:multi-lens-plane}.
\begin{figure}
\begin{center}
\includegraphics[scale=0.7]{Figures/pdf/multi_lens_plane.pdf}
\end{center}
\caption{Multi--lens--plane algorithm schematics: the trajectory of a single light ray from the observer to the source at $\chi_s$ is shown in red as it undergoes multiple deflections.}
\label{fig:3:multi-lens-plane}
\end{figure}
%
If we want to compute the angular position of a light ray at the $k+1$-th step, we need to know its position at the two previous steps $k,k-1$. Simple geometric arguments, and the fact that angles are small, tell us that 

\begin{equation}
\label{eq:3:betakp1-2}
\pbeta_{k+1} = \frac{1}{\chi_{k+1}}\left[(\chi\pbeta)_{k} + \left(\frac{(\chi\pbeta)_{k}-(\chi\pbeta)_{k-1}}{\chi_k-\chi_{k-1}}+\palpha_k\right)(\chi_{k+1}-\chi_k) \right]
\end{equation}
%
Note that equations (\ref{eq:3:betakp1}) and (\ref{eq:3:betakp1-2}) are equivalent if the steps are equally spaced, $\chi_k=k\Delta$. This tells us that the quantity $\palpha_k$, which is connected to the gradient of the potential as in (\ref{eq:3:alphak}), is the deflection angle that a light ray experiences upon impact with a two dimensional lens plane of thickness $\Delta$ positioned at a longitudinal distance $\chi_k$. This is why the procedure of solving (\ref{eq:3:geodesic-fo-beta-2}) in discrete $\chi$ steps takes the name of multi--lens--plane algorithm \citep{RayTracingJain,RayTracingHartlap}: the solving procedure is equivalent to considering a discrete set of trajectory deflections $\palpha_k$ caused by a discrete set of two dimensional lens planes, each described by a lensing potential which is essentially the three dimensional gravitational potential $\Phi$ projected along the longitudinal direction. We observe that equation (\ref{eq:3:alphak}) is essentially the longitudinal integral of $\Phi$ performed with one discrete step of size $\Delta$. Using the initial conditions

\begin{equation}
\label{eq:3:initcond}
\pbeta_0 = \pbeta_1 = \pt 
\end{equation}    
%
we can use the recurrence relation (\ref{eq:3:betakp1}) to compute the full light ray trajectory until the arrival point $\pbeta_s$. It turns out that, because the coefficient that multiplies $\pbeta_k$ in (\ref{eq:3:betakp1}), $2\chi_k/(\chi_k+\Delta)$ is usually bigger than 1, this explicit method of solution leads to roundoff errors which blow up exponentially in $k$. To keep the accuracy of the geodesic solver under control we recast (\ref{eq:3:betakp1}) in a slightly different form by defining $\delta\pbeta_k \equiv \pbeta_k-\pbeta_{k-1}$. It is straightforward to show

\begin{equation}
\label{eq:3:betak-sum}
\pbeta_k = \pt + \sum_{i=1}^k\delta\pbeta_i
\end{equation}
%
\begin{equation}
\label{eq:3:betakp1-delta}
\delta\pbeta_{k+1} = \left(\frac{\chi_k-\Delta}{\chi_k+\Delta}\right)\delta\pbeta_k + \left(\frac{\Delta}{\chi_k+\Delta}\right)\palpha_k
\end{equation} 
%
It turns out that, because the coefficients that multiply $\delta\pbeta,\palpha$ are smaller than 1, (\ref{eq:3:betak-sum}) and (\ref{eq:3:betakp1-delta}) offer a more accurate numerical solution to the geodesic equation (\ref{eq:3:geodesic-fo-beta-2}). We can solve the geodesic equation for light rays with different initial conditions $\pt$, and study how the solution varies with $\pt$. This allows to translate the recurrence relations (\ref{eq:3:betak-sum}), (\ref{eq:3:betakp1-delta}) into recurrence relations for the lensing Jacobian $\bb{A}$. Noting that 

\begin{equation}
\label{eq:3:Tk}
\frac{\partial (\alpha_i)_k}{\partial \theta_j} = \frac{2\Delta}{\chi_k}\partial_{\beta_i}\partial_{\beta_l}\Phi(\chi_k,\pbeta_k)\frac{\partial (\beta_l)_k}{\partial\theta_j}
\end{equation}
%
we can define the projected tidal field 

\begin{equation}
\label{eq:3:tidal-proj}
\bb{T}_k = 2\chi_k\Delta\bb{T}^\Phi(\chi_k,\pbeta_k) 
\end{equation}
%
and define the recurrence relations for the Jacobian $\bb{A}$

\begin{equation}
\label{eq:3:jack-sum}
\bb{A}_k = \mathds{1}_{2\times 2} + \sum_{i=1}^k\delta\bb{A}_i
\end{equation}
%
\begin{equation}
\label{eq:3:jackp1-delta}
\delta\bb{A}_{k+1} = \left(\frac{\chi_k-\Delta}{\chi_k+\Delta}\right)\delta\bb{A}_k + \left(\frac{\Delta}{\chi_k+\Delta}\right)\bb{T}_k\bb{A}_k
\end{equation} 
%
The recurrence relations (\ref{eq:3:jack-sum}),(\ref{eq:3:jackp1-delta}) can be used to estimate the WL quantities $\kappa_s,\pmb{\gamma}_s$ at an arbitrary angle in the sky $\pt$ in $O(N_l)$ time, given a set of discrete deflections $\palpha_k$ and tidal distortions $\bb{T}_k$, which can be computed from the potential $\Phi$. In the next section we will review the numerical methods necessary to solve the Poisson equation (\ref{eq:2:poisson}) that connects the potential $\Phi$ to the matter density contrast $\delta$. 

\subsection{Poisson solver}
In the previous section we showed that the lensing Jacobian $\bb{A}$ and the corresponding WL observables can be calculated numerically (using the relations \ref{eq:2:dfl-inverse}) once one knows the lens deflections and tidal distortions caused by each lens plane. These are ultimately determined by the matter density fluctuations that are responsible for the lensing effect in the first place. We define the longitudinally projected potential $\psi$ for a lens plane centered at comoving distance $\chi$ with thickness $\Delta$ as follows

\begin{equation}
\label{eq:3:projected-phi}
\psi(\chi,\pbeta) = \frac{2}{\chi}\int_{\chi-\Delta/2}^{\chi+\Delta/2}d\chi' \Phi(\chi',\pbeta)
\end{equation} 
%
Now, using the definition of (\ref{eq:3:projected-phi}) we can obtain the relations for the deflections and tidal distortions in terms of $\psi$

\begin{equation}
\label{eq:3:alphak-psi}
\palpha_k = \nabla_{\pbeta}\psi(\chi_k,\pbeta_k)
\end{equation}
%
\begin{equation}
\label{eq:3:Tk-psi}
\bb{T}_k = \nabla_{\pbeta}\nabla^T_{\pbeta}\psi(\chi_k,\pbeta_k)
\end{equation}
%
Plugging (\ref{eq:3:projected-phi}) into the Poisson equation (\ref{eq:2:poisson}) we obtain that $\psi$ must satisfy a Poisson-like equation itself

\begin{equation}
\label{eq:3:poisson-psi-1}
\nabla^2_{\pbeta}\psi(\chi,\pbeta) = \frac{2}{\chi}\int_{\chi-\Delta/2}^{\chi+\Delta/2}d\chi' \chi'^2\left(\nabla^2-\frac{\partial^2}{\partial \chi'^2}\right)\Phi(\chi',\chi'\pbeta)
\end{equation}
%
We made an assumption of $\Delta$ being small in approximating $\nabla^2_{\pbeta}\approx \chi^2(\nabla^2-\partial_\chi^2)$, so that we can neglect the time evolution of $\Phi$ within the lens. If $\Delta$ is small we can also treat the $\partial^2_\chi$ integrated term as a boundary term which vanishes when appropriate boundary conditions for the Poisson equation are chosen (for example periodic boundary conditions). With the help of (\ref{eq:2:poisson}) in the end we obtain

\begin{equation}
\label{eq:3:poisson-psi-2}
\nabla^2_{\pbeta}\psi(\chi,\pbeta) = -\sigma(\chi,\pbeta)  
\end{equation} 

\begin{equation}
\label{eq:3:lens-sigma}
\sigma(\chi,\pbeta) = \frac{8\pi G \chi a(\chi)^2\Delta}{c^2}\bar{\rho}_m(\chi)\delta(\chi,\chi\pbeta) = \frac{3H_0^2\Omega_m\chi\Delta}{c^2 a(\chi)}\delta(\chi,\chi\pbeta)
\end{equation}
%
The dimensionless surface density $\sigma$ in (\ref{eq:3:lens-sigma}) for a lens plane can be estimated directly from the outputs of $N$--body simulations using a particle number count histogram to measure the density contrast $\delta$. Suppose we have a list of $N_p$ particle positions $\{(x_p,y_p,z_p)\}$ computed at time $\chi$, and let's assume without loss of generality that $z$ is the longitudinal direction and $(x,y)$ are the transverse coordinates. We consider a two dimensional regularly spaced transverse grid $\{(x_i,y_i)\}$ with $L_P$ pixels per side, each of size $L_b/L_P$. We assign to each pixel a particle number count

\begin{equation}
\label{eq:3:number-hist}
n(\chi, \pbeta_i) = \sum_{p=1}^{N_p}w(\bb{x}_p,\chi,\pbeta_i)
\end{equation}  
%
where 

\begin{equation}
\label{eq:3:number-kernel}
w(\bb{x}_p,\chi,\pbeta_i) = 
\begin{cases}
1 \,\,\,\, {\rm if} \,\,\,\, (x_p,y_p) \,\, {\rm in} \,\, \chi\pbeta_i \, , \, z_p\in [\chi-\Delta/2,\chi+\Delta/2] \\
0 \,\,\,\, {\rm otherwise}
\end{cases}
\end{equation}
%
We can then estimate the density contrast $\delta$ at each pixel from the histogram, assuming that all the simulation particles have the same mass

\begin{equation}
\label{eq:3:delta-hist}
\delta(\chi,\chi\pbeta_i) = \frac{n(\chi,\pbeta_i)L_bL_P^2 }{\Delta N_p} - 1
\end{equation}
%
Once the density contrast is estimated from the $N$--body outputs, the two dimensional Poisson equation (\ref{eq:3:poisson-psi-2}) can be solved on the regular transverse grid, at each of the discrete steps $\chi_k$. If we impose periodic boundary conditions on the edges of the lens plane, an efficient solution to (\ref{eq:3:poisson-psi-2}) can be calculated using the FFT of $\psi,\sigma$. Note that, because both these quantities are real, a real FFT is sufficient. Inverting the laplacian operator in Fourier space yields

\begin{equation}
\label{eq:3:poisson-sol-fft}
\tilde{\psi}(\chi_k,\pell) = \tilde{\sigma}(\chi_k,\pell)\frac{e^{-\ell^2\theta_G^2}}{\ell^2}
\end{equation} 
%
We decided to apply a Gaussian smoothing with scale $\theta_G$ to the solution (\ref{eq:3:poisson-sol-fft}) to get rid of sub-pixel particle shot noise. We chose $\theta_G$ to be the angular size of one lens pixel in real space. The time complexity of the potential calculation from the $N$--body outputs is dominated by the Poisson solver \citep{lenstools}, which has a runtime of $O(L_P^2\log L_P)$. Figure \ref{fig:3:lens} shows an example lens (density and potential) plane based on equations (\ref{eq:3:delta-hist}), (\ref{eq:3:poisson-sol-fft}). 

\begin{figure}
\begin{center}
\includegraphics[scale=0.4]{Figures/eps/lens_plane.eps}
\end{center}
\caption{Dimensionless density $\sigma$ (left) and corresponding potential $\psi$ for a lens plane at $z_l=0.7$, cut from a $N_p=512^3, L_b=240\,{\rm Mpc}/h$ $N$--body simulation.}
\label{fig:3:lens}
\end{figure}

\subsection{Cosmic variance sampling}
\label{sec:3:sampling}
The multi--lens--plane integration of (\ref{eq:2:geosol-jac}) suggests a way of producing multiple WL image realizations from a single $N$--body simulation. This is essentially due to the fact that the field of view spanned by the observed ray positions $\pt$, when projected at redshift $z>0$, usually covers the simulation box only partially since $L_b$ can be chosen so that $\chi_k\theta<L_b$. Periodic shifts of the lens planes along directions perpendicular to the line of sight yield lenses with identical statistical properties, but lead to different realizations of $\kappa,\pmb{\gamma}$ images. We can summarize the procedure of constructing the lens system as follows:

\begin{itemize}
\item For a lens plane at distance $\chi_k$, choose a random $N$--body simulation among a set of $N_s$ independent simulations ($N_s=1$ if only one full simulation has been run)
\item Choose a random direction between $(\h{x},\h{y},\h{z})$ to be the longitudinal direction. The other two directions will be the transverse coordinates $\xp$
\item Cut a random slice of size $\Delta$ from the $N$--body output at $t_k$, along the chosen longitudinal direction
\item Periodically shift the lens along the transverse directions by a random amount
\item Repeat for the next lens plane at distance $\chi_{k+1}$
\end{itemize}
%
This procedure produces different realizations of $\{\palpha_k,\bb{T}_k\}$ and hence allows to recycle the outputs of $N_s$ independent $N$--body simulations to produce $N_r\gg N_s$ realizations of WL observables. These simulated ensembles can be used to estimate the scatter in observable estimators coming from cosmic variance, as well as the estimator means. Because $N_r$ is bigger than $N_s$, these realizations will only be pseudo--independent, but can be safely treated as effectively independent if $N_r$ is not too large. This approximate independent issue, along with its implications on WL observation analysis, has been investigated in \citep{PetriVariance}. 

%%%%%%%%%%%%%%%%%%%%%%%%%%%%%%%%%%%%%%%%%%%%%%%%%%%%%%%%%%%%%%%%%%%%%%%%%%%

\section{Approximate methods}
In this section we review the numerical implementation of the approximate methods shown in (\ref{eq:2:kappa-1}), (\ref{eq:2:kappa-2-ll}) and (\ref{eq:2:kappa-2-gp}), which allow to compute the Born and first post--Born contributions to the convergence $\kappa$ as line--of--sight integrals on the unperturbed light ray trajectories. 

\subsection{Born approximation}
The computation of the Born contribution to $\kappa$ for sources at distance $\chi_s$ involves a single integral over $\chi$, which can be readily computed using the first order method in (\ref{eq:3:int-fo}). We can write, at $O(\Delta)$ precision 

\begin{equation}
\label{eq:3:kappa-fo-num-1}
\kappa^{(1)}_s(\pt) = -\Delta\sum_{k=1}^{N_l} W_{ks}\chi_k\nabla^2_\perp\Phi(\chi_k,\chi_k\pt)
\end{equation} 
%
Where we used the compact notation $W_{kk'}=1-\chi_k/\chi_{k'}$. If we use the relations (\ref{eq:3:projected-phi}) and (\ref{eq:3:poisson-psi-2}) we can relate the first order convergence $\kappa^{(1)}$ to the discrete set of dimensionless lens densities $\{\sigma_k\equiv\sigma(\chi_k,\chi_k\pt)\}$

\begin{equation}
\label{eq:3:kappa-fo-num-2}
\kappa^{(1)}_s = \frac{1}{2}\sum_{k=1}^{N_l} \sigma_k W_{ks}
\end{equation}
%
As it is evident from (\ref{eq:3:kappa-fo-num-2}), the Born--approximated convergence can be efficiently calculated in $O(N_l)$ time. Moreover such approximate approach does not require knowledge of the solution to the Poisson equation (\ref{eq:3:poisson-psi-2}). At linear order in the potential $\Phi$, the shear field $\pmb{\gamma}$ can be calculated using the KS relation (\ref{eq:2:gamma-c-ks}).  

\begin{figure}
\begin{center}
\includegraphics[scale=0.4]{Figures/eps/csample.eps}
\end{center}
\caption{Sample $\kappa$ reconstruction from one $N$--body simulation with $L_b=260\,{\rm Mpc}/h$ and $N_p=512^3$. The lens planes have a thickness of $\Delta=L_b/3$ and are resolved with $L_P^2=4096^2$ pixels. The $\kappa$ maps are reconstructed with $2048^2$ light rays arranged in a regular grid. The source galaxies were placed at a single redshift $z_s=2$.}
\label{fig:3:csample}
\end{figure}

\subsection{Post--Born corrections}
The evaluation of the second order corrections to $\kappa$ in equations (\ref{eq:2:kappa-2-ll}) and (\ref{eq:2:kappa-2-gp}) involve double integrals over $\chi$ which, if naively implemented, lead to an $O(N_l^2)$ runtime algorithm. Applying the first order method in (\ref{eq:3:int-fo}) twice we obtain

\begin{equation}
\label{eq:3:kappa-ll-num}
\kappa_s^{(2-{\rm ll})} = -\frac{1}{2}\sum_{k=1}^{N_l}\sum_{m=1}^k W_{ks}W_{mk} \Tr(\bb{T}_m\bb{T}_k)
\end{equation}
%
\begin{equation}
\label{eq:3:kappa-gp-num}
\kappa_s^{(2-{\rm gp})} = \frac{1}{2}\sum_{k=1}^{N_l}\sum_{m=1}^k W_{ks}W_{mk} (\palpha_m\cdot\nabla\sigma_k)
\end{equation}
%
Note that, since we are performing the integrals along unperturbed trajectories, the angular arguments of $\sigma_k,\palpha_k,\bb{T}_k$ have been fixed to $\pbeta_k\equiv\pt$ (for a single light ray). Note also that the gradient in (\ref{eq:3:kappa-gp-num}) is to be intended as with respect to the angular argument of $\sigma_k$. As previously noted, the naive implementation explained in (\ref{eq:3:kappa-ll-num}) and (\ref{eq:3:kappa-gp-num}) leads to an $O(N_l^2)$ runtime, which can be quite inefficient for a large number of lenses and light rays. A more efficient $O(N_l)$ algorithm can be employed noting that, as we compute the partial sums, we can cache the quantities 

\begin{equation}
\label{eq:3:cache}
\begin{matrix}
I^{\palpha,0}_k = \sum_{m=1}^k \palpha_m & ; & I^{\palpha,1}_k = \sum_{m=1}^k \chi_m\palpha_m & \\ \\
I^{\bb{T},0}_k = \sum_{m=1}^k \bb{T}_m & ; & I^{\bb{T},1}_k = \sum_{m=1}^k \chi_m\bb{T}_m 
\end{matrix} 
\end{equation} 
%
With the definitions (\ref{eq:3:cache}), the second order post--Born corrections to $\kappa$ can be computed in linear time as  

\begin{equation}
\label{eq:3:kappa-ll-num-lin}
\kappa_s^{(2-{\rm ll})} = -\frac{1}{2}\sum_{k=1}^{N_l} W_{ks} \Tr\left[\bb{T}_k\left(I^{\bb{T},0}_k-\frac{I^{\bb{T},1}_k}{\chi_k}\right)\right]
\end{equation}
%
\begin{equation}
\label{eq:3:kappa-gp-num-lin}
\kappa_s^{(2-{\rm gp})} = \frac{1}{2}\sum_{k=1}^{N_l} W_{ks}\nabla\sigma_k\cdot\left(I^{\palpha,0}_k - \frac{I^{\palpha,1}_k}{\chi_k}\right)
\end{equation}
%
Figure \ref{fig:3:csample} shows a sample $\kappa$ reconstruction from one $N$--body simulation, including the full ray--tracing map and a comparison between the residuals $\kappa-\kappa^{(1)}$ and the second order terms $\kappa^{(2-{\rm ll})},\kappa^{(2-{\rm gp})}$. 

%%%%%%%%%%%%%%%%%%%%%%%%%%%%%%%%%%%%%%%%%%%%%%%%%%%%%%%%%%%%%%%%%%%%%%%%%%%

\section{The \LT\, software package}
\label{sec:3:lt}
In this section we present \LT \citep{lenstools}, a {\sc python} software package that we developed to handle the numerical operations exposed in this Chapter in an efficient fashion. As its main purpose, \LT\, serves as a pipeline of operations that allow to produce simulated $\kappa,\pmb{\gamma}$ images starting from a set of parameters that define a $\Lambda$CDM cosmological model described in Chapter 1. The sequence of operations is described by the diagram in Figure \ref{fig:3:lt-flow}  
%
\begin{figure}
\begin{center}
\includegraphics[scale=0.5]{Figures/eps/lt_flow.eps}
\end{center}
\caption{Schematic of the \LT\, pipeline flow. Vertical arrows are directed from a particular application to its input. Horizontal arrows are directed from the input to the output products.}
\label{fig:3:lt-flow}
\end{figure}
%
The \LT\, pipeline glues together the \ttt{CAMB}, \ttt{N-GenIC} and \ttt{Gadget2} public codes, used for the $N$--body simulations, with {\sc python} code for the $\Phi$ calculations and ray--tracing operations. The solution to the Poisson equation (\ref{eq:3:poisson-psi-2}) can be efficiently found via FFT, which \LT\, performs via the {\sc numpy} FFTPack \citep{scipy}. The ray--tracing operations (\ref{eq:3:betak-sum}), (\ref{eq:3:betakp1-delta}), (\ref{eq:3:jack-sum}), (\ref{eq:3:jackp1-delta}) can also be performed efficiently in {\sc numpy} using vectorized linear algebra routines. \LT\, also provides efficient implementations of the approximate methods for computing $\kappa$ shown in equations (\ref{eq:3:kappa-fo-num-2}), (\ref{eq:3:kappa-ll-num-lin}) and (\ref{eq:3:kappa-gp-num-lin}). 

Table \ref{tbl:3:lt-benchmark} shows CPU time benchmarks for a test run performed on the XSEDE Stampede computer cluster (see \url{https://portal.xsede.org/tacc-stampede}). Each the $N_p$ particles in each snapshot are divided in $N_t$ files, which are read in parallel by $N_t$ independent tasks. After the gridding procedure (\ref{eq:3:number-hist}) is performed by each task, the total surface density (computed for a plane of $L_P^2$ pixels) is collected by the master task, which then proceeds in solving the Poisson equation via FFT according to (\ref{eq:3:poisson-sol-fft}) and saves the output to disk. In a subsequent step, the lens potential files are read from disk, and the geodesic equation (\ref{eq:3:geodesic-fo-beta-2}) is solved for $N_R$ different observed ray positions $\pt$. This allows to reconstruct the shear and convergence $\kappa,\pmb{\gamma}$ over the field of view spanned by $\pt$. Multiple $\kappa,\pmb{\gamma}$ realizations can be obtained with the sampling procedure described in \S~\ref{sec:3:sampling}. 
%
\begin{table}
\begin{center}
\begin{tabular}{l|c|c|c}
\toprule
{Step} &            Complexity &            Test case &           Runtime \\ \hline \hline
\midrule
\multicolumn{4}{c}{\textbf{Lens plane generation}} \\ \hline
$N$--body input\footnote{Perfect input performance is assumed in the complexity analysis} & $O(N_p/N_t)$  & $N_p=512^3$, $N_t=16$  & 2.10\,s  \\
Density estimation (\ref{eq:3:number-hist})       & $O(N_p/N_t)$   & $N_p=512^3$, $N_t=16$  & 0.20\,s \\
\ttt{MPI} Communication  & $O(L_P^2\log{N_t})$   & $N_t=16$, $L_P=4096$  & 0.76\,s   \\
Poisson solver (\ref{eq:3:poisson-sol-fft})           & $O(L_P^2\log{L_P})$ & $L_P=4096$  &  2.78\,s    \\
Lens plane output & $O(L_P)$ & $L_P=4096$   & 0.04\,s  \\ \hline \hline

\multicolumn{4}{c}{\textbf{Ray tracing}} \\ \hline
Lens plane input &  $O(L_P^2)$ & $L_P=4096$ & 0.32\,s \\
Random plane shift &  $O(L_P)$ & $L_P=4096$ & 0.15\,s \\
$\palpha_k,\bb{T}_k$ calculations (\ref{eq:3:alphak-psi}),(\ref{eq:3:Tk-psi})  &  $O(N_R)$ & $N_R=2048^2$   & 1.54\,s  \\
Tensor products $\bb{T}_k\bb{A}_k$ in (\ref{eq:3:jackp1-delta}) &  $O(N_R)$ & $N_R=2048^2$   &  1.29\,s \\ \hline \hline

\bottomrule
\end{tabular}
\caption{Ray--tracing operation benchmarks (see \citep{lenstools}). The numbers refer to tests conducted on the XSEDE Stampede cluster. Parallel operations are implemented with \ttt{mpi4py} \citep{mpi4py}, a {\sc python} wrapper of the \ttt{MPI} library \citep{MPI}.}
\label{tbl:3:lt-benchmark}
\end{center}
\end{table}
%
Figure \ref{fig:3:lt-memory} shows the memory load as a function of the runtime for the plane generation and ray--tracing operations for the same test case shown in Table \ref{tbl:3:lt-benchmark}. The Figure shows that, for the considered test case, computer clusters with at least 2\,GB of memory per core are suitable for safely handling the \LT\, operations without exhausting the resources. 

The pipeline products are organized in a hierarchical directory structure whose levels correspond to specifications of $\Lambda$CDM cosmological parameters, choices of $L_b,N_p$, random seeds for the initial conditions $\tilde{\delta}(\bb{k})$ and choices of the lens plane parameters $L_P,\Delta$. Separate directory tree levels are dedicated to the WL products $\kappa,\pmb{\gamma}$. Both single redshift images and shear catalogs can be produced. \LT\, provides an API to initialize, navigate and update the pipeline directory tree in an efficient way, allowing easy retrieval of WL simulation products for further post--processing. For a throughout presentation of \LT, we direct the reader to the code documentation at \url{http://lenstools.rtfd.io}. 
\begin{figure}
\begin{center}
\includegraphics[scale=0.5]{Figures/eps/lt_memory_usage.eps}
\end{center}
\caption{Memory load as a function of runtime for plane generation (black) and ray--tracing operations (black). Each vertical line corresponds to the completion of a $\psi$ plane calculation (black) and a lens crossing during ray--tracing (red).}
\label{fig:3:lt-memory}
\end{figure}
    

%\bibliography{ref}