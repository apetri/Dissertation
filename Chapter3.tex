%%%%%%%%%%%%%%%%%%%%%%%%%%%%%%%%%%%%%%%%%%%%%%%%%%%%%%%%%%%%%%%%%%%%%%%%%%%

\chapter{Numerical Weak Lensing}
%\lhead[\fancyplain{}{\thepage}]{\fancyplain{}{\rightmark}}
 \thispagestyle{plain}
\setlength{\parindent}{10mm}

\label{chp:3}
In this Chapter we describe the relevant numerical methods for simulating WL observations. We use publicly available software to trace the non--linear time evolution of the matter density contrast $\delta(t)$. We then solve the geodesic equation numerically by adding the multiple deflections which light experiences when traveling from sources to observers. The solution of the geodesic equation for a sufficient number of light rays allows to reconstruct the spatial profiles of the WL observables $\kappa,\pmb{\gamma},\omega$. We conclude the Chapter by presenting our ray--tracing software \LT, which we have publicly released for the use of the WL community.

%%%%%%%%%%%%%%%%%%%%%%%%%%%%%%%%%%%%%%%%%%%%%%%%%%%%%%%%%%%%%%%%%%%%%%%%%%%

\section{Cosmological simulations}
The evolution of the matter density contrast $\delta$ is controlled, at linear stage, by equation (\ref{eq:1:delta-lin}). For WL studies, however, $\delta$ becomes too big at the redshifts of interest ($z\sim 1$) for the linear approximation to still be valid and exact solutions of (\ref{eq:1:b-monopole}), (\ref{eq:1:b-dipole}) and (\ref{eq:1:einstein-full}) are required. A popular approach to solve Boltzmann's equation (\ref{eq:1:boltzmann}) for collision--free Dark Matter is the $N$--body method, which proceeds in a discretization of phase space using a large but finite number $N_p$ of particle tracers (see \citep{gadget2}). The particles are placed in a cubical periodic box of comoving size $L_b$ and are assigned initial conditions which correspond to the density contrast at high $z$, for which the linear approximation is still valid. The particle system is then evolved with a Hamiltonian that mimics Newtonian gravitational interactions (PN corrections are neglected under the assumption $L_b\ll c/H$), described by the potential $\Phi$.  

\subsection{Initial conditions}
The starting step of an $N$--body simulation is a configuration of particle positions $\{\bb{x}_i\}$ and velocities $\{\bb{v}_i\}$ at some initial high redshift $z_{\rm in}\gg 1$. The starting particle configuration traces the linear density contrast $\delta$. The particles are initially are arranged in a \textit{glass} pattern and are given positions $\{\bb{x}_i^g\}$ which correspond to a uniform density profile ($\delta\approx0$). The particles are displaced from their position in the glass by a small amount $\bb{d}(\bb{x}_i^g)$, which is chosen so that the new density profile matches an arbitrary input $\delta$ profile. Because mass is conserved by the displacement transformation, we can impose the condition

\begin{equation}
\label{eq:3:masscons}
\rho_m d^3 x^g = \rho_m(1+\delta)d^3x,
\end{equation} 
%
which relates the density contrast $\delta$ to the Jacobian of the displacement transformation, $\bb{x}(\bb{d}) = \bb{x}^g+\bb{d}$, as 

\begin{equation}
\label{eq:3:delta-displ}
1+\delta = \left\vert\mathds{1}_{3\times 3}+\frac{\partial \bb{d}}{\partial \bb{x}}\right\vert^{-1}
\end{equation}
%
Using the matrix identity

\begin{equation}
\label{eq:3:det-identity}
\vert\mathds{1}+\lambda\bb{M}\vert = 1 + \lambda\Tr\bb{M} + O(\lambda^2)
\end{equation}
%
for a generic square matrix $\bb{M}$ and real $\lambda$, and noting that at high redshift we expect $\delta$ and $\bb{d}$ to be small, we obtain a linear relation between the density contrast and the displacement field, which reads 

\begin{equation}
\label{eq:3:delta-displ-real}
\delta = -\nabla\cdot\bb{d}
\end{equation}
%
A possible solution to equation (\ref{eq:3:delta-displ-real}) is best expressed in Fourier space if we assume the displacement to be longitudinal (which is a good assumption since the peculiar velocity field is approximately curl--free)

\begin{equation}
\label{eq:3:delta-displ-fourier}
\tilde{\bb{d}}(\bb{k}) = \frac{i\bb{k}}{k^2}\tilde{\delta}(\bb{k})
\end{equation}
%
Equation (\ref{eq:3:delta-displ-fourier}) takes the name of Zel'dovich approximation (see \citep{ZeldovichWhite} for a review), and it essentially states that the displacement field that corresponds to the input $\delta$ profile is proportional to the gradient of the local gravitational potential. The Fourier coefficients $\tilde{\delta}(\bb{k})$ are random draws from a normal distribution with variance $P_\delta(k,z_{\rm in})$. The linear $\delta$ power spectrum $P_\delta(k,z_{\rm in})$ can be analytically computed with Einstein--Boltzmann software such as \ttt{CAMB} \citep{CAMB}. We assign the initial peculiar velocities $\bb{v}=\dot{\bb{d}}$ in the context of the Zel'dovich approximation using the time derivative of $\delta$. Since we limit ourselves to Dark Matter density perturbations, we can assume a self--similar linear growth model described by the linear growth factor $D(z)$, which appears in equation (\ref{eq:1:growth-diff}). In order to imprint baryon physics in the initial conditions, we adopt a hybrid approach in which we use \ttt{CAMB} to compute the linear matter power spectrum $P^{\rm lin}_\delta(\bb{k},0)$ at the present time, with baryons included. We then scale $P^{\rm lin}_\delta$ back to $z_{\rm in}=100$ using the linear growth factor

\begin{equation}
\label{eq:3:hybrid-power}
P_\delta(\bb{k},z_{\rm in}) = P^{\rm lin}_\delta(\bb{k},0)\left(\frac{D(z_{\rm in})}{D(0)}\right)^2 
\end{equation}
%
This initial condition (which includes baryon effects) is then evolved according to Dark Matter only collision--free dynamics. Random realizations of $\tilde{\delta}$ are drawn from a normal distribution with variance $P_\delta^{\rm lin}$ and the peculiar particle velocities $\bb{v}$ are assigned according to 

\begin{equation}
\label{eq:3:peculiar-initv}
\tilde{\bb{v}}(\bb{k}) = \frac{i\bb{k}}{k^2}\tilde{\delta}(\bb{k})\left(\frac{\dot{z}}{D(z)}\frac{dD(z)}{dz}\right)_{z=z_{\rm in}}
\end{equation}  
%
We used the \ttt{N-GenIC} software add--on to \ttt{Gadget2} \citep{gadget2} in order to generate random realizations of the $\bb{d},\bb{v}$ initial conditions from the linear $\delta$ power spectrum $P^{\rm lin}_\delta(\bb{k},0)$.   

\subsection{Time evolution}
Once generated, the initial conditions specified by equations (\ref{eq:3:delta-displ-fourier}) and (\ref{eq:3:peculiar-initv}) are evolved in time from $z=z_{\rm in}$ until the present redshift $z=0$. Since we consider collision--free Dark Matter, which interacts only via gravitational forces, the Hamiltonian $\mathcal{H}$ of the particle system (ignoring PN corrections, since we are in the limit $L_b\ll c/H$) can be written as 

\begin{equation}
\label{eq:3:collisionless-ham}
\mathcal{H} = \sum_{i=1}^{N_p} \frac{\bb{p}_i^2}{2m_i a(t)^2} + \frac{1}{2}\sum_{i\neq j}m_im_j\varphi(\bb{x}_i-\bb{x}_j)
\end{equation} 
%
We denoted the particle masses as $m_i$, the particle momenta conjugated to the comoving coordinates $\bb{x}_i$ as $\bb{p}_i$ and the pair interaction potential per unit mass as as $\varphi$. If periodic boundary conditions are imposed on the boundary of the simulation box, the interaction potential satisfies the Poisson equation

\begin{equation}
\label{eq:3:part-poisson}
\nabla^2\varphi(\bb{x}) = \frac{4\pi G}{a}\left(\sum_{\bb{n}\in\mathds{Z}^3}\delta_{r_s}(\bb{x}-\bb{n}L_b)-\frac{1}{L_b^3}\right) 
\end{equation}
%
where $\delta_{r_s}$ is the Dirac delta function $\delta^D$ convolved with a softening kernel of scale $r_s$. The softening is introduced because the $N$--body particles are in reality extended objects and the Newtonian interaction potential needs to be smoothed out on interaction scales smaller than $r_s$. In our simulations $r_s$ has been fixed to $r_s\approx 10\,{\rm kpc}/h$. Note that the summation can be dropped if we restrict $\bb{x}$ to be inside the box, but is important in order to enforce the periodic boundary conditions. We can relate $\varphi$ to the gravitational potential in (\ref{eq:1:poisson}) by    

\begin{equation}
\label{eq:3:macro-potential}
\Phi(\bb{x},t) = -\frac{1}{c^2}\sum_{i=1}^{N_p} m_i\varphi(\bb{x}-\bb{x}_i(t))
\end{equation} 
%
We can observe that, inside the simulation box, equation (\ref{eq:3:macro-potential}) leads to 

\begin{equation}
\label{eq:3:macro-potential-poisson}
\nabla^2\Phi(\bb{x},t) = -\frac{4\pi Ga^2}{c^2}\left(\sum_{i=1}^{N_p}m_i\delta_{r_s}\left(a(\bb{x}-\bb{x}_i(t))\right)-\frac{1}{a^3L_b^3}\sum_{i=1}^{N_p}m_i\right)
\end{equation}
%
Note that (\ref{eq:3:macro-potential-poisson}) is essentially the discretized version of (\ref{eq:1:poisson}) for a system made of $N_p$ particles, where gravitational forces are softened on scales below $r_s$. The Hamiltonian equations of motion derived from (\ref{eq:3:collisionless-ham}) can be numerically integrated and yield a trajectory $\bb{x}_i(t)$ for each particle. To preserve the Hamiltonian nature of the time evolution, \citep{gadget2} suggest adopting a Kick--Drift--Kick (KDK) numerical integration scheme. The drift step updates the particle coordinates from their momenta, while the kick updates the momenta using the local force field. The force field calculation requires the solution of (\ref{eq:3:part-poisson}) and a summation over all particle pairs, which leads to an $O(N_p^2)$ time complexity. In the limit of collision--free dynamics, approximate force field calculations can be performed with a significantly lower complexity using the hybrid Tree Particle Mesh (TreePM) approach. The details of the force field calculation, time integration and TreePM implementation can be found in the \ttt{Gadget2} paper \citep{gadget2}. We used the publicly available version of the \ttt{Gadget2} code to perform the $N$--body simulations on which our WL simulations are based. We stored the $N$--body simulation outputs $\{\bb{x}_i(t)\}$ at a discrete set of time steps $\{t_k\}$. We then used these outputs to estimate the potential $\Phi$ necessary from which WL observables $\kappa,\pmb{\gamma}$ can be reconstructed. We describe the numerical details of the WL simulations in the next section.  

%%%%%%%%%%%%%%%%%%%%%%%%%%%%%%%%%%%%%%%%%%%%%%%%%%%%%%%%%%%%%%%%%%%%%%%%%%%

\section{The multi--lens--plane algorithm}

\subsection{Geodesic solver}
In this section we review the algorithm used to solve the light geodesic equation (\ref{eq:2:geodesic-fo-3}). This algorithm allows us to compute the $\pbeta$ (\ref{eq:2:geosol-beta}) and $\bb{A}$ (\ref{eq:2:geosol-jac}) integrals in an efficient and numerically stable fashion. In the remainder of the Chapter we will assume that source galaxies are positioned at fixed longitudinal comoving distance $\chi_s$. A particular light ray is observed at an angular position $\pt$ on the sky due to lensing, but its originating angular position is $\pbeta(\chi_s,\pt)$. $\pt$ and $\pbeta$ are related through equation (\ref{eq:2:geodesic-fo-3}). Numerical integration of (\ref{eq:2:geosol-beta}) is performed dividing the interval $\chi\in[0,\chi_s]$ in $N_l$ equally spaced steps, each of size $\Delta = \chi_s/N_l$ and using a first order explicit method

\begin{equation}
\label{eq:3:int-fo}
\int_0^{\chi_s} f(\chi)d\chi = \Delta\sum_{i=1}^{N_l}f(\chi_k) + O\left(\frac{1}{N_l}\right)
\end{equation} 
%
\begin{equation}
\label{eq:3:int-steps}
\chi_k = k\Delta
\end{equation}  
%
In this notation, $f$ is a generic function of $\chi$ and can be identified with either $\pbeta$ or $\bb{A}$. Before applying the numerical integration method to (\ref{eq:2:geodesic-fo-3}), it is convenient to rewrite the geodesic equation as an equation for $\pbeta=\xp/\chi$ 

\begin{equation}
\label{eq:3:geodesic-fo-beta-1}
\frac{d^2}{d\chi^2}(\chi\pbeta(\chi)) = \frac{2}{\chi}\nabla_{\pbeta}\Phi(\chi,\pbeta(\chi))
\end{equation} 
%
We promoted the $\xp$ dependency of $\Phi$ to a $\beta$ dependency using $\Phi(\chi,\xp=\chi\pbeta)\rightarrow\Phi(\chi,\pbeta)$. Equation (\ref{eq:3:geodesic-fo-beta-1}) is equivalent to 

\begin{equation}
\label{eq:3:geodesic-fo-beta-2}
\frac{d^2\pbeta(\chi)}{d\chi^2} + \frac{2}{\chi}\frac{d\pbeta(\chi)}{d\chi} - \frac{2}{\chi^2}\nabla_{\pbeta}\Phi(\chi,\pbeta(\chi)) = 0
\end{equation}
%
Now let us consider an intermediate discrete step $k$ and introduce the compact notation 
\begin{equation}
\label{eq:3:compactnotation}
\begin{matrix}
f_k\equiv f(\chi_k) & ; & f'_k\equiv\left.\frac{df}{d\chi}\right\vert_{\chi=\chi_k} & ; & f''_k\equiv\left.\frac{d^2f}{d\chi^2}\right\vert_{\chi=\chi_k}
\end{matrix}
\end{equation}  
%
We define
\begin{equation}
\label{eq:3:alphak}
\palpha_k = \frac{2\Delta}{\chi_k}\nabla_{\pbeta}\Phi(\chi_k,\pbeta_k).
\end{equation}
%
Using the first order finite difference approximations for the $\pbeta$ derivatives

\begin{equation}
\label{eq:3:finitediff}
\begin{matrix}
\pbeta'_k = \frac{\pbeta_{k+1}-\pbeta_{k-1}}{2\Delta} + O(\Delta^2) & ; & \pbeta''_k = \frac{\pbeta_{k+1}+\pbeta_{k-1}-2\pbeta_k}{\Delta^2} + O(\Delta^2),
\end{matrix}
\end{equation}
%
we can rewrite equation (\ref{eq:3:geodesic-fo-beta-2}) as 
\begin{equation}
\label{eq:3:geodesic-fo-beta-disc}
\frac{\pbeta_{k+1}+\pbeta_{k-1}-2\pbeta_k}{\Delta^2} + \frac{\pbeta_{k+1}-\pbeta_{k-1}}{\chi_k\Delta} - \frac{\palpha_k}{\chi_k\Delta} = 0
\end{equation} 
%
Once we solve (\ref{eq:3:geodesic-fo-beta-disc}) for $\pbeta_{k+1}$, we immediately find

\begin{equation}
\label{eq:3:betakp1}
\pbeta_{k+1} = \frac{2\pbeta_k\chi_k-(\chi_k-\Delta)\pbeta_{k-1}+\Delta\palpha_k}{\chi_k+\Delta}
\end{equation}  
%
The expression (\ref{eq:3:betakp1}) has a simple physical interpretation that we can understand by looking at the diagram in Figure \ref{fig:3:multi-lens-plane}.
\begin{figure}
\begin{center}
\includegraphics[scale=0.7]{Figures/pdf/multi_lens_plane.pdf}
\end{center}
\caption{Multi--lens--plane algorithm schematics: the trajectory of a single light ray from the observer to the source at $\chi_s$ is shown in red as it undergoes the multiple deflections caused by the lensing effect.}
\label{fig:3:multi-lens-plane}
\end{figure}
%
If we want to calculate the angular position of a light ray at the $k+1$-th step, we need to know its position at the two previous steps $k,k-1$. Simple geometric arguments, combined with the small deflection assumption, tell us that 

\begin{equation}
\label{eq:3:betakp1-2}
\pbeta_{k+1} = \frac{1}{\chi_{k+1}}\left[(\chi\pbeta)_{k} + \left(\frac{(\chi\pbeta)_{k}-(\chi\pbeta)_{k-1}}{\chi_k-\chi_{k-1}}+\palpha_k\right)(\chi_{k+1}-\chi_k) \right]
\end{equation}
%
Note that equations (\ref{eq:3:betakp1}) and (\ref{eq:3:betakp1-2}) are equivalent if the steps are equally spaced, which is the case in our integration scheme defined by $\chi_k=k\Delta$. This equivalence tells us that the quantity $\palpha_k$, which is proportional to the gradient of the potential as stated in (\ref{eq:3:alphak}), is the deflection angle that a light ray experiences upon impact with a two dimensional lens plane of thickness $\Delta$ positioned at a longitudinal distance $\chi_k$. This is why the procedure of solving (\ref{eq:3:geodesic-fo-beta-2}) in discrete $\chi$ steps takes the name of multi--lens--plane algorithm \citep{RayTracingJain,RayTracingHartlap}. The solution is obtained by summing up a discrete set of trajectory deflections $\palpha_k$ which are caused by a discrete set of two dimensional lens planes. Each plane is characterized by a lensing potential which is the three dimensional gravitational potential $\Phi$ projected along the longitudinal direction. We observe that equation (\ref{eq:3:alphak}) is essentially the discrete longitudinal integral of $\nabla_\perp\Phi$ performed with a step of size $\Delta$. Using the initial conditions

\begin{equation}
\label{eq:3:initcond}
\pbeta_0 = \pbeta_1 = \pt 
\end{equation}    
%
we can use the recurrence relation (\ref{eq:3:betakp1}) to compute the light ray trajectory from the observed to the starting angle $\pbeta_s$. It turns out that, because the coefficient that multiplies $\pbeta_k$ in (\ref{eq:3:betakp1}), $2\chi_k/(\chi_k+\Delta)$ is usually bigger than 1, this explicit method of solution leads to roundoff errors which blow up exponentially in $k$. To keep the accuracy of the geodesic solver under control we recast (\ref{eq:3:betakp1}) in a slightly different form by defining $\delta\pbeta_k \equiv \pbeta_k-\pbeta_{k-1}$. It is straightforward to show that

\begin{equation}
\label{eq:3:betak-sum}
\pbeta_k = \pt + \sum_{i=1}^k\delta\pbeta_i
\end{equation}
%
\begin{equation}
\label{eq:3:betakp1-delta}
\delta\pbeta_{k+1} = \left(\frac{\chi_k-\Delta}{\chi_k+\Delta}\right)\delta\pbeta_k + \left(\frac{\Delta}{\chi_k+\Delta}\right)\palpha_k
\end{equation} 
%
It turns out that, because the coefficients that multiply $\delta\pbeta,\palpha$ are smaller than 1, (\ref{eq:3:betak-sum}) and (\ref{eq:3:betakp1-delta}) offer a more accurate numerical solution to the geodesic equation (\ref{eq:3:geodesic-fo-beta-2}). We can solve the geodesic equation for light rays with different initial conditions $\pt$, and study how the solution varies with $\pt$. This allows to translate the recurrence relations (\ref{eq:3:betak-sum}), (\ref{eq:3:betakp1-delta}) into recurrence relations for the lensing Jacobian $\bb{A}$. Observing that 

\begin{equation}
\label{eq:3:Tk}
\frac{\partial (\alpha_i)_k}{\partial \theta_j} = \frac{2\Delta}{\chi_k}\partial_{\beta_i}\partial_{\beta_l}\Phi(\chi_k,\pbeta_k)\frac{\partial (\beta_l)_k}{\partial\theta_j},
\end{equation}
%
we define the projected tidal field 

\begin{equation}
\label{eq:3:tidal-proj}
\bb{T}_k = 2\chi_k\Delta\bb{T}^\Phi(\chi_k,\pbeta_k). 
\end{equation}
%
The recurrence relations for the Jacobian $\bb{A}$ can then be written as

\begin{equation}
\label{eq:3:jack-sum}
\bb{A}_k = \mathds{1}_{2\times 2} + \sum_{i=1}^k\delta\bb{A}_i
\end{equation}
%
\begin{equation}
\label{eq:3:jackp1-delta}
\delta\bb{A}_{k+1} = \left(\frac{\chi_k-\Delta}{\chi_k+\Delta}\right)\delta\bb{A}_k + \left(\frac{\Delta}{\chi_k+\Delta}\right)\bb{T}_k\bb{A}_k
\end{equation} 
%
The recurrence relations (\ref{eq:3:jack-sum}),(\ref{eq:3:jackp1-delta}) are used to estimate the WL quantities $\kappa_s,\pmb{\gamma}_s$ at an arbitrary angle $\pt$ on the sky in $O(N_l)$ time. The set of discrete deflections $\palpha_k$ and tidal distortions $\bb{T}_k$ are calculated from the potential $\Phi$. In the next sub--section we will describe the numerical methods necessary to solve the Poisson equation (\ref{eq:2:poisson}) that relates the potential $\Phi$ to the matter density contrast $\delta$. 

\subsection{Poisson solver}
The ray deflections and tidal distortions experienced after each lens crossing are determined by the density fluctuations which are responsible for the WL effect. We define the two--dimensional projected potential $\psi$ for a lens plane centered at comoving distance $\chi$ with thickness $\Delta$ as

\begin{equation}
\label{eq:3:projected-phi}
\psi(\chi,\pbeta) = \frac{2}{\chi}\int_{\chi-\Delta/2}^{\chi+\Delta/2}d\chi' \Phi(\chi',\pbeta)
\end{equation} 
%
Using the definition in (\ref{eq:3:projected-phi}), we obtain expressions for the deflections and tidal distortions in terms of $\psi$

\begin{equation}
\label{eq:3:alphak-psi}
\palpha_k = \nabla_{\pbeta}\psi(\chi_k,\pbeta_k)
\end{equation}
%
\begin{equation}
\label{eq:3:Tk-psi}
\bb{T}_k = \nabla_{\pbeta}\nabla^T_{\pbeta}\psi(\chi_k,\pbeta_k)
\end{equation}
%
Inserting (\ref{eq:3:projected-phi}) into the Poisson equation (\ref{eq:2:poisson}) we observe that $\psi$ itself satisfies a Poisson--like equation

\begin{equation}
\label{eq:3:poisson-psi-1}
\nabla^2_{\pbeta}\psi(\chi,\pbeta) = \frac{2}{\chi}\int_{\chi-\Delta/2}^{\chi+\Delta/2}d\chi' \chi'^2\left(\nabla^2-\frac{\partial^2}{\partial \chi'^2}\right)\Phi(\chi',\chi'\pbeta)
\end{equation}
%
In approximating $\nabla^2_{\pbeta}\approx \chi^2(\nabla^2-\partial_\chi^2)$ we made an assumption of small $\Delta$, so that we can neglect the time evolution of $\Phi$ within the lens. If $\Delta$ is small we can also treat the $\partial^2_\chi$ term in the integral as a boundary term, which vanishes when appropriate boundary conditions for the Poisson equation are imposed (we can choose periodic boundary conditions as an example). With the help of (\ref{eq:2:poisson}) we obtain

\begin{equation}
\label{eq:3:poisson-psi-2}
\nabla^2_{\pbeta}\psi(\chi,\pbeta) = -\sigma(\chi,\pbeta)  
\end{equation} 

\begin{equation}
\label{eq:3:lens-sigma}
\sigma(\chi,\pbeta) = \frac{8\pi G \chi a(\chi)^2\Delta}{c^2}\bar{\rho}_m(\chi)\delta(\chi,\chi\pbeta) = \frac{3H_0^2\Omega_m\chi\Delta}{c^2 a(\chi)}\delta(\chi,\chi\pbeta)
\end{equation}
%
The dimensionless surface density $\sigma$ which appears in (\ref{eq:3:lens-sigma}) can be estimated from the outputs of $N$--body simulations using a particle number count histogram which measures the density contrast $\delta$. The $N$--body outputs consist in a list of $N_p$ particle positions $\{(x_p,y_p,z_p)\}$ computed at times $t(\chi_k)$. Let us assume without loss of generality that $z$ is the longitudinal direction and $(x,y)$ are the transverse coordinates. We divide the lens plane in a two--dimensional regularly spaced grid $\{(x_i,y_i)\}$ in the transverse direction. The grid has $L_P$ pixels per side, each of comoving size size $L_b/L_P$. We assign to each pixel on the grid a particle number count

\begin{equation}
\label{eq:3:number-hist}
n(\chi, \pbeta_i) = \sum_{p=1}^{N_p}w_n(\bb{x}_p,\chi,\pbeta_i)
\end{equation}  
%
where 

\begin{equation}
\label{eq:3:number-kernel}
w_n(\bb{x}_p,\chi,\pbeta_i) = 
\begin{cases}
1 \,\,\,\, {\rm if} \,\,\,\, (x_p,y_p) \,\, {\rm in} \,\, \chi\pbeta_i \, , \, z_p\in [\chi-\Delta/2,\chi+\Delta/2] \\
0 \,\,\,\, {\rm otherwise}
\end{cases}
\end{equation}
%
We then estimate the density contrast $\delta$ at each grid pixel from the histogram as

\begin{equation}
\label{eq:3:delta-hist}
\delta(\chi,\chi\pbeta_i) = \frac{n(\chi,\pbeta_i)L_bL_P^2 }{\Delta N_p} - 1
\end{equation}
%
We assigned the same mass $m=\Omega_m\rho_cL_b^3/N_p$ to all the particles in the simulation. Once the density contrast is estimated from the $N$--body outputs, the two dimensional Poisson equation (\ref{eq:3:poisson-psi-2}) can be solved on the regular transverse grid, at each of the discrete time steps $\chi_k$. If we impose periodic boundary conditions on the edges of the lens plane, an efficient solution to (\ref{eq:3:poisson-psi-2}) can be obtained using the FFTs of $\psi$ and $\sigma$. Note that, because both of these quantities are real, a real FFT is sufficient. Inverting the laplacian operator in Fourier space yields the relation

\begin{equation}
\label{eq:3:poisson-sol-fft}
\tilde{\psi}(\chi_k,\pell) = \tilde{\sigma}(\chi_k,\pell)\frac{e^{-\ell^2\theta_G^2/2}}{\ell^2}
\end{equation} 
%
We applied a Gaussian smoothing smoothing factor $e^{-\ell^2\theta_G^2/2}$ to the solution (\ref{eq:3:poisson-sol-fft}) in order to suppress sub--pixel particle shot noise. We chose $\theta_G$ to be the angular size of one lens pixel in real space. The time complexity of the potential calculation from the $N$--body outputs is dominated by the Poisson solver \citep{lenstools}, which has a runtime of $O(L_P^2\log L_P)$. Figure \ref{fig:3:lens} shows an example lens (density and potential) plane based on equations (\ref{eq:3:delta-hist}), (\ref{eq:3:poisson-sol-fft}). 

\begin{figure}
\begin{center}
\includegraphics[scale=0.4]{Figures/pdf/lens_plane.pdf}
\end{center}
\caption{Dimensionless density $\sigma$ (left) and corresponding potential $\psi$ for a lens plane at $z_l=0.7$, cut from a $N_p=512^3, L_b=240\,{\rm Mpc}/h$ $N$--body simulation.}
\label{fig:3:lens}
\end{figure}

\subsection{Cosmic variance sampling}
\label{sec:3:sampling}
The multi--lens--plane integration scheme for equation (\ref{eq:2:geosol-jac}) suggests a way of producing multiple WL image realizations starting from a single $N$--body simulation. This is possible thanks to the fact that the size of the box $L_b$ can be chosen to be big enough so that the field of view spanned by the observed ray positions $\pt$ covers the simulation box only partially, $\chi_k\theta<L_b$. Periodic shifts of the lens planes along directions perpendicular to the line of sight yield different lenses with identical statistical properties, and lead to different realizations of $\kappa,\pmb{\gamma}$ images. For each realization, the lens system is constructed according to the following procedure:

\begin{itemize}
\item Consider a discrete step $\chi_k$ and choose a random $N$--body simulation among a set of $N_s$ independent simulations ($N_s=1$ if only one $N$--body simulation has been run)
\item Choose a random direction between $(\h{x},\h{y},\h{z})$ to be the longitudinal direction. The other two directions will be the transverse coordinates $\xp$
\item Cut a random slice of size $\Delta$ from the $N$--body output at $t(\chi_k)$, along the chosen longitudinal direction
\item Calculate the surface density contrast $\sigma_k$ on the slice and solve the Poisson equation (\ref{eq:3:poisson-psi-2})
\item Periodically shift the lens along the transverse directions by a random amount
\item Repeat the steps for the next lens plane at distance $\chi_{k+1}$
\end{itemize}
%
We follow this prescription to recycle the outputs of $N_s$ independent $N$--body simulations and to produce $N_r\gg N_s$ realizations of WL observables. These simulated WL ensembles can be used to estimate the estimator scatters caused by cosmic variance, as well as estimator means. Because $N_r$ is bigger than $N_s$, these WL realizations are pseudo--independent, but can be treated as effectively independent if $N_r$ is not too large. This approximate independence issue, along with its implications on WL observation analysis, has been investigated in \citep{PetriVariance} and will be one of the topics in Chapter \ref{chp:5}.  

%%%%%%%%%%%%%%%%%%%%%%%%%%%%%%%%%%%%%%%%%%%%%%%%%%%%%%%%%%%%%%%%%%%%%%%%%%%

\section{Approximate methods}
In this section we describe the numerical implementation of the approximate methods shown in (\ref{eq:2:kappa-1}), (\ref{eq:2:kappa-2-ll}) and (\ref{eq:2:kappa-2-gp}). These methods provide us with a recipe to compute the Born contribution and first post--Born corrections to the convergence $\kappa$ as line--of--sight integrals on the unperturbed ray trajectories. 

\subsection{Born approximation}
The Born contribution to $\kappa$ for sources at distance $\chi_s$ involves a single integral over $\chi$ and can be readily obtained using the first order method in (\ref{eq:3:int-fo}). At $O(\Delta)$ precision we can write 

\begin{equation}
\label{eq:3:kappa-fo-num-1}
\kappa^{(1)}_s(\pt) = -\Delta\sum_{k=1}^{N_l} W_{ks}\chi_k\nabla^2_\perp\Phi(\chi_k,\chi_k\pt)
\end{equation} 
%
where we introduced the compact notation $W_{kk'}=1-\chi_k/\chi_{k'}$. Using the relations (\ref{eq:3:projected-phi}) and (\ref{eq:3:poisson-psi-2}), we can relate the first order convergence $\kappa^{(1)}$ to the discrete set of dimensionless lens densities $\{\sigma_k\equiv\sigma(\chi_k,\chi_k\pt)\}$ as

\begin{equation}
\label{eq:3:kappa-fo-num-2}
\kappa^{(1)}_s = \frac{1}{2}\sum_{k=1}^{N_l} \sigma_k W_{ks}
\end{equation}
%
Note that not only the Born--approximated convergence can be efficiently computed in $O(N_l)$ time, but such approximate approach does not even require knowledge of the solution to the Poisson equation (\ref{eq:3:poisson-psi-2}). At linear order in the potential $\Phi$, the shear field $\pmb{\gamma}$ can be calculated from the Born--approximated $\kappa$ via the use of the KS relation (\ref{eq:2:gamma-c-ks}).  

\begin{figure}
\begin{center}
\includegraphics[scale=0.4]{Figures/pdf/csample.pdf}
\end{center}
\caption{Sample $\kappa$ reconstruction from one $N$--body simulation with $L_b=260\,{\rm Mpc}/h$ and $N_p=512^3$. The lens planes have a thickness of $\Delta=L_b/3$ and are resolved with $L_P^2=4096^2$ pixels. The $\kappa$ maps are reconstructed with $2048^2$ light rays arranged in a regular grid. The source galaxies are placed at redshift $z_s=2$. The residuals $\kappa-\kappa^{(1)}$ are dominated by the geodesic term $\kappa^{(2-{\rm gp})}$.}
\label{fig:3:csample}
\end{figure}

\subsection{Post--Born corrections}
The evaluation of the second order corrections to $\kappa$ that appear in equations (\ref{eq:2:kappa-2-ll}) and (\ref{eq:2:kappa-2-gp}) involve two integrals over $\chi$. This computation, if implemented naively, leads to an $O(N_l^2)$ runtime algorithm. When we apply the first order method in (\ref{eq:3:int-fo}) twice we obtain

\begin{equation}
\label{eq:3:kappa-ll-num}
\kappa_s^{(2-{\rm ll})} = -\frac{1}{2}\sum_{k=1}^{N_l}\sum_{m=1}^k W_{ks}W_{mk} \Tr(\bb{T}_m\bb{T}_k)
\end{equation}
%
\begin{equation}
\label{eq:3:kappa-gp-num}
\kappa_s^{(2-{\rm gp})} = \frac{1}{2}\sum_{k=1}^{N_l}\sum_{m=1}^k W_{ks}W_{mk} (\palpha_m\cdot\nabla\sigma_k)
\end{equation}
%
Note that, since we are performing the integrals along unperturbed trajectories, the angular arguments of $\sigma_k,\palpha_k,\bb{T}_k$ are fixed to be $\pbeta_k\equiv\pt$ for each light ray. Note also that the gradient in (\ref{eq:3:kappa-gp-num}) is taken in the angular coordinates. As previously stated, the naive implementation defined by (\ref{eq:3:kappa-ll-num}) and (\ref{eq:3:kappa-gp-num}) leads to an $O(N_l^2)$ runtime, which can be quite inefficient if the number of lenses and light rays is large. We can design a more efficient algorithm, which runs in linear time, if we cache the partial sums 

\begin{equation}
\label{eq:3:cache}
\begin{matrix}
I^{\palpha,0}_k = \sum_{m=1}^k \palpha_m & ; & I^{\palpha,1}_k = \sum_{m=1}^k \chi_m\palpha_m & \\ \\
I^{\bb{T},0}_k = \sum_{m=1}^k \bb{T}_m & ; & I^{\bb{T},1}_k = \sum_{m=1}^k \chi_m\bb{T}_m 
\end{matrix} 
\end{equation} 
%
The cached algorithm runs in linear time, as can be seen in the following relations   

\begin{equation}
\label{eq:3:kappa-ll-num-lin}
\kappa_s^{(2-{\rm ll})} = -\frac{1}{2}\sum_{k=1}^{N_l} W_{ks} \Tr\left[\bb{T}_k\left(I^{\bb{T},0}_k-\frac{I^{\bb{T},1}_k}{\chi_k}\right)\right]
\end{equation}
%
\begin{equation}
\label{eq:3:kappa-gp-num-lin}
\kappa_s^{(2-{\rm gp})} = \frac{1}{2}\sum_{k=1}^{N_l} W_{ks}\nabla\sigma_k\cdot\left(I^{\palpha,0}_k - \frac{I^{\palpha,1}_k}{\chi_k}\right)
\end{equation}
%
Figure \ref{fig:3:csample} shows a sample $\kappa$ reconstruction from one $N$--body simulation, including the full ray--tracing map and a comparison between the residuals $\kappa-\kappa^{(1)}$ and the second order terms $\kappa^{(2-{\rm ll})},\kappa^{(2-{\rm gp})}$. 

%%%%%%%%%%%%%%%%%%%%%%%%%%%%%%%%%%%%%%%%%%%%%%%%%%%%%%%%%%%%%%%%%%%%%%%%%%%

\section{The \LT\, software package}
\label{sec:3:lt}
In this section we present \LT \citep{lenstools}, a {\sc python} software package that we developed in order to efficiently handle the WL operations discussed in this Chapter. \\ \LT\, implements pipeline of operations which allow to produce simulated $\kappa,\pmb{\gamma}$ images starting from a set of $\Lambda$CDM parameters (see Chapter \ref{chp:1}). The sequence of operations in the pipeline is described by the diagram in Figure \ref{fig:3:lt-flow}  
%
\begin{figure}
\begin{center}
\includegraphics[scale=0.5]{Figures/pdf/lt_flow.pdf}
\end{center}
\caption{Scheme of the \LT\, pipeline flow. Vertical arrows are directed from a particular application to its input. Horizontal arrows are directed from the input to the output products.}
\label{fig:3:lt-flow}
\end{figure}
%
The \LT\, pipeline glues together the \ttt{CAMB}, \ttt{N-GenIC} and \ttt{Gadget2} public codes, used in the $N$--body simulations, with {\sc python} code. The $\Phi$ calculations and ray--tracing operations are also implemented in {\sc python}. The solution to the Poisson equation (\ref{eq:3:poisson-psi-2}) can be efficiently found via FFT, which \LT\, performs using the {\sc numpy} FFTPack \citep{scipy}. The ray--tracing operations (\ref{eq:3:betak-sum}), (\ref{eq:3:betakp1-delta}), (\ref{eq:3:jack-sum}), (\ref{eq:3:jackp1-delta}) are also efficiently implemented with {\sc numpy} taking advantage of vectorized linear algebra routines. \LT\, also provides efficient implementations of the second order approximate methods for $\kappa$, which are defined by equations (\ref{eq:3:kappa-fo-num-2}), (\ref{eq:3:kappa-ll-num-lin}) and (\ref{eq:3:kappa-gp-num-lin}). 

Table \ref{tbl:3:lt-benchmark} shows CPU time benchmarks for a test run performed on the XSEDE Stampede computer cluster (see \url{https://portal.xsede.org/tacc-stampede}). The $N_p$ particles in each snapshot are divided between $N_t$ files, which are read in parallel by $N_t$ independent tasks. After the particle counting procedure (\ref{eq:3:number-hist}) is performed by each task on the regular grid, the total surface density (calculated on a plane of $L_P^2$ pixels) is assembled by the master task, which then proceeds with the solution the Poisson equation via FFT according to (\ref{eq:3:poisson-sol-fft}). The $\psi$ outputs are then saved to disk. In a subsequent step, the lensing potential files are read from disk, and the geodesic equation (\ref{eq:3:geodesic-fo-beta-2}) is solved for $N_R$ different observed ray positions $\pt$. This leads to the reconstruction of the WL shear $\pmb{\gamma}$ and convergence $\kappa$ profiles in the field of view spanned by $\pt$. Multiple $\kappa,\pmb{\gamma}$ realizations can be obtained with the sampling procedure described in \S~\ref{sec:3:sampling}. 
%
\begin{table}
\begin{center}
\begin{tabular}{l|c|c|c}
\toprule
{Step} &            Complexity &            Test case &           Runtime \\ \hline \hline
\midrule
\multicolumn{4}{c}{\textbf{Lens plane generation}} \\ \hline
$N$--body input\footnote{Perfect input performance is assumed in the complexity analysis} & $O(N_p/N_t)$  & $N_p=512^3$, $N_t=16$  & 2.10\,s  \\
Density estimation (\ref{eq:3:number-hist})       & $O(N_p/N_t)$   & $N_p=512^3$, $N_t=16$  & 0.20\,s \\
\ttt{MPI} Communication  & $O(L_P^2\log{N_t})$   & $N_t=16$, $L_P=4096$  & 0.76\,s   \\
Poisson solver (\ref{eq:3:poisson-sol-fft})           & $O(L_P^2\log{L_P})$ & $L_P=4096$  &  2.78\,s    \\
Lens plane output & $O(L_P)$ & $L_P=4096$   & 0.04\,s  \\ \hline \hline

\multicolumn{4}{c}{\textbf{Ray tracing}} \\ \hline
Lens plane input &  $O(L_P^2)$ & $L_P=4096$ & 0.32\,s \\
Random plane shift &  $O(L_P)$ & $L_P=4096$ & 0.15\,s \\
$\palpha_k,\bb{T}_k$ calculations (\ref{eq:3:alphak-psi}),(\ref{eq:3:Tk-psi})  &  $O(N_R)$ & $N_R=2048^2$   & 1.54\,s  \\
Tensor products $\bb{T}_k\bb{A}_k$ in (\ref{eq:3:jackp1-delta}) &  $O(N_R)$ & $N_R=2048^2$   &  1.29\,s \\ \hline \hline

\bottomrule
\end{tabular}
\caption{Ray--tracing operation benchmarks (see \citep{lenstools}). The numbers refer to tests conducted on the XSEDE Stampede cluster. Parallel operations are implemented with \ttt{mpi4py} \citep{mpi4py}, a {\sc python} wrapper of the \ttt{MPI} library \citep{MPI}.}
\label{tbl:3:lt-benchmark}
\end{center}
\end{table}
%
Figure \ref{fig:3:lt-memory} shows the memory load as a function of the runtime for the plane generation and ray--tracing operations for the same test case shown in Table \ref{tbl:3:lt-benchmark}. The plot shows that, for the considered test case, computer clusters with at least 2\,GB of memory per core are suitable for safely handling the \LT\, operations (for this test case) without exhausting the resources. 

The pipeline products are organized in a hierarchical directory structure whose levels correspond to specifications of $\Lambda$CDM cosmological parameters, choices of $L_b,N_p$, random seeds for the initial conditions $\tilde{\delta}(\bb{k})$ and choices of the lens plane parameters $L_P,\Delta$. Separate directory tree levels are dedicated to the WL products $\kappa,\pmb{\gamma}$. Both single redshift images and shear catalogs can be produced. \LT\, provides an API to initialize, navigate and update the pipeline directory tree in a clean and efficient way, thus allowing easy retrieval of WL simulation products for further post--processing. For a throughout presentation of \LT, we direct the reader to the code documentation at \url{http://lenstools.rtfd.io}. 
\begin{figure}
\begin{center}
\includegraphics[scale=0.5]{Figures/pdf/lt_memory_usage.pdf}
\end{center}
\caption{Memory load as a function of runtime for plane generation (black) and ray--tracing operations (red). Each vertical line corresponds to the completion of a $\psi$ plane calculation (black) and a lens crossing during ray--tracing (red).}
\label{fig:3:lt-memory}
\end{figure}
    

%\bibliography{ref}