%%%%%%%%%%%%%%%%%%%%%%%%%%%%%%%%%%%%%%%%%%%%%%%%%%%%%%%%%%%%%%%%%%%%%%%%%%%

\chapter{Conclusions}
%\lhead[\fancyplain{}{\thepage}]{\fancyplain{}{\rightmark}}
 \thispagestyle{plain}
\setlength{\parindent}{10mm}
\label{chp:8}

%%%%%%%%%%%%%%%%%%%%%%%%%%%%%%%%%%%%%%%%%%%%%%%%%%%%%%%%%%%%%%%%%%%%%%%%%%%

As a conclusion of this thesis, we give a summary the results we obtained and discuss possible future developments of cosmology with WL. 

\section{Overview of the results}

\subsection{Forward modeling}
Inference of cosmological parameters from WL observations require a forward model that maps the $\Lambda$CDM parameter space onto the space of observations (or features). Although analytical forward models exist for the $\kappa$ power spectrum \citep{Nicaea,Nicaea17,Coyote2}, when considering higher order $\kappa$ features one must rely on numerical simulations. In this work we presented a WL simulation pipeline which is capable of producing multiple realizations of $\kappa$ images in a given cosmology. This pipeline (published in \citep{lenstools}), combined with image feature extraction techniques, provides the forward model capabilities needed in the analysis of WL observations. 

\subsection{Parameter constraints}
The main goal of this thesis was to study the cosmological information carried by higher order $\kappa$ features. We also wanted to see if this new information complements the one already supplied by the angular power spectrum. Although a precise quantification of the additional information depends on the details of the analysis such as the survey area, galaxy distribution in redshift and feature binning choices, we can safely conclude that the higher order statistics considered in this work contain a significant amount of information that quadratic $\kappa$ descriptors ignore. We can see this in \S~\ref{sec:5:constraints} where we examine constraint forecasts on the $(\Omega_m,w_0,\sigma_8)$ parameter triplet: under the assumption that all source galaxies lie at a single redshift, higher $\kappa$ moments, morphological descriptors and peak counts can deliver constraints which are 1.5 to 2 times tighter than the ones delivered by the power spectrum alone. When we use higher order features to constrain cosmology from CFHTLenS data in Chapter \ref{chp:6}, we conclude that, although $w_0$ remains essentially unconstrained, $\kappa$ moments deliver a constraint on $(\Omega_m,\sigma_8)$ which is much tighter than the one provided by the power spectrum alone. Although the CFHTLenS constraints obtained with the $\kappa$ power spectrum and moments are compatible with the ones obtained with Planck, the same is not true for confidence intervals inferred with Minkowski functionals, which are affected by residual uncorrected systematics in the CFHTLenS data. 

\subsection{Noise in simulations}
In this work we observed how higher order statistics tighten the constraints on $\Lambda$CDM parameters. Because forward modeling these statistics is done with the use of numerical simulations, sample variance is introduced in the models which are then used to fit the data. We investigated this issue in Chapter \ref{chp:5}, where we studied parameter error bar degradations due to noise in the covariance matrix. We verified that, when $N_r$ realizations are used to estimate a $N_d\times N_d$ feature covariance matrix, error bars are enlarged by a factor which scales roughly as  $1+O(N_d/N_r)$, with $O((N_d/N_r)^2)$ terms becoming non--negligible for high $N_d$. We indicated dimensionality reduction techniques such as PCA as possible methods to make this issue less severe. We also found that only few $N$--body simulations with boxes of size $L_b=240\,{\rm Mpc}/h$ (which is big enough to cover a $(3.5\,{\rm deg})^2$ field of view at $z_s=2$) are needed for unbiased modeling of feature cosmic variance.  

\subsection{Weak Lensing systematics}
In Chapter \ref{chp:7} we confirmed that a large survey such as LSST has the statistical power to constrain the Dark Energy equation of state $w_0$ to a percent level and hence has the potential to answer the long standing question whether $w_0$ is equal or not to $-1$. With this increased precision in measurements, though, stricter requirements on the accuracy of WL forward models must be enforced in order to avoid bias in parameter constraints. We studied different contamination sources to the WL signal and evaluated their effects on $\Lambda$CDM constraints. We found that, although CCD imperfections such as the tree ring effect and the variation of pixel sizes are a completely negligible effect, the same is not true for atmospheric spurious shear contaminations and photometric redshift errors. We found that, if left uncorrected, both of these systematic effects affect higher order $\kappa$ features and lead to biases with a significance level bigger than $1\sigma$. As a consequence, in upcoming galaxy surveys, these systematics must be either mitigated or modeled and marginalized over. We also concluded that, while the Born approximation is accurate enough to model the $\kappa$ power spectrum, full ray--tracing needs to be employed for higher order $\kappa$ moments.      

\section{Future prospects}

\subsection{Curse of dimensionality}
Accurate detection of structures in feature space is crucial for obtaining better parameter constraints from WL. With the increased area and resolution of future surveys and with the advent of WL tomography, the typical dimensionality of the feature space is expected to increase significantly compared to previous generation experiments. Along with the possibility of tighter confidence intervals, this brings along a series of numerical challenges involved with high dimensionality, the most serious of which involves the estimation of feature covariance matrices. A possible solution to the constraint degradation described in \S~\ref{sec:5:degrade} is to reduce the dimensionality of the feature space while preserving the cosmological information contained in it. Although PCA offers a possibility in this sense, this is not the only way to go. Projections onto feature sub--spaces defined by orthonormal vectors (such as PCA) are not the most general. Non orthonormal projectors might be explored in the future as means of obtaining tighter constraints on $\Lambda$CDM. Moreover, physical insight in the Standard Model of cosmology could lead to scale invariant techniques for dimensionality reduction, thus removing the arbitrarity associated with feature whitening procedures. Another intriguing direction of investigation is to consider non linear techniques such as Locally Linear Embedding \citep{astroMLText}. Although promising, these techniques require much bigger simulated datasets in order to be trained. Better estimators of the feature covariance matrix, which do not suffer from the numerical issues illustrated in \S~\ref{sec:5:degrade}, can also be employed. Shrinkage \citep{ShrinkagePope} is one of such techniques, in which the specification of a theory--motivated target covariance leads to an estimator that, although slightly biased, degrades less severely with $N_d$.    

\subsection{Weak Lensing of the CMB}
Background source galaxies are not only tracers of the WL effect: an intriguing possibility for an independent observable is the CMB \citep{CMBLens}. Background photons that originate from the surface of last scattering situated at $z_*\approx 1100$ undergo lensing from Large Scale Structure as well. The primary CMB temperature anisotropy profile $T(\pt)$ is lensed by LSS, which yields a modified profile $T_{\rm lensed}(\pt)$. Lensing does not change the surface brightness of the sources, but only alters their shapes (see equation (\ref{eq:2:quadrupole})). This consideration, applied to the CMB temperature profile, translates to 

\begin{equation}
\label{eq:8:tlens}
T_{\rm lensed}(\pt) = T(\pt + \nabla \psi_{\rm lens}(\pt)),
\end{equation} 
%
where the lensing potential $\psi_{\rm lens}$ is related to $\kappa$ via the Poisson equation $\nabla^2\psi_{\rm lens}=-2\kappa$. Since the non--lensed CMB temperature $T$ is a Gaussian field, spatial correlations that probe non--Gaussianities in $T_{\rm lensed}$ can be used to estimate $\kappa$ \citep{CMBLens,HuCMBLens}. The same methods described in this work can then be used to extract features from the CMB--estimated $\kappa$ and to infer $\Lambda$CDM parameters. CMB lensing provides a powerful probe for the Standard Model of cosmology when combined with galaxy lensing, because the systematic effects involved in the reconstruction of $\kappa$ are independent in the two cases. \citep{CMBIA} for example proposed a method for cross--correlating galaxy and CMB lensing observations in order to mitigate intrinsic alignment effects. Our \LT\, pipeline can be adapted to study higher order image features and non--Gaussianities in CMB--reconstructed $\kappa$ images (see \citep{PetriCMB} for a first application). Future prospects in this field include the study of the performance of the Born approximation in constructing CMB $\kappa$ maps, both in feature accuracy \citep{CMBCalabrese,CMBPrattenLewis} and parameter constraints. Combination of galaxy and CMB lensing observations have the potential of significantly improving constraints on the Dark Energy equation of state and on the neutrino masses.   


%\bibliography{ref}