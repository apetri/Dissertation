%%%%%%%%%%%%%%%%%%%%%%%%%%%%%%%%%%%%%%%%%%%%%%%%%%%%%%%%%%%%%%%%%%%%%%%%%%%

\chapter{Conclusions}
\lhead[\fancyplain{}{\thepage}]{\fancyplain{}{\rightmark}}
 \thispagestyle{plain}
\setlength{\parindent}{10mm}
\label{chp:8}

%%%%%%%%%%%%%%%%%%%%%%%%%%%%%%%%%%%%%%%%%%%%%%%%%%%%%%%%%%%%%%%%%%%%%%%%%%%

To conclude this work, we give a brief overview of the results we obtained and discuss possible future developments of WL which have not been explored in this thesis. 

\section{Results overview}

\subsection{Forward modeling}
Parameter inference pipelines need the specification of a forward model that maps the $\Lambda$CDM parameter space onto the observation (or feature) space. While several analytical forward model codes exist for the $\kappa$ power spectrum \citep{Nicaea,Coyote2}, the same is not true for higher order $\kappa$ features, such as higher moments and morphological descriptors. In this work we presented a WL pipeline that is capable of simulating multiple realizations of $\kappa$ images in a given $\Lambda$CDM cosmology. This pipeline (published in \citep{lenstools}), combined with image feature extraction techniques, provides the forward model capability in needed in WL analyses. 

\subsection{Parameter constraints}
The main goal of this thesis was to evaluate the amount of complementary cosmological information that higher order $\kappa$ features carry, in addition to the one already supplied by the $\kappa$ power spectrum. Although a precise quantification of this additional information depends on the details of the analysis such as the survey area, galaxy distribution in redshift and the choice of the feature space binning, we can safely conclude that the higher order statistics considered in this work contain a significant amount of information that quadratic $\kappa$ descriptors ignore. We can see this in \S~\ref{sec:5:constraints} where we examine constraint forecasts on the $(\Omega_m,w_0,\sigma_8)$ parameter triplet: under the assumption that all source galaxies lie at a single redshift, higher $\kappa$ moments, morphological descriptors and peak counts can deliver constraints which are 1.5 to 2 times tighter than the ones delivered by the power spectrum alone. When we use higher order features to constrain cosmology from CFHTLenS data in Chapter \ref{chp:6}, we conclude that, although $w_0$ remains essentially unconstrained, $\kappa$ moments deliver a constraint on $(\Omega_m,\sigma_8)$ which is much tighter than the one provided by the power spectrum alone. Although the constraints obtained with the $\kappa$ power spectrum and moments are compatible with the ones obtained with Planck, the same is not true for the constraints obtained with the Minkowski functionals, which seem to be sensitive to residual uncorrected systematics in the CFHTLenS data. 

\subsection{Noise in simulations}
In this work we observed how higher order statistics help in tightening the constraints on $\Lambda$CDM parameters. Because the forward modeling of these statistics require the use of simulations, sampling noise is introduced in the models which are used to fit the data. We investigated these issues in Chapter \ref{chp:5}, where we studied both the constraint degradation due to noise in the covariance matrix and the pseudo--independence of two--dimensional WL image realizations that are drawn from a limited number of three--dimensional $N$--body simulations. We verified that when $N_r$ realizations are used to estimate a $N_d\times N_d$ feature covariance matrix, a $O(N_d/N_r)$ degradation in parameter constraints is introduced, with $((N_d/N_r)^2)$ terms becoming non--negligible for high $N_d$. We found in dimensionality reduction techniques such as PCA a possible mitigation method for this problem. We also found that a small number of $N$--body simulations with boxes of size $L_b=240\,{\rm Mpc}/h$ (which is big enough to cover a $(3.5\,{\rm deg})^2$ field of view at $z_s=2$) is needed in order for feature ensemble means to be unbiased.  

\subsection{Weak Lensing systematics}
In Chapter \ref{chp:8} we have confirmed that a large survey such as LSST is able in principle to constrain the Dark Energy equation of state $w_0$ to a percent level, possibly answering the long standing question whether $w_0$ is equal or not to $-1$. With this increased precision, though, stricter requirements on the accuracy of WL forward models must be enforced, in order to avoid biased parameter constraints. We studied a variety of contamination sources to WL signals and evaluated their effects on $\Lambda$CDM parameter constraints. We found that although CCD imperfections, such as the tree ring effect and the variation of pixel sizes, are a completely negligible effect, the same is not true for atmospheric spurious shear contaminations and  photometric redshift errors. We found that, if left uncorrected, both of these systematic effects are important when looking at higher order $\kappa$ features, they lead to parameter bias which have a significance level bigger than $1\sigma$ and need to mitigated in future surveys. We also concluded that, while the Born approximation is accurate enough to predict the $\kappa$ power spectrum, full ray--tracing must be employed for higher order $\kappa$ moments in order to avoid biased parameter inferences.      

\section{Future prospects}

\subsection{Curse of dimensionality}
Accurate detection of structures in feature space is going to be crucial to obtain better parameter constraints from WL data. With the increased area and resolution of futures surveys and with the advent of lensing tomography, the dimensionality of the typical feature space used in the analysis is expected to be significantly bigger than in the previous generation experiments. Along with the possibility of tighter confidence intervals, this brings along a series of problems involved with high dimensionality, the most serious of which is the estimation of feature covariance matrices. A possible solution to the constraint degradation described in \S~\ref{sec:5:degrade} is to reduce the dimensionality of the feature space while preserving the cosmological information contained in it. Although PCA offers a possibility in this sense, this is not the only way to go. Projections onto feature sub--spaces defined by orthonormal vectors (such as PCA) are not the most general available. Projections based on non orthonormal bases might be explored in the feature as a mean of obtaining better $\Lambda$CDM constraints. Physical insight in the $\Lambda$CDM model could offer dimensionality reduction techniques that offer a more motivated scale selection, instead of the arbitrary one involved in the PCA whitening procedure. Another direction in which one could move is to consider more involved dimensionality non linear reduction techniques such as Locally Linear Embedding. Although promising, these non linear techniques will require much larger simulated datasets in order to be trained. Better estimation of the feature covariance matrix, that do not suffer from the numerical issues illustrated in \S~\ref{sec:5:degrade}, can also be employed. Shrinkage \citep{ShrinkagePope} is one of such techniques, in which the specification of a theory motivated target covariance matrix leads to the construction of a better estimator which degrades less severely with $N_d$.    

\subsection{Weak Lensing of the CMB}
Throughout this work we used background source galaxies as the primary tracers for WL: correlations in shape distortions of nearby galaxies allow us to reconstruct the convergence $\kappa$, on which we focused our analysis and inference for $\Lambda$CDM parameters. Another intriguing possibility for a WL tracer is the CMB \citep{CMBLens}, whose background photons originate from the surface of last scattering situated at $z_*\approx 1100$. The primary CMB temperature anisotropy profile $T(\pt)$ is lensed by the LSS of the universe, yielding a different profile $T_{\rm lensed}(\pt)$. Lensing does not change the surface brightness of a source, but only alter its shape (see equation (\ref{eq:2:quadrupole})), which for the CMB translates to 

\begin{equation}
\label{eq:8:tlens}
T_{\rm lensed}(\pt) = T(\pt + \nabla \psi_{\rm lens}(\pt))
\end{equation} 
%
where the lensing potential $\psi_{\rm lens}$ is related to $\kappa$ via the Poisson equation $\nabla^2\psi_{\rm lens}=\kappa$. Since the non lensed CMB temperature $T$ is a Gaussian field, estimators that probe the non--Gaussianity in $T_{\rm lensed}$ can be used to reconstruct $\kappa$ \citep{CMBLens}. The same methods described in this work can then be used to extract features from CMB estimated $\kappa$ fields and infer $\Lambda$CDM parameters. CMB lensing provides a powerful probe for the Standard Model of cosmology when combined with galaxy lensing, as the systematics involved in reconstructing $\kappa$ are independent in the two cases. \citep{CMBIA} for example proposed a technique to cross--correlate galaxy and CMB lensing data in order to mitigate intrinsic alignment effects. Our \LT\, pipeline can be adapted to study higher order image features and non--Gaussianities in CMB reconstructed $\kappa$ maps (see \citep{PetriCMB} for a first application). Future prospects in this field include the study of the performance of the Born approximation in constructing CMB $\kappa$ maps, both in feature accuracy \citep{CMBCalabrese,CMBPrattenLewis} and parameter constraints. Combination of galaxy and CMB lensing observations have the potential of significantly tightening confidence intervals around the Dark Energy equation of state parameters and the mass of neutrinos.   


\bibliography{ref}