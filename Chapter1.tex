%%%%%%%%%%%%%%%%%%%%%%%%%%%%%%%%%%%%%%%%%%%%%%%%%%%%%%%%%%%%%%%%%%%%%%%%%%%

\chapter{The $\Lambda$CDM cosmological model}
\lhead[\fancyplain{}{\thepage}]{\fancyplain{}{\rightmark}}
 \thispagestyle{plain}
\setlength{\parindent}{10mm}

%%%%%%%%%%%%%%%%%%%%%%%%%%%%%%%%%%%%%%%%%%%%%%%%%%%%%%%%%%%%%%%%%%%%%%%%%%%

\section{The Friedmann-Robertson-Walker model}
In the course of this work, we will assume the Universe to be described by a flat Friedmann-Robertson-Walker (FRW) model with time dependent scale factor $a(t)$ and Hubble parameter $H(t)=\dot{a}(t)/a(t)$. We introduce comoving spatial coordinates $\bb{x}$, centered on an Earth based observer, and we define a spacetime 4--vector $dx^{\mu}=(cdt,d\bb{x})$. We can define a 4--momentum associated with the $x^\mu$ coordinates as $P^\mu=dx^\mu/ds$, with $ds=cd\tau$ (this definition says that $\tau$ can be identified as proper time for massive particles. For the case of photons we can identify $\tau$ as a geodesic line parameter). Assuming an homogeneous and isotropic Universe, the line element can be written as  

\begin{equation}
\label{eq:1:frw}
ds^2 = g_{\mu\nu}dx^\mu dx^\nu = -c^2dt^2 + a(t)^2d\bb{x}^2
\end{equation}
%
Where we introduced the diagonal metric tensor $g_{\mu\nu}$, defined as 

\begin{equation}
\label{eq:1:metric}
g_{\mu\nu}(\bb{x},t) = 
\begin{pmatrix}
-1 & 0 & 0 & 0 \\
0 & a(t)^2 & 0 & 0 \\
0 & 0 & a(t)^2 & 0 \\
0 & 0 & 0 & a(t)^2 
\end{pmatrix}
\end{equation}
%
In the remainder of the Chapter we will use the notation $g\equiv-\vert\bb{g}\vert$ and we will use the metric to raise and lower indexes, i.e. we define $V_\mu = g_{\mu\nu}V^\nu$ for a generic 4--vector $V^\mu$. We also define $g^{\mu\nu}=(\bb{g}^{-1})_{\mu\nu}$.
The contents in the FRW universe are assumed to be perfect fluids, described by spatially uniform mass densities $\rho_{(i)}$ and pressures $\mathcal{P}_{(i)}$. We consider a reference frame in which the stress--energy tensor for species $i$ can be written as 

\begin{equation}
\label{eq:1:stressenergy}
\mathcal{T}_{(i)}^{\mu\nu} = \left(\rho_{(i)}c^2+\mathcal{P}_{(i)}\right)U^\mu U^\nu + \mathcal{P}_{(i)}g^{\mu\nu}
\end{equation} 
%
Here $U^\mu$ is the 4--velocity of a fluid element centered at $x^\mu$. Note that, for a fundamental observer, in absence of perturbations the fluid elements 4--velocity must be consistent the homogeneity assumption 

\begin{equation}
\label{eq:1:4-momentum-unif}
U^\mu = (1,0,0,0)
\end{equation}
%
Note also that the stress--energy tensor must is covariantly conserved

\begin{equation}
\label{eq:1:covconservation}
\nabla_\mu \mathcal{T}^{\mu\nu}_{(i)} = \partial_\mu \mathcal{T}^{\mu\nu}_{(i)} + \Gamma_{\mu\alpha}^\mu \mathcal{T}^{\alpha\nu}_{(i)} + \Gamma^\nu_{\mu\alpha}\mathcal{T}^{\mu\alpha}_{(i)} = 0 
\end{equation} 
%
The only non zero components of the affine connection $\Gamma$ for the FRW metric (\ref{eq:1:frw}) are

\begin{equation}
\label{eq:1:connection-unif}
\Gamma^i_{j0} = {\Gamma^0_{ij}}{a^2} = \frac{H}{c}\delta_{ij}
\end{equation}
%
Combining (\ref{eq:1:connection-unif}) with (\ref{eq:1:stressenergy}), the 0--th component of the conservation condition (\ref{eq:1:covconservation}) reads

\begin{equation}
\label{eq:1:rho-conservation}
\dot{\rho}_{(i)} + 3H\left(\rho_{(i)} + \mathcal{P}_{(i)}/c^2\right) = 0
\end{equation}
%
The metric $g_{\mu\nu}$ has to satisfy Einstein equation with source terms $\mathcal{T}_{\mu\nu}$

\begin{equation}
\label{eq:1:einstein-full}
\mathcal{R}^{\mu\nu}-\frac{1}{2}\mathcal{R}g^{\mu\nu} = \frac{8\pi G}{c^4}\sum_i \mathcal{T}^{\mu\nu}_{(i)}
\end{equation}
%
We indicated as $\mathcal{R}^{\mu\nu}$ the Ricci tensor and as $\mathcal{R}=g_{\mu\nu}\mathcal{R}^{\mu\nu}$ the Ricci scalar. It can be shown (see for example \citep{Dodelson-C1}) that the components of the Ricci tensor for the metric (\ref{eq:1:frw}) are 

\begin{equation}
\label{eq:1:ricci}
\begin{matrix}
\displaystyle \mathcal{R}_{00} = -\frac{3\ddot{a}}{ac^2} & ; & \displaystyle \mathcal{R}_{ij} = \left(\frac{a\ddot{a}+2a^2H^2}{c^2}\right)\delta_{ij} & ; & \displaystyle \mathcal{R} = \frac{6}{c^2}\left(\frac{\ddot{a}}{a}+H^2\right) \\ \\ 
\end{matrix}
\end{equation}
%
With the symmetries in the metric (\ref{eq:1:frw}), the Einstein equations (\ref{eq:1:einstein-full}) have two independent components, which take the name of Friedmann equations 

\begin{equation}
\label{eq:1:friedmann-1}
H^2 = \frac{8\pi G}{3}\sum_i\rho_{(i)}
\end{equation} 
%
\begin{equation}
\label{eq:2:friedmann-2}
\frac{\ddot{a}}{a} = -\frac{4\pi G}{3}\sum_i\left(\rho_{(i)}+\frac{3\mathcal{P}_{(i)}}{c^2}\right)
\end{equation}
%
Once relations between the components pressures and densities are specified, the conservation equation (\ref{eq:1:rho-conservation}) and the Friedmann equations (\ref{eq:1:friedmann-1}), (\ref{eq:2:friedmann-2}) can be solved explicitely to obtain the time dependencies of $a,\rho_{(i)},\mathcal{P}_{(i)}$. Later in the Chapter, we will derive these solutions explicitly for the relevant cases studied in this work.  

\subsection{Distance-redshift relation}
In this paragraph we will summarize the basics of the cosmological redshift effect, which is a direct consequence of the FRW geometry. Consider a source (for example a galaxy) at a comoving distance $\chi_s$ from the observer, which emits light at a frequency $\nu_s$. Due to the Universe expansion parametrized by the scale factor $a$, the wavelength of the light gets stretched as photons travel from the source to the observer. Indicating the observed frequency on Earth as $\nu_0$, we can define a redshift parameter associated with the source

\begin{equation}
\label{eq:1:redshift}
z_s = \frac{\nu_s}{\nu_0}-1
\end{equation}  
%
It can be shown that there is a one--to--one correspondence between redshift and scale factor

\begin{equation}
\label{eq:1:z-a}
z_s = \frac{1}{a(t_s)}-1
\end{equation}
%
Where $t_s$ is the emission time of a photon that reaches Earth at the present time $t_0$, for which we assumed $a(t_0)=1$. The FRW metric (\ref{eq:1:frw}) establishes correspondences between the source redshift $z_s$, the photon emission time $t_s$ and the source distance $\chi_s$. Using the fact that $ds^2=0$ along a photon spacetime trajectory, we can write

\begin{equation}
\label{eq:1:chi-a}
\chi_s = -c\int_{t_0}^{t_s}\frac{dt}{a} = -c\int_{a(t_0)}^{a(t_s)}\frac{da}{a^2H}  
\end{equation}

\begin{equation}
\label{eq:1:t-a}
t_s = \int_{0}^{\chi_s}\frac{a d\chi}{c} = -\int_{a(t_0)}^{a(t_s)}\frac{da}{aH}  
\end{equation}
%
Note that, using (\ref{eq:1:z-a}), the relations (\ref{eq:1:chi-a}), (\ref{eq:1:t-a}) can be rewritten as 

\begin{equation}
\label{eq:1:chi-z}
\chi_s = c\int_0^{z_s}\frac{dz}{H(z)}
\end{equation} 
%
\begin{equation}
\label{eq:1:t-z}
t_s = \int_0^{z_s}\frac{dz}{(1+z)H(z)}
\end{equation} 
%
In practical observations, source redshifts $z_s$ are measured using photometric \citep{LSST} or spectroscopic \citep{DESI} techniques and $\chi_s,t_s$ are then inferred from equations (\ref{eq:1:chi-z}), (\ref{eq:1:t-z}) with the help of Friedmann equation (\ref{eq:1:friedmann-1}), which sets the time dependence of the Hubble parameter $H$. The late universe ($z\ll 3000$) is well described in terms of two components, namely cold matter and dark energy (hence the $\Lambda$CDM denomination), which effects we will explore in the next paragraphs. 

\subsection{Cold dark matter}
In this work we model dark matter at late times as a non--relativistic species of particles with mass $m$. When the equilibrium temperature $T$ is much smaller than $mc^2/k_B$, we can neglect the pressure term in (\ref{eq:1:rho-conservation}) and obtain a scaling for the dark matter mass density $\rho_m$ with $a$

\begin{equation}
\label{eq:1:matter-scaling}
\rho_m(a) = \rho_m(a_0)\left(\frac{a(t_0)}{a}\right)^3 = \rho_m(a_0)(1+z)^3 
\end{equation} 
%
Plugging (\ref{eq:1:matter-scaling}) in the Friedmann equation (\ref{eq:1:friedmann-1}) we can get the scale dependence on time for a pure dark matter Universe

\begin{equation}
\label{eq:1:matter-only-a}
a(t) = a(t_0)\left(\frac{t}{t_0}\right)^{2/3}
\end{equation}

\subsection{Dark energy}
The existence of dark energy was postulated by observational evidence about the accelerated expansion of the universe. Suppose that dark energy can be described as a perfect fluid with density $\rho_\Lambda$ and pressure $\mathcal{P}_\Lambda$ which are related by

\begin{equation}
\label{eq:1:de-eos}
\mathcal{P}_\Lambda = w\rho_\Lambda c^2
\end{equation}
%
In this relation, $w$ takes the name of dark energy \textit{equation of state}. Looking at equation (\ref{eq:2:friedmann-2}) we note that a necessary condition for dark energy to cause $\ddot{a}>0$ is $w<-1/3$. Because of the challenges posed by modeling a fluid with negative $w$ from first principles, dark energy is usually described by a phenomenological, $a$ dependent, equation of state in the form \citep{LinderDE}

\begin{equation}
\label{eq:1:de-eos-running}
w(a) = w_0 + w_a(1-a)
\end{equation}
%
with $w_0,w_a$ constant. With the assumption (\ref{eq:1:de-eos-running}), the conservation equation (\ref{eq:1:rho-conservation}) can be solved for the scale dependency of $\rho_\Lambda$

\begin{equation}
\label{eq:1:de-scaling}
\rho_\Lambda(a) = \rho_\Lambda(a_0)\left(\frac{a_0}{a}\right)^{3(1+w_0+w_a)}e^{3w_a(a-a(t_0))} 
\end{equation}
%
We can consider a few limit cases of (\ref{eq:1:de-scaling}). For $w_0=-1,w_a=0$ the dark energy density $\rho_\Lambda$ does not depend on $a$ and behaves as a cosmological constant. If $w_0=-1$ and $w_a\neq 0$, on the other hand, there is a non--trivial scaling of $\rho_\Lambda$ with $a$. In order for this scaling to reproduce the observational fact that $\rho_\Lambda$ is negligible at recombination ($z\sim 1100$), we have that, if not null, $w_a$ has to be negative. Figure \ref{fig:1:distred} shows the alteration of the distance--redshift relation (\ref{eq:1:chi-z}) due to the presence of dark energy. Using Supernovae IA as standard candles, \citep{PerlmutterNobel} have measured the $\chi(z)$ relation with sufficient precision to establish dark energy as the dominant energy component in the present Universe, earning the Nobel Prize in 2011.  

\begin{figure}
\begin{center}
\includegraphics[scale=0.5]{Figures/eps/distred.eps}
\end{center}
\caption{Distance--redshift relation for two different Universe models, with and without dark energy}
\label{fig:1:distred}
\end{figure}    

%%%%%%%%%%%%%%%%%%%%%%%%%%%%%%%%%%%%%%%%%%%%%%%%%%%%%%%%%%%%%%%%%%%%%%%%%%%%%%%%%%%%%%%%%%%%%%%%%%%%%%

\section{Matter density perturbations}
\label{sec:1:density-pert}
In this section we study the deviations from the homogeneous FRW Universe and describe how scalar perturbations evolve under the effect of gravity. This will be particularly relevant when studying the Gravitational Lensing (GL) effect in the next Chapter, as light ray geodesics will deviate from straight lines in the presence of density inhomogeneities. These geodesic deflections have a tangible effect in observations of distant sources, as observed galaxy shapes show apparent distortions that trace the metric perturbations. We review the basic model that describes density perturbations of collisionless cold dark matter in an expanding Universe. Scalar perturbations to the FRW metric (\ref{eq:1:frw}) can be parametrized in the conformal Newtonian gauge \citep{Dodelson-C4} by two scalar potentials $\Phi,\Psi$ as follows

\begin{equation}
\label{eq:1:confnewt}
ds^2 = -c^2dt^2(1+2\Psi(\bb{x},t)) + a(t)^2d\bb{x}^2(1+2\Phi(\bb{x},t))
\end{equation}
%
For the scope of the present work we can safely ignore vector and tensor perturbations to the FRW metric as their effects are negligible in WL observations. The phase space distribution of dark matter is described in terms of a distribution function $f_m(x^\mu,\bb{P})$, which is defined by the fact that $f_m(x^\mu,\bb{P})g d^3x d^3P$ is the number of particles contained in a phase space volume $d^3x d^3P$. We used the notations $P^\mu=(P^0,\bb{P})$, $d^3 P = dP^x dP^y dP^z$. Note that the momentum dependence of $f_m$ can be expressed in terms of $\bb{P}$ only, since the 4--momentum has to satisfy the constraint

\begin{equation}
\label{eq:1:m-constraint}
g_{\mu\nu}P^\mu P^\nu = -1
\end{equation} 
%
Because the phase space volume element $g d^3x d^3P$ is invariant under coordinate transformations, $f_m$ too must be invariant to conserve the number of particles. If we assume $f_m$ to describe a dark matter fluid at local equilibrium, we know that $f_m$ has to depend only on the invariant energy $e$, defined by \citep{JuttnerCov}

\begin{equation}
\label{eq:1:invariant-e}
e = g_{\mu\nu}P^\mu U^\nu
\end{equation}
%
Where the fluid 4--velocity $U^\mu = (U^0,\bb{U})$ obeys the usual constraint $U^\mu U_\mu=-1$. We can relate the distribution function to the dark matter 4--velocity, stress--energy tensor \citep{JuttnerCov} as

\begin{equation}
\label{eq:1:f-velocity}
\int \frac{d^3 P}{P_0}\sqrt{g} P^\mu f_m = \rho_m U^\mu 
\end{equation}
%
\begin{equation}
\label{eq:1:f-stress}
\int \frac{d^3 P}{P_0}\sqrt{g} P^\mu P^\nu f_m = \mathcal{T}^{\mu\nu} 
\end{equation}
%
For notational simplicity we set $\mathcal{T}\equiv \mathcal{T}_m$. Note that equations (\ref{eq:1:stressenergy}) and (\ref{eq:1:f-stress}) can be manipulated to obtain expressions for the matter density and pressure in terms of $f_m$

\begin{equation}
\label{eq:1:f-rho}
\rho_m = \int \frac{d^3 P}{P_0}\sqrt{g}e^2 f_m
\end{equation}
%
\begin{equation}
\label{eq:1:f-press}
\mathcal{P}_m = \frac{1}{3}\int \frac{d^3 P}{P_0}\sqrt{g}(e^2-1) f_m
\end{equation}
%
Using the expression (\ref{eq:1:f-press}), it is easy to show that $\mathcal{P}_m = O(\bb{U}^2)$ and the pressure can be neglected in the non--relativistic limit, as expected. 
We parametrize the dark matter density as 

\begin{equation}
\label{eq:1:matter-rho}
\rho_m(\bb{x},t) = \bar{\rho}_m(t)(1+\delta(\bb{x},t))
\end{equation}
%
where $\bar{\rho}_m(t)$ is the spatially averaged density and $\delta(\bb{x},t)$ is the spatially dependent density contrast. In the next paragraph, we will use the Boltzmann equation for $f$ to relate the evolution of $\delta,\bb{U}$ at linear stage, in the non--relativistic limit.      

\subsection{Collisionless Boltzmann equation}
In the absence of collisions between dark matter particles, the phase space volume is preserved in the system evolution, and the distribution function must satisfy the source free Boltzmann equation 

\begin{equation}
\label{eq:1:boltzmann}
\frac{df_m(x^\mu,\bb{P})}{ds} = P^\mu\frac{\partial f_m(x^\mu,\bb{P})}{\partial x^\mu} + \frac{d P^i}{ds}\frac{\partial f_m(x^\mu,\bb{P})}{\partial P^i} = 0
\end{equation}
%
The 4--momentum variation rate $dP^i/ds$ is obtained from the equations of motion, i.e. the geodesic equations for the metric (\ref{eq:1:confnewt})

\begin{equation}
\label{eq:1:geodesic}
\frac{dP^\mu}{ds} = -\Gamma^\mu_{\alpha\beta}P^\alpha P^\beta
\end{equation}
%
The collisionless Boltzmann equation (\ref{eq:1:boltzmann}) then becomes

\begin{equation}
\label{eq:1:boltzmann-2}
P^0\partial_0 f_m + P^i\partial_i f_m - \frac{\partial f_m}{\partial P^i}\left(P^0P^0\Gamma_{00}^i + 2\Gamma_{0j}^iP^0P^j + \Gamma_{jk}^i P^jP^k \right) = 0
\end{equation}
%
Equations for $\rho_m,U^\mu$ can be obtained from the $P$--moments of the Boltzmann equation (\ref{eq:1:boltzmann-2}). We can integrate (\ref{eq:1:boltzmann-2}) in $d^3P$ directly, or we can multiply it by $P^j$ and then integrate. To perform the calculations, we will use the expressions

\begin{equation}
\label{eq:1:boltzmann-mom}
\begin{matrix}
\displaystyle \int d^3P P^\mu f_m = \frac{\mathcal{T}^\mu_0}{\sqrt{g}} & ; & \displaystyle \int d^3P P^0 P^i \frac{\partial f_m}{\partial P^j} = \frac{\mathcal{T}^i_j-\delta_{ij}\mathcal{T}^0_0}{\sqrt{g}} \\ \\
\displaystyle \int d^3P P^0P^0 \frac{\partial f_m}{\partial P^i} = \frac{2\rho_m U_i}{\sqrt{g}} & ; & \displaystyle \int d^3P P^iP^j f_m =  O(\bb{U}^2) \\ \\ 
\displaystyle \int d^3P P^0P^i f_m = \frac{\rho_mU^i}{\sqrt{g}} + O(\bb{U}^2) & ; & \displaystyle \int d^3P P^iP^j \frac{\partial f_m}{\partial P^k} = -\frac{\delta_{ki}U^j + \delta_{kj}U^i}{\sqrt{g}} + O(\bb{U}^2)  \\ \\
\displaystyle \int d^3P P^0P^iP^j \frac{\partial f_m}{\partial P^k} = -\frac{\delta_{ki}U^j + \delta_{kj}U^i}{\sqrt{g}} + O(\bb{U}^2) & ; & \displaystyle \int d^3P P^0P^0P^i \frac{\partial f_m}{\partial P^j} = -\frac{\rho_m\delta_{ij}}{\sqrt{g}} + O(\bb{U}^2)
\end{matrix}
\end{equation}
%
In addition to the results (\ref{eq:1:boltzmann-mom}), we will use the approximate expressions for the stress--energy tensor 

\begin{equation}
\label{eq:1:stress-energy-nr}
\begin{matrix}
\mathcal{T}^0_0 = -\rho_m + O(\bb{U}^2) & ; & \mathcal{T}^i_0 = -(1+2\Psi)\rho_m U^i + O(\bb{U}^2)  & ; & \mathcal{T}^i_j = O(\bb{U}^2)
\end{matrix}
\end{equation}
%
Now we can perform an integration in $d^3P$ of equation (\ref{eq:1:boltzmann-2}) and take the non--relativistic limit discarding all $O(\bb{U}^2)$ terms, obtaining

\begin{equation}
\label{eq:1:b-monopole}
\frac{\dot{\rho}_m}{c} + \nabla\cdot[(1+2\Psi)\rho_m\bb{U}] - \rho_m[\partial_t\log \sqrt{g} +  2(U_i\Gamma^i_{00} - 2\Gamma^i_{0i}) + (1+2\Psi)(U^i\Gamma_{ij}^j+U^i\Gamma^i_{jj})] = 0
\end{equation}
%
We can also multiply (\ref{eq:1:boltzmann-2}) by $P^j$ and integrate, taking again the non--relativistic limit. The integration leads to 

\begin{equation}
\label{eq:1:b-dipole}
\partial_t(\rho_m U^j) - \rho_m (U^j\partial_t\log \sqrt{g} - c\Gamma^j_{00} - 8c\Gamma_{0i}^jU^i) = 0 
\end{equation}
%
Although the system of equations (\ref{eq:1:b-monopole}) and (\ref{eq:1:b-dipole}) can be closed with the help of Einstein equation (\ref{eq:1:einstein-full}), its exact solution is complicated to obtain because of the non--linearity of the system, and usually numerical methods have to be employed. In the limit in which the perturbations are still at linear stage, i.e. the perturbation quantities are small, we can trust the linearized version of (\ref{eq:1:b-monopole}), (\ref{eq:1:b-dipole}). We can use the linear expression for the affine connection $\Gamma$

\begin{equation}
\label{eq:1:connection}
\begin{matrix}
\Gamma_{00}^0 = \dot{\Psi}/c & ; & \Gamma_{0i}^0=\Gamma_{i0}^0 = \partial_i\Psi \\  
\Gamma_{ij}^0 = [H+2H(\Phi-\Psi)+\dot{\Phi}]a^2\delta_{ij}/c & ; & \Gamma_{00}^i = \partial_i \Psi/a^2 \\
\Gamma^i_{j0} = \Gamma^i_{0j} = (H+\dot{\Phi})\delta_{ij}/c & ; & \Gamma_{jk}^i = (\delta_{ij}\partial_k+\delta_{ik}\partial_j-\delta_{jk}\partial_i)\Phi\\
\end{matrix}
\end{equation}
%
which, plugged in (\ref{eq:1:b-monopole}), (\ref{eq:1:b-dipole}) gives 

\begin{equation}
\label{eq:1:b-monopole-lin}
\dot{\delta} + c \nabla\cdot \bb{U} + 3\dot{\Phi} - \dot{\Psi} = 0
\end{equation}

\begin{equation}
\label{eq:1:b-dipole-lin}
\partial_t(\bar{\rho}_m\bb{U}) + 5H\bar{\rho}_m\bb{U} + \frac{c\nabla\Psi}{a^2} = 0
\end{equation}
%
In this derivation we used the fact that, in the non--relativistic limit $\partial_t\bar{\rho}_m + 3H\bar{\rho}_m=0$, which can also be deducted from the $O(1)$ terms in equation (\ref{eq:1:b-monopole}). We observe that, if one ignores the $\Phi,\Psi$ terms in (\ref{eq:1:b-monopole-lin}), this relation is a continuity equation that describes mass conservation. In this fashion, $\bb{v}=c\bb{U}$ can be identified as the peculiar velocity $\dot{\bb{x}}$ of a fluid element, on top of the Universe expansion. 

\subsection{Einstein equation}
The system composed by the linear equations for $\delta$ and $\bb{U}$ (\ref{eq:1:b-monopole-lin}), (\ref{eq:1:b-dipole-lin}) can be closed making use of the Einstein equation, which relates the potentials $\Phi,\Psi$ to the components of the stress--energy tensor. Since we limit our study to scalar perturbations, there are only two independent components of the Einstein equation that we need to consider. WL physics is dominated by the late time behavior of density perturbations, and hence we will ignore relativistic particles focus on cold matter only. Under this assumption the $00,0i$ and $ij$ components of the linearized Einstein equation (\ref{eq:1:einstein-full}) become respectively (see \citep{Dodelson-C5})

\begin{equation}
\label{eq:1:einstein-00}
\nabla^2\Phi +\frac{3a^2}{c^2}(H^2\Psi-H\dot{\Phi}) = -\frac{4\pi Ga^2\bar{\rho}_m\delta}{c^2}
\end{equation}

\begin{equation}
\label{eq:1:einstein-0i}
\nabla(\dot{\Phi}-H\Psi) = \frac{4\pi Ga^2\bar{\rho}_m\bb{v}}{c^2}
\end{equation}

\begin{equation}
\label{eq:1:einstein-ij}
\nabla^2(\Phi+\Psi) = 0
\end{equation}
%
A few considerations are in order here. First of all, the terms in (\ref{eq:1:einstein-00}) which contain powers of $aH$ are sub--dominant for the WL case of interest, as the laplacian term will always dominate for modes with wavenumber $k$ well inside the Hubble horizon $kc\gg aH$ (see \citep{PNLensing} for a discussion of the importance of PN terms). We can hence drop these terms from (\ref{eq:1:einstein-00}), which then reduces to a Poisson--like equation 

\begin{equation}
\label{eq:1:poisson}
\nabla^2\Phi(\bb{x},t) = -\frac{4\pi Ga(t)^2}{c^2}\bar{\rho}_m(t)\delta(\bb{x},t)
\end{equation}
%
Moreover, since equation (\ref{eq:1:einstein-ij}) comes from the traceless part of the spatial Einstein equation, it is sourced by anisotropic stresses in the matter components. Because such stresses are proportional to the quadrupole of the momentum distributions, which are negligible for non relativistic species, they can be safely neglected when studying WL. We will then use (\ref{eq:1:einstein-ij}) to conclude $\Psi=-\Phi$, since we assume no singularities in $\Psi,\Phi$.  

\subsection{Linear growth factor}
The Poisson equation (\ref{eq:1:poisson}) allows us to obtain a linear equation for the density contrast $\delta$. We combine the time derivative of (\ref{eq:1:b-monopole-lin}) with the divergence of (\ref{eq:1:b-dipole-lin}) and we ignore the terms proportional to $\dot{\Psi},\dot{\Phi}$ (which give rise to PN corrections). After a few manipulations we get 

\begin{equation}
\label{eq:1:delta-lin}
\ddot{\delta} + 2H\dot{\delta} - 4\pi G\bar{\rho}_m\delta = 0  
\end{equation} 
%
Because of the linearity of equations (\ref{eq:1:b-monopole-lin}) and (\ref{eq:1:b-dipole-lin}), each Fourier mode $\tilde{\delta}(\bb{k},t)$ evolves independently in time. Moreover, in absence of pressure terms (which would bring in terms proportional to $\nabla^2\delta$), the perturbations $\delta$ grow self--similarly

\begin{equation}
\label{eq:1:selfsim}
\tilde{\delta}(\bb{k},t) = D(t)\tilde{\delta}(\bb{k},0)
\end{equation}
%
Where the linear growth factor $D$ does not depend on the spatial scale. Equation (\ref{eq:1:delta-lin}) can be converted in an equation for the linear growth factor $D$ with the use of the time-redshift relation (\ref{eq:1:t-z}) and the Friedmann equations (\ref{eq:1:friedmann-1}), (\ref{eq:2:friedmann-2}). After a few manipulations we get

\begin{equation}
\label{eq:1:growth-diff}
\frac{d^2D(z)}{dz^2} + \frac{4\pi G}{3}\left(\frac{\bar{\rho}_m(z)+\rho_\Lambda(z)[1+3w(z)]}{(1+z)H(z)^2}\right)\frac{dD(z)}{dz}-\frac{8\pi G\Omega_m(z)}{H(z)^2(1+z)^2}D(z)= 0
\end{equation}
%
\begin{figure}
\begin{center}
\includegraphics[scale=0.5]{Figures/eps/growth.eps}
\end{center}
\caption{Linear growth factor $D(z)$ obtained solving (\ref{eq:1:growth-diff}) for 4 different $\Lambda$CDM cosmologies. The initial condition has been set for a unit density perturbation at $z=1000$, namely $D(1000)=1, \dot{D}(1000)=0$.}
\label{fig:1:growth}
\end{figure}
%
In the limiting case of a pure dark matter universe ($\rho_\Lambda=0$), (\ref{eq:1:growth-diff}) reduces to

\begin{equation}
\label{eq:1:growth-diff-cold}
\frac{d^2D(z)}{dz^2} + \left(\frac{1}{2(1+z)}\right)\frac{dD(z)}{dz}-\frac{3D(z)}{(1+z)^2} = 0
\end{equation}
% 
Which admits a solution $D(z)\propto (1+z)^{-1}=a$. Figure \ref{fig:1:growth} shows the evolution of the linear growth factor $D$ with redshift for different combinations of the $\Lambda$CDM parameters. 

%%%%%%%%%%%%%%%%%%%%%%%%%%%%%%%%%%%%%%%%%%%%%%%%%%%%%%%%%%%%%

\section{$\Lambda$CDM cosmological parameters}
The main goal of the research presented in this work is studying how WL observations can be used to constrain the free parameters the describe the $\Lambda$CDM universe. In the conclusion of this Chapter we present the parametrization we will use consistently throughout this dissertation writeup. The present day Hubble parameter $H_0\equiv H(t_0)$ is expressed in terms of the dimensionless number $h$ by

\begin{equation}
\label{eq:1:hubble-present}
H_0 = 100h\,{\rm km}\,{\rm s}^{-1}\,{\rm Mpc}^{-1}
\end{equation} 
%
The density of the components that source the Einstein equation are usually quoted in the literature in terms of their ratios with the present critical density $\rho_c=3H_0^2/8\pi G$. We define

\begin{equation}
\label{eq:1:omega-def}
\Omega_i = \frac{8\pi G\rho_i(t_0)}{3H_0^2}
\end{equation}
%
In addition to dark matter and dark energy, the present universe contain a significant fraction of baryonic matter ($\Omega_b\approx \Omega_m/6$), whose physics is more complicated to model than the one of cold matter, as the Boltzmann equation for baryons contains pressure terms and collisional terms. We will ignore baryonic physics in the simulations on which this work is based, although investigation of baryonic physics in WL and in cosmology is an active topic of research. The initial conditions for the density inhomogeneities described in \S~\ref{sec:1:density-pert} are believed to be generated by quantum perturbations during an early epoch of accelerated expansion called inflation (\citep{Inflation}). Inflation is believed to generate random Gaussian, statistically isotropic, initial conditions that are nearly scale invariant 

\begin{equation}
\label{eq:1:initial-cond}
\langle\tilde{\delta}(\bb{k},z_{\rm in})\tilde{\delta}^*(\bb{k}',z_{\rm in})\rangle = (2\pi)^3 P_\delta(k,z_{\rm in})\delta^D(\bb{k}-\bb{k}')
\end{equation}
%
\begin{equation}
\label{eq:1:initial-ps}
P_\delta(k,z_{\rm in}) = \frac{A^2_s}{k^3}\left(\frac{k}{k_0}\right)^{n_s-1}
\end{equation}
%
Here $n_s$ is a parameter that describes the deviation from scale invariance ($n_s=1$ correspond to scale invariant initial conditions). The overall normalization of the initial density perturbations $A_s$ is usually expressed in terms of an equivalent parameter, $\sigma_8$, defined as 

\begin{equation}
\label{eq:1:sigma8}
\sigma_8 = \int \frac{d^3k}{(2\pi)^3} P^{\rm lin}_\delta(k,z=0)e^{-k^2 r_8^2}
\end{equation} 
%
The notation in equation (\ref{eq:1:sigma8}) means that $P_\delta^{\rm lin}$ is calculated from $P_\delta(k,z_{\rm in})$ using linear evolution. $\sigma_8$ is defined from the variance of the linearly evolved density contrast smoothed with a Gaussian window of size $r_8=8\,{\rm Mpc}/h$. 

\begin{table}
\begin{center}
\begin{tabular}[h]{c|c|c}

\textbf{Parameter} & \textbf{Planck 2015} & \textbf{Fiducial} \\ \hline 

$h$ & & \\
$\Omega_m$ & & \\
$\Omega_\Lambda$ & & \\
$\Omega_b$ & & \\
$w_0$ & & \\
$w_a$ & & \\
$\sigma_8$ & & \\
$n_s$ & & \\

\end{tabular}
\end{center}
\caption{Cosmological parameters}
\label{tab:1:cosmopar}
\end{table}

\bibliography{ref}