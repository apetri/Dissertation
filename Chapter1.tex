%%%%%%%%%%%%%%%%%%%%%%%%%%%%%%%%%%%%%%%%%%%%%%%%%%%%%%%%%%%%%%%%%%%%%%%%%%%

\chapter{The $\Lambda$CDM cosmological model}
%\lhead[\fancyplain{}{\thepage}]{\fancyplain{}{\rightmark}}
 \thispagestyle{fancyplain}
\setlength{\parindent}{10mm}
\label{chp:1}
%%%%%%%%%%%%%%%%%%%%%%%%%%%%%%%%%%%%%%%%%%%%%%%%%%%%%%%%%%%%%%%%%%%%%%%%%%%

In this Chapter we discuss the main features of the Standard Model of cosmology. We first approximate the Universe as a homogeneous and isotropic system, following the guidelines of the Friedmann model \citep{Friedman1922}. We then study the physics of the large scale density fluctuations that are present on top of the uniform background. We list and define the free parameters in the Standard Model. 

\section{The Friedmann-Robertson-Walker model}
In the course of this work, we assume the Universe to be described by a flat Friedmann-Robertson-Walker (FRW) model with time dependent scale factor $a(t)$ and Hubble parameter $H(t)=\dot{a}(t)/a(t)$. We introduce comoving spatial coordinates $\bb{x}$, centered on an Earth--based observer, and we define a spacetime 4--vector $dx^{\mu}=(cdt,d\bb{x})$. We can define a 4--momentum associated with the $x^\mu$ coordinates as $P^\mu=dx^\mu/ds$, with $ds=cd\tau$ (this definition states that $\tau$ can be identified as proper time for massive particles. For the case of photons we identify $\tau$ as a geodesic line parameter). Assuming an homogeneous and isotropic Universe, the line element can be written as  

\begin{equation}
\label{eq:1:frw}
ds^2 = g_{\mu\nu}dx^\mu dx^\nu = -c^2dt^2 + a(t)^2d\bb{x}^2
\end{equation}
%
We introduced the diagonal metric tensor $g_{\mu\nu}$, defined as 

\begin{equation}
\label{eq:1:metric}
g_{\mu\nu}(\bb{x},t) = 
\begin{pmatrix}
-1 & 0 & 0 & 0 \\
0 & a(t)^2 & 0 & 0 \\
0 & 0 & a(t)^2 & 0 \\
0 & 0 & 0 & a(t)^2 
\end{pmatrix}
\end{equation}
%
In the remainder of the Chapter we will use the notation $g\equiv-\vert\bb{g}\vert$ and we will use the metric to raise and lower indexes, i.e. we define $V_\mu = g_{\mu\nu}V^\nu$ for a generic 4--vector $V^\mu$. We also define $g^{\mu\nu}=(\bb{g}^{-1})_{\mu\nu}$.
The contents of the FRW universe are assumed to be perfect fluids, described by spatially uniform mass densities $\rho_{(i)}$ and pressures $\mathcal{P}_{(i)}$. We consider the reference frame of a fundamental observer, in which the stress--energy tensor for species $i$ assumes the form 

\begin{equation}
\label{eq:1:stressenergy}
\mathcal{T}_{(i)}^{\mu\nu} = \left(\rho_{(i)}c^2+\mathcal{P}_{(i)}\right)U^\mu U^\nu + \mathcal{P}_{(i)}g^{\mu\nu}
\end{equation} 
%
Here $U^\mu$ is the 4--velocity of a fluid element centered at $x^\mu$. Note that, for a fundamental observer, in absence of perturbations the fluid elements 4--velocity must be consistent the homogeneity assumption 

\begin{equation}
\label{eq:1:4-momentum-unif}
U^\mu = (1,0,0,0)
\end{equation}
%
Note also that the stress--energy tensor must obey the covariant conservation law

\begin{equation}
\label{eq:1:covconservation}
\nabla_\mu \mathcal{T}^{\mu\nu}_{(i)} = \partial_\mu \mathcal{T}^{\mu\nu}_{(i)} + \Gamma_{\mu\alpha}^\mu \mathcal{T}^{\alpha\nu}_{(i)} + \Gamma^\nu_{\mu\alpha}\mathcal{T}^{\mu\alpha}_{(i)} = 0 
\end{equation} 
%
The only non zero components of the affine connection $\Gamma$ for the FRW metric (\ref{eq:1:frw}) are

\begin{equation}
\label{eq:1:connection-unif}
\Gamma^i_{j0} = {\Gamma^0_{ij}}{a^2} = \frac{H}{c}\delta_{ij}
\end{equation}
%
Combining (\ref{eq:1:connection-unif}) with (\ref{eq:1:stressenergy}), the 0--th component of the conservation condition (\ref{eq:1:covconservation}) reads

\begin{equation}
\label{eq:1:rho-conservation}
\dot{\rho}_{(i)} + 3H\left(\rho_{(i)} + \mathcal{P}_{(i)}/c^2\right) = 0
\end{equation}
%
The metric $g_{\mu\nu}$ has to satisfy Einstein equation with source terms $\mathcal{T}_{\mu\nu}$

\begin{equation}
\label{eq:1:einstein-full}
\mathcal{R}^{\mu\nu}-\frac{1}{2}\mathcal{R}g^{\mu\nu} = \frac{8\pi G}{c^4}\sum_i \mathcal{T}^{\mu\nu}_{(i)}
\end{equation}
%
In (\ref{eq:1:einstein-full}) we indicated the Ricci tensor as $\mathcal{R}^{\mu\nu}$ and the Ricci scalar as $\mathcal{R}=g_{\mu\nu}\mathcal{R}^{\mu\nu}$. It can be shown (see for example \citep{Dodelson-C1}) that the components of the Ricci tensor for the metric (\ref{eq:1:frw}) are 

\begin{equation}
\label{eq:1:ricci}
\begin{matrix}
\displaystyle \mathcal{R}_{00} = -\frac{3\ddot{a}}{ac^2} & ; & \displaystyle \mathcal{R}_{ij} = \left(\frac{a\ddot{a}+2a^2H^2}{c^2}\right)\delta_{ij} & ; & \displaystyle \mathcal{R} = \frac{6}{c^2}\left(\frac{\ddot{a}}{a}+H^2\right) \\ \\ 
\end{matrix}
\end{equation}
%
With the symmetries in the metric (\ref{eq:1:frw}), the Einstein equations (\ref{eq:1:einstein-full}) have two independent components, which take the name of Friedmann equations 

\begin{equation}
\label{eq:1:friedmann-1}
H^2 = \frac{8\pi G}{3}\sum_i\rho_{(i)}
\end{equation} 
%
\begin{equation}
\label{eq:2:friedmann-2}
\frac{\ddot{a}}{a} = -\frac{4\pi G}{3}\sum_i\left(\rho_{(i)}+\frac{3\mathcal{P}_{(i)}}{c^2}\right)
\end{equation}
%
Once relations between the components pressures and densities are specified, the conservation equation (\ref{eq:1:rho-conservation}) and the Friedmann equations (\ref{eq:1:friedmann-1}), (\ref{eq:2:friedmann-2}) can be solved explicitly to obtain the time dependencies of $a,\rho_{(i)},\mathcal{P}_{(i)}$. Later in the Chapter, we will derive these solutions explicitly for the cases relevant to this work.  

\subsection{Distance-redshift relation}
In this section we summarize the basics of the cosmological redshift effect, which is a direct consequence of the FRW geometry. Consider a source (for example a galaxy) at a comoving distance $\chi_s$ from the observer on Earth, which emits light at a frequency $\nu_s$. Due to the expansion of the Universe, which is parametrized by the scale factor $a$, the wavelength of the light gets stretched as photons travel from the source to the observer. Indicating the observed frequency on Earth as $\nu_0$, we can define a redshift parameter associated with the source

\begin{equation}
\label{eq:1:redshift}
z_s = \frac{\nu_s}{\nu_0}-1
\end{equation}  
%
It can be shown that there is a one--to--one correspondence between redshift and scale factor, defined by

\begin{equation}
\label{eq:1:z-a}
z_s = \frac{1}{a(t_s)}-1
\end{equation}
%
where $t_s$ is the emission time of a photon that reaches Earth at the present time $t_0$, for which we assumed $a(t_0)=1$. The FRW metric (\ref{eq:1:frw}) establishes correspondences between the source redshift $z_s$, the photon emission time $t_s$ and the source distance $\chi_s$. Using the fact that $ds^2=0$ along a photon spacetime trajectory, we can write

\begin{equation}
\label{eq:1:chi-a}
\chi_s = -c\int_{t_0}^{t_s}\frac{dt}{a} = -c\int_{a(t_0)}^{a(t_s)}\frac{da}{a^2H}  
\end{equation}

\begin{equation}
\label{eq:1:t-a}
t_s = \int_{0}^{\chi_s}\frac{a d\chi}{c} = -\int_{a(t_0)}^{a(t_s)}\frac{da}{aH}  
\end{equation}
%
Note that, using (\ref{eq:1:z-a}), the relations (\ref{eq:1:chi-a}), (\ref{eq:1:t-a}) can be rewritten as 

\begin{equation}
\label{eq:1:chi-z}
\chi_s = c\int_0^{z_s}\frac{dz}{H(z)}
\end{equation} 
%
\begin{equation}
\label{eq:1:t-z}
t_s = \int_0^{z_s}\frac{dz}{(1+z)H(z)}
\end{equation} 
%
In practical observations, source redshifts $z_s$ are measured using photometric \citep{LSST} or spectroscopic \citep{DESI} techniques and $\chi_s,t_s$ are then inferred from equations (\ref{eq:1:chi-z}), (\ref{eq:1:t-z}) with the help of Friedmann equation (\ref{eq:1:friedmann-1}), which sets the time dependence of the Hubble parameter $H$. The late universe ($z\ll 3000$) is well described in terms of two components, namely cold matter and Dark Energy (hence the $\Lambda$CDM denomination), which effects we explore in the next sections. 

\subsection{Cold Dark Matter}
In this work we model Dark Matter at late times as a non--relativistic species of particles with mass $m$. When the equilibrium temperature $T$ is much smaller than $mc^2/k_B$, we can neglect the pressure term in (\ref{eq:1:rho-conservation}) and obtain a scaling relation for the dark matter mass density $\rho_m$ with $a$

\begin{equation}
\label{eq:1:matter-scaling}
\rho_m(a) = \rho_m(a_0)\left(\frac{a(t_0)}{a}\right)^3 = \rho_m(a_0)(1+z)^3 
\end{equation} 
%
Plugging (\ref{eq:1:matter-scaling}) in the Friedmann equation (\ref{eq:1:friedmann-1}) we can get the time dependence of $a$ form a pure Dark Matter Universe

\begin{equation}
\label{eq:1:matter-only-a}
a(t) = a(t_0)\left(\frac{t}{t_0}\right)^{2/3}
\end{equation}

\subsection{Dark energy}
The existence of Dark Energy was postulated after observational evidence of the accelerated expansion of the Universe. Suppose that Dark Energy is described by a perfect fluid with density $\rho_\Lambda$ and pressure $\mathcal{P}_\Lambda$ which are related by

\begin{equation}
\label{eq:1:de-eos}
\mathcal{P}_\Lambda = w\rho_\Lambda c^2
\end{equation}
%
$w$ takes the name of Dark Energy \textit{equation of state}. Looking at equation (\ref{eq:2:friedmann-2}), we note that a necessary condition for Dark Energy to cause $\ddot{a}>0$ is $w<-1/3$. Because of the challenges posed by modeling a fluid with negative $w$ from first principles, Dark Energy is usually described by a phenomenological, $a$ dependent, equation of state in the form

\begin{equation}
\label{eq:1:de-eos-running}
w(a) = w_0 + w_a(1-a)
\end{equation}
%
with $w_0,w_a$ constant \citep{LinderDE}. With the assumption (\ref{eq:1:de-eos-running}), the conservation equation (\ref{eq:1:rho-conservation}) can be solved for the scale dependency of $\rho_\Lambda$

\begin{equation}
\label{eq:1:de-scaling}
\rho_\Lambda(a) = \rho_\Lambda(a_0)\left(\frac{a_0}{a}\right)^{3(1+w_0+w_a)}e^{3w_a(a-a(t_0))} 
\end{equation}
%
We can consider a few limit cases of (\ref{eq:1:de-scaling}). For $w_0=-1,w_a=0$ the Dark Energy density $\rho_\Lambda$ does not depend on $a$ and behaves as a cosmological constant. If $w_0=-1$ and $w_a\neq 0$, on the other hand, there is a non--trivial scaling of $\rho_\Lambda$ with $a$. In order for this scaling relation to reproduce the observational fact that $\rho_\Lambda$ is negligible at recombination time ($z\sim 1100$), $w_a$ must be negative or zero. Figure \ref{fig:1:distred} shows the alteration of the distance--redshift relation (\ref{eq:1:chi-z}) due to the presence of Dark Energy. Using Supernovae IA as standard candles, \citep{PerlmutterNobel} have measured the $\chi(z)$ relation with sufficient precision to establish Dark Energy as the dominant component in the present Universe, earning the Nobel Prize in 2011.  

\begin{figure}
\begin{center}
\includegraphics[scale=0.5]{Figures/eps/distred.eps}
\end{center}
\caption{Distance--redshift relation for two different Universe models, with and without Dark Energy}
\label{fig:1:distred}
\end{figure}    

%%%%%%%%%%%%%%%%%%%%%%%%%%%%%%%%%%%%%%%%%%%%%%%%%%%%%%%%%%%%%%%%%%%%%%%%%%%%%%%%%%%%%%%%%%%%%%%%%%%%%%

\section{Matter density perturbations}
\label{sec:1:density-pert}
In this section we study the deviations from the homogeneous FRW Universe and describe how scalar density perturbations evolve under the effect of gravity. This will be particularly relevant when studying the Gravitational Lensing (GL) effect in the next Chapter, as light ray geodesics deviate from straight lines in the presence of density inhomogeneities. These geodesic deflections have a tangible effect in observations of distant sources, as observed galaxy shapes show apparent distortions that trace the metric perturbations. We review the basic model that describes density perturbations of collision--free Cold Dark Matter in an expanding Universe. Scalar perturbations to the FRW metric (\ref{eq:1:frw}) can be parametrized in the conformal Newtonian gauge \citep{Dodelson-C4} by two scalar potentials $\Phi,\Psi$ in the time--time and space--space components of the metric as 

\begin{equation}
\label{eq:1:confnewt}
ds^2 = -c^2dt^2(1+2\Psi(\bb{x},t)) + a(t)^2d\bb{x}^2(1+2\Phi(\bb{x},t))
\end{equation}
%
For the scope of the present work we can safely ignore vector and tensor perturbations to the FRW metric as their effects are negligible in WL observations. The phase space distribution of Dark Matter is described in terms of a distribution function $f_m(x^\mu,\bb{P})$. In this description, $f_m(x^\mu,\bb{P})g d^3x d^3P$ is the number of particles contained in a phase space volume $d^3x d^3P$. We used the notations $P^\mu=(P^0,\bb{P})$, $d^3 P = dP^x dP^y dP^z$. Note that the momentum dependence of $f_m$ can be expressed in terms of $\bb{P}$ only, since the Dark Matter 4--momentum has to satisfy the constraint

\begin{equation}
\label{eq:1:m-constraint}
g_{\mu\nu}P^\mu P^\nu = -1
\end{equation} 
%
Because the phase space volume element $g d^3x d^3P$ is invariant under coordinate transformations, $f_m$ too must be invariant for the number of particles to be conserved. If we assume $f_m$ to describe a Dark Matter fluid in local equilibrium, we know that $f_m$ depends on the invariant energy $e$ only. Following \citep{JuttnerCov}, $e$ is defined defined by 

\begin{equation}
\label{eq:1:invariant-e}
e = g_{\mu\nu}P^\mu U^\nu
\end{equation}
%
The fluid bulk 4--velocity $U^\mu = (U^0,\bb{U})$ obeys the usual constraint $U^\mu U_\mu=-1$. We can relate the distribution function to the Dark Matter 4--velocity and stress--energy tensor \citep{JuttnerCov} as

\begin{equation}
\label{eq:1:f-velocity}
\int \frac{d^3 P}{P_0}\sqrt{g} P^\mu f_m = \rho_m U^\mu 
\end{equation}
%
\begin{equation}
\label{eq:1:f-stress}
\int \frac{d^3 P}{P_0}\sqrt{g} P^\mu P^\nu f_m = \mathcal{T}^{\mu\nu} 
\end{equation}
%
where, for notational simplicity, we set $\mathcal{T}\equiv \mathcal{T}_m$. Note that equations (\ref{eq:1:stressenergy}) and (\ref{eq:1:f-stress}) can be manipulated to obtain expressions for the matter density and pressure in terms of $f_m$

\begin{equation}
\label{eq:1:f-rho}
\rho_m = \int \frac{d^3 P}{P_0}\sqrt{g}e^2 f_m
\end{equation}
%
\begin{equation}
\label{eq:1:f-press}
\mathcal{P}_m = \frac{1}{3}\int \frac{d^3 P}{P_0}\sqrt{g}(e^2-1) f_m
\end{equation}
%
Using expression (\ref{eq:1:f-press}), it is easy to show that $\mathcal{P}_m = O(\bb{U}^2)$ and that the pressure term can be neglected in the non--relativistic limit, as expected. 
We parametrize the Dark Matter density as 

\begin{equation}
\label{eq:1:matter-rho}
\rho_m(\bb{x},t) = \bar{\rho}_m(t)(1+\delta(\bb{x},t))
\end{equation}
%
where $\bar{\rho}_m(t)$ is the spatially averaged density and $\delta(\bb{x},t)$ is the spatially dependent density contrast. In the next sub--section, we will use the Boltzmann equation for $f_m$ to relate the evolution of $\delta$ and $\bb{U}$ in the non--relativistic limit.      

\subsection{Collision--free Boltzmann equation}
In the absence of collisions between Dark Matter particles, the phase space volume is preserved in the system evolution, and the distribution function satisfies the source--free Boltzmann equation 

\begin{equation}
\label{eq:1:boltzmann}
\frac{df_m(x^\mu,\bb{P})}{ds} = P^\mu\frac{\partial f_m(x^\mu,\bb{P})}{\partial x^\mu} + \frac{d P^i}{ds}\frac{\partial f_m(x^\mu,\bb{P})}{\partial P^i} = 0
\end{equation}
%
The 4--momentum variation rate $dP^i/ds$ can be calculated from the equations of motion, i.e. the geodesic equations for the metric (\ref{eq:1:confnewt})

\begin{equation}
\label{eq:1:geodesic}
\frac{dP^\mu}{ds} = -\Gamma^\mu_{\alpha\beta}P^\alpha P^\beta
\end{equation}
%
The collision--free Boltzmann equation (\ref{eq:1:boltzmann}) then becomes

\begin{equation}
\label{eq:1:boltzmann-2}
P^0\partial_0 f_m + P^i\partial_i f_m - \frac{\partial f_m}{\partial P^i}\left(P^0P^0\Gamma_{00}^i + 2\Gamma_{0j}^iP^0P^j + \Gamma_{jk}^i P^jP^k \right) = 0
\end{equation}
%
Equations for $\rho_m,U^\mu$ can be obtained from the $P$--moments of the Boltzmann equation (\ref{eq:1:boltzmann-2}). We can integrate (\ref{eq:1:boltzmann-2}) in $d^3P$ directly, or we can multiply it by $P^j$ and then integrate. To perform the calculations, we make use of the expressions

\begin{equation}
\label{eq:1:boltzmann-mom-1}
\int d^3P P^\mu f_m = \frac{\mathcal{T}^\mu_0}{\sqrt{g}}
\end{equation}
%
\begin{equation}
\label{eq:1:boltzmann-mom-2}
\int d^3P P^0 P^i \frac{\partial f_m}{\partial P^j} = \frac{\mathcal{T}^i_j-\delta_{ij}\mathcal{T}^0_0}{\sqrt{g}}
\end{equation}
%
\begin{equation}
\label{eq:1:boltzmann-mom-3}
\int d^3P P^0P^0 \frac{\partial f_m}{\partial P^i} = \frac{2\rho_m U_i}{\sqrt{g}}
\end{equation}
%
\begin{equation}
\label{eq:1:boltzmann-mom-4}
\int d^3P P^iP^j f_m =  O(\bb{U}^2)
\end{equation}
%
\begin{equation}
\label{eq:1:boltzmann-mom-5}
\int d^3P P^0P^i f_m = \frac{\rho_mU^i}{\sqrt{g}} + O(\bb{U}^2)
\end{equation}
%
\begin{equation}
\label{eq:1:boltzmann-mom-6}
\int d^3P P^iP^j \frac{\partial f_m}{\partial P^k} = -\frac{\delta_{ki}U^j + \delta_{kj}U^i}{\sqrt{g}} + O(\bb{U}^2)
\end{equation}
%
\begin{equation}
\label{eq:1:boltzmann-mom-7}
\int d^3P P^0P^iP^j \frac{\partial f_m}{\partial P^k} = -\frac{\delta_{ki}U^j + \delta_{kj}U^i}{\sqrt{g}} + O(\bb{U}^2)
\end{equation}
%
\begin{equation}
\label{eq:1:boltzmann-mom-8}
\int d^3P P^0P^0P^i \frac{\partial f_m}{\partial P^j} = -\frac{\rho_m\delta_{ij}}{\sqrt{g}} + O(\bb{U}^2)
\end{equation}
%
In addition to the above results, we  use the approximate expressions for the stress--energy tensor 

\begin{equation}
\label{eq:1:stress-energy-nr}
\begin{matrix}
\mathcal{T}^0_0 = -\rho_m + O(\bb{U}^2) & ; & \mathcal{T}^i_0 = -(1+2\Psi)\rho_m U^i + O(\bb{U}^2)  & ; & \mathcal{T}^i_j = O(\bb{U}^2)
\end{matrix}
\end{equation}
%
We can now perform an integration of equation (\ref{eq:1:boltzmann-2}) in $d^3P$, taking the non--relativistic limit to discard all $O(\bb{U}^2)$ terms. We obtain

\begin{equation}
\label{eq:1:b-monopole}
\frac{\dot{\rho}_m}{c} + \nabla\cdot[(1+2\Psi)\rho_m\bb{U}] - \rho_m[\partial_t\log \sqrt{g} +  2(U_i\Gamma^i_{00} - 2\Gamma^i_{0i}) + (1+2\Psi)U^i(\Gamma_{ij}^j+\Gamma^i_{jj})] = 0
\end{equation}
%
We can also multiply (\ref{eq:1:boltzmann-2}) by $P^j$ and integrate, taking again the non--relativistic limit. The integration yields 

\begin{equation}
\label{eq:1:b-dipole}
\partial_t(\rho_m U^j) - \rho_m (U^j\partial_t\log \sqrt{g} - c\Gamma^j_{00} - 8c\Gamma_{0i}^jU^i) = 0 
\end{equation}
%
Although the system of equations (\ref{eq:1:b-monopole}) and (\ref{eq:1:b-dipole}) can be closed with the help of the Einstein equation (\ref{eq:1:einstein-full}), its exact solution is complicated to calculate because of the non--linearity of the system, and usually involves numerical methods \citep{gadget2} or heuristics based on the halo model \citep{HaloModel}. In the limit in which the perturbations are still at linear stage, i.e. when the density contrast $\delta$ is small, we can trust the linearized version of (\ref{eq:1:b-monopole}), (\ref{eq:1:b-dipole}). We make use of the linear expression for the affine connection $\Gamma$

\begin{equation}
\label{eq:1:connection}
\begin{matrix}
\Gamma_{00}^0 = \dot{\Psi}/c & ; & \Gamma_{0i}^0=\Gamma_{i0}^0 = \partial_i\Psi \\  
\Gamma_{ij}^0 = [H+2H(\Phi-\Psi)+\dot{\Phi}]a^2\delta_{ij}/c & ; & \Gamma_{00}^i = \partial_i \Psi/a^2 \\
\Gamma^i_{j0} = \Gamma^i_{0j} = (H+\dot{\Phi})\delta_{ij}/c & ; & \Gamma_{jk}^i = (\delta_{ij}\partial_k+\delta_{ik}\partial_j-\delta_{jk}\partial_i)\Phi\\
\end{matrix}
\end{equation}
%
which, when plugged in (\ref{eq:1:b-monopole}), (\ref{eq:1:b-dipole}) leads to

\begin{equation}
\label{eq:1:b-monopole-lin}
\dot{\delta} + c \nabla\cdot \bb{U} + 3\dot{\Phi} - \dot{\Psi} = 0
\end{equation}

\begin{equation}
\label{eq:1:b-dipole-lin}
\partial_t(\bar{\rho}_m\bb{U}) + 5H\bar{\rho}_m\bb{U} + \frac{c\nabla\Psi}{a^2} = 0
\end{equation}
%
In this derivation we used the fact that, in the non--relativistic limit, $\partial_t\bar{\rho}_m + 3H\bar{\rho}_m=0$ (this relation can also be deducted from the $O(1)$ terms in equation (\ref{eq:1:b-monopole})). We observe that, if one ignores the $\Phi,\Psi$ terms in (\ref{eq:1:b-monopole-lin}), this relation is a continuity equation which describes mass conservation. In this fashion, $\bb{v}=c\bb{U}$ can be identified as the peculiar velocity $\dot{\bb{x}}$ of a fluid element, on top of the Universe expansion. 

\subsection{Einstein equation}
The system composed by the linear equations for $\delta$ and $\bb{U}$ (\ref{eq:1:b-monopole-lin}), (\ref{eq:1:b-dipole-lin}) can be closed with the Einstein equation, which relates the potentials $\Phi,\Psi$ to the components of the stress--energy tensor. Since we limit our study to scalar perturbations, there are only two independent components of the Einstein equation that need to be considered. WL physics is dominated by the late time behavior of density perturbations, and hence we can ignore relativistic particles and focus on cold matter only. Under this assumption, the $00,0i$ and $ij$ components of the linearized Einstein equation (\ref{eq:1:einstein-full}) become respectively (see \citep{Dodelson-C5})

\begin{equation}
\label{eq:1:einstein-00}
\nabla^2\Phi +\frac{3a^2}{c^2}(H^2\Psi-H\dot{\Phi}) = -\frac{4\pi Ga^2\bar{\rho}_m\delta}{c^2}
\end{equation}

\begin{equation}
\label{eq:1:einstein-0i}
\nabla(\dot{\Phi}-H\Psi) = \frac{4\pi Ga^2\bar{\rho}_m\bb{v}}{c^2}
\end{equation}

\begin{equation}
\label{eq:1:einstein-ij}
\nabla^2(\Phi+\Psi) = 0
\end{equation}
%
A few considerations are in order here. First of all, the terms in (\ref{eq:1:einstein-00}) which contain powers of $aH$ are sub--dominant for the WL case of interest, as the laplacian term is dominant for modes with wavenumber $k$ well inside the Hubble horizon $kc\gg aH$ (see \citep{PNLensing} for a discussion of higher order Post Newtonian terms). We can hence drop these terms from (\ref{eq:1:einstein-00}), which then reduces to a Poisson--like equation 

\begin{equation}
\label{eq:1:poisson}
\nabla^2\Phi(\bb{x},t) = -\frac{4\pi Ga(t)^2}{c^2}\bar{\rho}_m(t)\delta(\bb{x},t).
\end{equation}
%
Equation (\ref{eq:1:einstein-ij}) comes from the traceless part of the spatial Einstein equation and its source term corresponds to anisotropic stresses in the matter components. Because such stresses are proportional to the momentum quadrupole of the their phase space distributions, which is negligible in the non--relativistic limit, anisotropic stresses can be safely neglected when studying WL. We will then use (\ref{eq:1:einstein-ij}) to conclude $\Psi=-\Phi$, since we assume no singularities in the spatial profiles of $\Psi,\Phi$.  

\subsection{Linear growth factor}
The Poisson equation (\ref{eq:1:poisson}) leads to an equation for the density contrast which is linear in $\delta$. We combine the time derivative of (\ref{eq:1:b-monopole-lin}) with the divergence of (\ref{eq:1:b-dipole-lin}) and we ignore terms proportional to $\dot{\Psi},\dot{\Phi}$ (which give rise to PN corrections). After a few algebraic manipulations we get 

\begin{equation}
\label{eq:1:delta-lin}
\ddot{\delta} + 2H\dot{\delta} - 4\pi G\bar{\rho}_m\delta = 0  
\end{equation} 
%
Because of the linearity of equations (\ref{eq:1:b-monopole-lin}) and (\ref{eq:1:b-dipole-lin}), each Fourier mode $\tilde{\delta}(\bb{k},t)$ evolves independently in time. Moreover, in absence of pressure terms (which would contribute with terms proportional to $\nabla^2\delta$), the density contrast $\delta$ evolves in a  self--similar fashion

\begin{equation}
\label{eq:1:selfsim}
\tilde{\delta}(\bb{k},t) = D(t)\tilde{\delta}(\bb{k},0)
\end{equation}
%
with the linear growth factor $D$ that does not depend on the wavenumber $k$. Equation (\ref{eq:1:delta-lin}) can be converted in a relation for the linear growth factor $D$ with the use of the time--redshift relation (\ref{eq:1:t-z}) and the Friedmann equations (\ref{eq:1:friedmann-1}), (\ref{eq:2:friedmann-2}). After a few algebraic manipulations we obtain

\begin{equation}
\label{eq:1:growth-diff}
\frac{d^2D(z)}{dz^2} + \frac{4\pi G}{3}\left(\frac{\bar{\rho}_m(z)+\rho_\Lambda(z)[1+3w(z)]}{(1+z)H(z)^2}\right)\frac{dD(z)}{dz} -\frac{8\pi G\Omega_m(z)}{H(z)^2(1+z)^2}D(z)= 0
\end{equation}
%
\begin{figure}
\begin{center}
\includegraphics[scale=0.5]{Figures/eps/growth.eps}
\end{center}
\caption{Linear growth factor $D(z)$ calculated solving (\ref{eq:1:growth-diff}) for 4 different $\Lambda$CDM cosmologies. The initial condition has been set for a unit density perturbation at $z=1000$, namely $D(1000)=1, \dot{D}(1000)=0$. Observe the fact that the growth of perturbations is suppressed by the presence of Dark Energy, which accelerates the expansion of the Universe and makes it harder for over--densities to grow.}
\label{fig:1:growth}
\end{figure}
%
In the limiting case of a pure Dark Matter universe ($\rho_\Lambda=0$), (\ref{eq:1:growth-diff}) reduces to

\begin{equation}
\label{eq:1:growth-diff-cold}
\frac{d^2D(z)}{dz^2} + \frac{1}{2(1+z)}\frac{dD(z)}{dz}-\frac{3D(z)}{(1+z)^2} = 0
\end{equation}
% 
which admits a solution $D(z)\propto (1+z)^{-1}=a$. Figure \ref{fig:1:growth} shows the evolution of the linear growth factor $D$ with redshift for different combinations of the $\Lambda$CDM parameters. 

%%%%%%%%%%%%%%%%%%%%%%%%%%%%%%%%%%%%%%%%%%%%%%%%%%%%%%%%%%%%%

\section{$\Lambda$CDM cosmological parameters}
One of the main goals of the research presented in this work is to study how WL observations can be used to constrain some of the free parameters that describe the $\Lambda$CDM universe. In the conclusion of this Chapter we present a parametrization will be used consistently throughout the dissertation writeup. The present day Hubble parameter $H_0\equiv H(t_0)$ is expressed in terms of the dimensionless number $h$ as

\begin{equation}
\label{eq:1:hubble-present}
H_0 = 100h\,{\rm km}\,{\rm s}^{-1}\,{\rm Mpc}^{-1}
\end{equation} 
%
The densities of the components that source the Einstein equation are usually quoted in the literature in terms of their ratios with the present critical density $\rho_c=3H_0^2/8\pi G$. We use the notation

\begin{equation}
\label{eq:1:omega-def}
\Omega_i = \frac{8\pi G\rho_i(t_0)}{3H_0^2}
\end{equation}
%
In addition to Dark Matter and Dark Energy, the present universe contains a significant fraction of baryons ($\Omega_b\approx \Omega_m/6$), whose physics is more complicated to model with respect to the one that controls cold matter, as the Boltzmann equation for baryons contains pressure terms and collisional terms. In this work we ignore baryon physics, although its investigation in cosmology and WL is currently an active area of research. The initial conditions for the density inhomogeneities described in \S~\ref{sec:1:density-pert} are believed to be set at early times by quantum perturbations, which are present during an early epoch of accelerated expansion called \textit{inflation} (\citep{Inflation}). Inflation is believed to generate random initial conditions which are Gaussian, statistically isotropic and nearly scale invariant 

\begin{equation}
\label{eq:1:initial-cond}
\langle\tilde{\delta}(\bb{k},z_{\rm in})\tilde{\delta}^*(\bb{k}',z_{\rm in})\rangle = (2\pi)^3 P_\delta(k,z_{\rm in})\delta^D(\bb{k}-\bb{k}')
\end{equation}
%
\begin{equation}
\label{eq:1:initial-ps}
P_\delta(k,z_{\rm in}) = \frac{A^2_s}{k^3}\left(\frac{k}{k_0}\right)^{n_s-1}
\end{equation}
%
In this notation, $n_s=d\log (k^3P_\delta)/d\log k$ is a parameter that describes the deviation from scale invariance ($n_s=1$ corresponds to scale invariant initial conditions). The overall normalization of the initial density perturbations $A_s$ is usually expressed in terms of an equivalent parameter, $\sigma_8$, defined as 

\begin{equation}
\label{eq:1:sigma8}
\sigma_8 = \int \frac{d^3k}{(2\pi)^3} P^{\rm lin}_\delta(k,z=0)e^{-k^2 r_8^2}
\end{equation} 
%
The meaning if the notation in equation (\ref{eq:1:sigma8}) is that $P_\delta^{\rm lin}$ is obtained from $\tilde{\delta}(\bb{k},z=0)$ calculated with the linear evolution equation (\ref{eq:1:delta-lin}). $\sigma_8$ is defined as the spatial variance of the present linearly evolved density contrast smoothed with a Gaussian window of size $r_8=8\,{\rm Mpc}/h$. The random nature of the initial conditions, which is a consequence of the quantum mechanical nature of inflation, is at the base of a phenomenon called \textit{cosmic variance}, which states that cosmological observable quantities are essentially random variables. As a consequence, $\Lambda$CDM parameter inferences from observations have to be related to the statistical properties of the observables (i.e. ensemble averages), rather than to the observables themselves. 
%
\begin{table}
\begin{center}
\begin{tabular}[h]{c|c|c}

\textbf{Parameter} & \textbf{Planck 2015} & \textbf{Fiducial} \\ \hline 

$h$ & 0.67 & 0.72 \\
$\Omega_m$ & 0.31 & 0.26 \\
$\Omega_\Lambda$ & 0.69 & 0.74 \\
$\Omega_b$ & 0.046 & 0.046 \\
$w_0$ & - & $-1$ \\
$w_a$ & - & 0 \\
$\sigma_8$ & 0.83 & 0.8 \\
$n_s$ & 0.96 & 0.96 \\

\end{tabular}
\end{center}
\caption{$\Lambda$CDM cosmological parameters from the Planck 2015 \citep{Planck15} data best fit (middle column) and the fiducial values used in this work (right column).}
\label{tab:1:cosmopar}
\end{table}
%
Table \ref{tab:1:cosmopar} shows a list of cosmological parameters measured from the Planck experiment \citep{Planck15}, as well as the fiducial values used throughout this work. The Dark Energy equation of state parameters $w_0,w_a$ are essentially left unconstrained by CMB experiments, as they are sensitive to early Universe physics in which the Dark Energy density is negligible. WL observations, on the other hand, trace density fluctuations at late times, when the effects of Dark Energy are tangible. In the next Chapter, we will review the Gravitational Lensing effect and show how it can be used as a tracer for the Dark Matter density fluctuations. 

%\bibliography{ref}