%%%%%%%%%%%%%%%%%%%%%%%%%%%%%%%%%%%%%%%%%%%%%%%%%%%%%%%%%%%%%%%%%%%%%%%%%%%

\chapter{Applications to the LSST survey: systematic challenges}
%\lhead[\fancyplain{}{\thepage}]{\fancyplain{}{\rightmark}}
 \thispagestyle{plain}
\setlength{\parindent}{10mm}
\label{chp:7}
%%%%%%%%%%%%%%%%%%%%%%%%%%%%%%%%%%%%%%%%%%%%%%%%%%%%%%%%%%%%%%%%%%%%%%%%%%%

In this Chapter we tackle some of the systematic issues that arise in a survey with large sky coverage such as LSST \citep{LSST}. In the previous Chapter we saw that a small WL survey, such as CFHTLenS, leaves the Dark Energy equation of state $w_0$ essentially unconstrained. LSST covers an area of roughly $12,000\,{\rm deg}^2$, which is 100 times bigger than CFHTLenS. This can in principle lead to constraints on cosmology which are 10 times more precise. Increased precision, however, comes at a cost, because systematic effects that were negligible for CFHTLenS due to large statistical errors, may not be negligible anymore when compared to smaller cosmic variance fluctuations. We discuss a variety of systematic effects that can affect parameter estimates: we focus on atmospheric contaminations to the shear signal, sensor effects, and inaccuracies in photometric redshift estimation. We also study potential bias that can arise from approximate forward models based on the Born approximation. To conclude, we mention additional systematic effects which we did not have the chance to investigate, and that we leave for future work.    

\section{Atmospheric/PSF spurious shear}
\label{sec:7:spurious}
The first systematic issue we investigate is the contamination of source galaxy shape measurements \citep{PetriSpShear} due to the presence of the atmosphere. Before hitting the sensors on the telescope plate, photons travel through the Earth's atmosphere, which dilutes the WL signal by convolving it with a characteristic Point Spread Function (PSF). This effect can be better understood thinking about a point source, like a star: when observed on the telescope, this point source looks like an extended object, which traces the angular profile of the PSF. This contamination adds the instrumentation specific issues, such as the telescope's own PSF, tracking errors and photon shot noise. All these effects are modeled and simulated using the \ttt{phosim} software package \citep{LSSTOperations}. We were provided with 20 independent realizations of a \ttt{phosim}--generated spurious shear catalog (see \citep{ChangLSST}) that contains information on $10^5$ galaxies spread over a 4\,deg$^2$ field of view. The properties listed in the catalog include residual spurious shear measurements for each galaxy, after PSF corrections via a polynomial model subtraction were attempted \citep{ChangLSST}. The stochastic component of the residual shear decreases approximately as the inverse of the number of exposures of the field of view. The spatial patterns of the shear residuals in 4 of these realizations are shown in Figure \ref{fig:7:spvisualize}. 
%
\begin{figure}
\begin{center}
\includegraphics[scale=0.4]{Figures/png/spurious_visualize.png}
\end{center}
\caption{4 independent realizations of the residual spurious $\kappa$ after subtractions performed with polynomial fits to the PSF \citep{ChangLSST}. We show the reconstructed $\kappa$ profiles obtained via the KS inversion procedure in (\ref{eq:2:kappa-ks}). A Gaussian smoothing window with scale $\theta_G=1'$ has been convolved with the images.}
\label{fig:7:spvisualize}
\end{figure}
%
Angular correlations in the patterns seen in Figure \ref{fig:7:spvisualize} can be quantified in terms of the shear--shear two point correlation function
\begin{equation}
\label{eq:7:shear2pt}
\xi_{\gamma\gamma}^+(\alpha) = \left\langle\gamma^1(\pt)\gamma^1(\pt+\palpha)+\gamma^2(\pt)\gamma^2(\pt+\palpha)\right\rangle,
\end{equation} 
%
which is related to the spurious shear $E$ and $B$ mode power spectra as 
\begin{equation}
\label{eq:7:shearPow}
\xi_{\gamma\gamma}^+(\alpha) = \int_0^\infty \frac{d\ell}{2\pi}\ell J_0(\ell\alpha)[S^{EE}(\ell)+S^{BB}(\ell)] 
\end{equation}
%
In equation (\ref{eq:7:shearPow}), $S^{EE}$ and $ S^{BB}$ refer to the power spectra of the $E$ and $B$ modes of the spurious shear, which defined in equation (\ref{eq:2:shear-eb}). $J_0$ is the 0--th order Bessel function of the first kind. A useful number to quote is the real space amplitude $\sigma_{\kappa,{\rm sp}}$ of the $\kappa$ contamination induced by spurious shear, defined by
\begin{equation}
\label{eq:7:sigmasys}
\sigma^2_{\kappa,{\rm sp}} = \int_0^\infty \frac{d\ell}{2\pi}\ell S^{EE}(\ell) 
\end{equation}
%
As we can see from Figure \ref{fig:7:eb2d}, we are allowed assume statistical isotropy assumption for this kind of contamination, as its power spectrum depends on $\ell=\vert\pell\vert$ only. We also observe that, contrary to what happens for the WL signal, the magnitude of the spurious $E$ and $B$ mode auto power spectra is comparable. This property can be used as a flag for other kind of systematic effects that contribute to the observed shear with a large $B$ mode. 
%
\begin{figure}
\begin{center}
\includegraphics[scale=0.75]{Figures/png/spurious_eb2D.png}
\end{center}
\caption{Two dimensional profiles of the power spectra measured from the spurious shear $E$ (left panel) and $B$ (middle panel) modes. We also measure the cross $EB$ term $\langle\tilde{\gamma}^E\tilde{\gamma}^B\rangle$ (right panel). The quantities shown are the average of 20 independent residual spurious shear realizations. The statistical isotropy of the patterns is evident, as well as the fact that $S^{EE}$ and $S^{BB}$, unlike the case for the WL signal, are comparable in magnitude.}
\label{fig:7:eb2d}
\end{figure} 
%
A popular model for the scale dependence of the residual spurious shear is encoded by a log--linear power spectrum \citep{AmaraSP}: 
\begin{equation}
\label{eq:7:loglin}
S^{EE}(\ell) = \frac{A}{\ell(\ell+1)}\left\vert1+n\log\left(\frac{\ell}{\ell_0}\right)\right\vert,
\end{equation}
%
where $A,n,\ell_0$ refer to the spurious shear amplitude, spectral index and $\ell$ pivot point respectively. \citep{AmaraSP} employ such a model in order to forecast parameter bias caused by uncorrected spurious shear. Using the 20 spurious shear realizations we were provided with, we found that the log--linear model (\ref{eq:7:loglin}) for the residual shear is only correct for small $\ell$ and breaks down on smaller scales \citep{PetriSpShear} (probably due in part to the effect of smoothing), as can be seen in Figure \ref{fig:7:spfit}. 
%
\begin{figure}
\begin{center}
\includegraphics[scale=0.6]{Figures/png/spurious_fit.png}
\end{center}
\caption{$\ell$ dependence of the residual spurious shear power spectra $S^{EE}$ (blue), $S^{BB}$ (green) and $S^{EB}$ (red). The mid points and error bars refer respectively to the mean and standard deviation of the power spectra measured from the 20 spurious shear realizations. The dashed black line shows the best fit to the $EE$ power spectrum performed with the empirical model in equation (\ref{eq:7:empmodel}).}
\label{fig:7:spfit}
\end{figure} 
%
We propose the following alternative model for the spurious shear power spectrum. The model is piecewise log--linear but has an exponential damping at high $\ell$, and provides a better fit to the instrument simulation than (\ref{eq:7:loglin}). We used the following empirical approximation

\begin{equation}
\label{eq:7:empmodel}
S^{EE}(\ell) = 
\begin{cases}
\frac{A_0}{\ell(\ell+1)}\left\vert1+n_0\log\left(\frac{\ell}{\ell_0}\right)\right\vert \,\,\,\,  {\rm if } \,\,\,\, \ell\leq 700 \\
\frac{A_1}{\ell(\ell+1)}\left\vert1+n_1\log\left(\frac{\ell}{\ell_0}\right)\right\vert \,\,\,\,  {\rm if } \,\,\,\, 700\leq\ell\leq 3300 \\
\frac{A_2\log{\ell}}{\ell(\ell+1)}\exp\left[-b(\log{\ell}-\mu)^2\right] \,\,\,\,  {\rm if } \,\,\,\, \ell > 3300
\end{cases} 
\end{equation}
%
We fixed the pivot point to $\ell_0=700$ and we found the best fit parameters to the pattern seen in Figure \ref{fig:7:spfit} to be $(A_0,n_0,A_1,n_1,A_2,b,\mu)=(3.17\cdot 10^{-5},1.36,1.6\cdot 10^{-4},7.54,4.4\cdot 10^{-5},15.37,3.41)$. If we model the residual spurious shear as an additive contamination to the WL signal, using equation (\ref{eq:5:linapprox-peak}) we can evaluate the $\Lambda$CDM parameter bias that is induced by leaving this systematic effect uncorrected for. We generated Gaussian spurious shear $\kappa$ mock images using the empirical model (\ref{eq:7:empmodel}). The Fourier coefficients for the spurious $\kappa$ maps were drawn from a normal distribution with zero mean and variance $S^{EE}(\ell)$. We added these spurious shear mock realizations on top of the WL signal maps (taken from the \ttt{IGS1} simulations, see Appendix), extracted the image features according to the procedures described in Chapter \ref{chp:4} and quantified parameter bias on the triplet $(\Omega_m,w_0,\sigma_8)$. The results are shown in Table \ref{tab:7:spbias}.
%
\begin{table}
\begin{center}
\begin{tabular}{c|ccc} 
\multicolumn{4}{c}{\textbf{Survey Assumptions}} \\
\multicolumn{4}{c}{$z_s=2$, $n_g=15\,{\rm galaxies/arcmin}^{-2}$, $\ell\in[100,2\cdot10^4]$, $\kappa_{\rm MF}\in[-2\sigma,2\sigma]$, $\kappa_{\rm pk}\in[-2\sigma,5\sigma]$} \\ \hline \hline

\textbf{Model} & $\Omega_m$ & $w_0$ & $\sigma_8$  \\ \hline \hline 
&\multicolumn{3}{c}{\textbf{$\kappa$ power spectrum}} \\ 
\textit{Log--linear} & $4.0\cdot 10^{-6}$  & $-2.69\cdot 10^{-4}$ & $2.5\cdot 10^{-5}$ \\
\textit{LSST simulation} &  $-6.22\cdot10^{-5}$ &  $2.94\cdot10^{-4}$ &  $1.32\cdot10^{-4}$ \\
\textit{LSST simulation} $\times 10$ & $-7.51\cdot10^{-4}$ &  0.0025 &  0.0015 \\ 
\textit{Error} ($1\sigma$) & 0.0015 & 0.01 & 0.0025 \\ \hline \hline

&\multicolumn{3}{c}{\textbf{Minkowski functionals}} \\ 
\textit{Log--linear} & 0.0026 &0.037 & $-0.0024$ \\
\textit{LSST simulation} & 0.0020 &  0.025 & $-0.0014$ \\
\textit{LSST simulation} $\times 10$ & 0.007 & 0.055 & $-0.0068$ \\ 
\textit{Error ($1\sigma$)} & 0.001 &0.005  &0.0014 \\ \hline

&\multicolumn{3}{c}{\textbf{$\kappa$ Moments}} \\ 
\textit{Log--linear} & $-2.8\cdot 10^{-5}$ & $-0.0011$  & $4.7\cdot 10^{-5}$ \\
\textit{LSST simulation} & $1.09\cdot10^{-5}$ & $-3.96\cdot10^{-4}$ & $-7.60\cdot10^{-6}$ \\
\textit{LSST simulation} $\times 10$ & $-2.84\cdot10^{-5}$ & $-4.72\cdot10^{-3}$ &  $1.26\cdot10^{-4}$ \\ 
\textit{Error} ($1\sigma$) & 0.0016 &0.008  & 0.002  \\ \hline \hline

&\multicolumn{3}{c}{\textbf{Peak counts}} \\ 
\textit{Log--linear} & 0.009 & 0.026 & $3.2\cdot 10^{-4}$ \\
\textit{LSST simulation} & 0.0011 &  0.018 &  $2.9\cdot10^{-4}$ \\
\textit{LSST simulation} $\times 10$ & 0.0026 & 0.046 & $4.0\cdot10^{-4}$ \\ 
\textit{Error} ($1\sigma$) & 0.0011  & 0.0062  & 0.0015 \\ \hline \hline

\end{tabular}
\end{center}

\caption{Bias on the parameter triplet $(\Omega_m,w_0,\sigma_8)$ calculated using three different models for the LSST spurious shear: \textit{Log--linear} (first rows) refers to the log--linear model (\ref{eq:7:loglin}) with $(A,n,l_0)=(10^{-6.6},0.7,700)$, with the normalization $\sigma^2_{\kappa,{\rm sp}}=4\times10^{-7}$. \textit{LSST simulation} (second rows) refers to the spurious shear mocks generated with the empirical model (\ref{eq:7:empmodel}) (the amplitudes have been divided by a factor of $N_{\rm exposures}=368$ to account for multiple field of view exposures), \textit{LSST simulation $\times$ 10} (third rows) refers to the same model but with the amplitude $\sigma^2_{\kappa,{\rm sp}}$ increased by a factor of 10. The $1\sigma$ error values (fourth rows) refer to the forecasts for an LSST--like survey obtained with equation (\ref{eq:5:linapprox-cov}).}
\label{tab:7:spbias}
\end{table}
%
The calculations show that, under the assumption that source galaxies are positioned at a constant redshift $z_s=2$, the effect of the spurious shear on cosmological constraints depends on the feature used in the analysis. Features that are polynomial in $\kappa$, such as the power spectrum and moments, deliver constraints which are essentially unbiased (for the study cases described in Table \ref{tab:7:spbias}). The same conclusion is not true for features which probe the morphology of $\kappa$: constraints from Minkowski functionals are biased with a several $\sigma$ significance, when spurious shear is left uncorrected for. The situation is not as dramatic for the peak counts constraint on $\sigma_8$, as the significance of the bias is below $1\sigma$. When we look at the $\Omega_m$ and $w_0$ constraints, however, the bias can be as large as $2\sigma$ for the spurious shear modeled by the LSST instrument simulation. Because the direction of the bias depends on the particular feature considered, possibilities of self--calibration could be explored in the future.

\section{CCD sensor effects}
\label{sec:7:ccd}
In this section we discuss issues that arise from imperfections in the sensors used to image source galaxies. Modern telescopes, such as LSST, use Charge--Coupled Devices (CCD) \citep{CCDBook,LSST,LSSTOperations} as means to covert photon counts into voltage signals, which are then mapped into digitized images. Impurity gradients in the silicon, of which CCDs are made, cause the presence of spurious transverse electric fields, which displace the photons captured by the CCD. Such displacements lead to distortions in shape measurements, which in principle affect reconstructed WL fields. The astrometric displacement $\bb{d}_E$ due to the transverse electric fields is usually modeled as radial field \citep{PetriCCD} on the surface of the CCD according to

\begin{equation}
\label{eq:7:displ}
\bb{d}_E = d(r)\bbh{r}
\end{equation}
%
At first order, this generates an additive contribution to the reconstructed $\kappa$ field, which takes the name of \textit{tree ring} effect. The induced contamination to the convergence, $\kappa_{\rm tree}$, can be calculated as (see \citep{PetriCCD})

\begin{equation}
\label{eq:7:ktree}
\kappa_{\rm tree} = -\frac{1}{2}\nabla\cdot\bb{d}_E = -\frac{1}{2}\left(\frac{d(r)}{r}+\frac{d}{dr}d(r)\right).
\end{equation}   
%
A visualization of the tree ring effect is shown in the left panel of Figure \ref{fig:7:sensvis}. 
%
\begin{figure}
\begin{center}
\includegraphics[scale=0.4]{Figures/png/sensors_visualize.png}
\end{center}
\caption{Spatial profiles of the additive contaminations to $\kappa$ due to the tree ring (left) and pixel size variations (right) effects. The images cover a field of view of $(0.2\,{\rm deg})^2$. In order to extend the mapping of the systematics to the entire LSST field of view of $(3.5\,{\rm deg})^2$, we repeated the patterns seen in this Figure across the whole field of view, applying random $90^\circ$ rotations at each replication.}
\label{fig:7:sensvis}
\end{figure}
%
An additional source of contamination that derives from CCD manufacture imperfections has to do with the variable size of the CCD pixels. If the pixel area is not uniform across the CCD surface, variations in photon counts are erroneously interpreted as variations in the intensity profile of the source. This creates an additional source of error in the measurement of galaxy shapes. The typical spatial profile of the convergence contamination due pixel size variations, $\kappa_{\rm pixel}$, is shown in the right panel of Figure \ref{fig:7:sensvis}. We remand the reader to \citep{PetriCCD} for a throughout discussion and modeling of the tree ring and pixel size variation effects. In order to evaluate the systematic effects on cosmological constraints, we make use of equation (\ref{eq:5:linapprox-peak}) and we use the $\kappa$ power spectrum $P_{\kappa\kappa}$ as an image feature. The bias estimate $\bbh{b}$ in the parameters is calculated as 

\begin{equation}
\label{eq:7:biasccd}
\bbh{b} = \bbh{p}_{\rm sp}-\bbh{p}_0 = \bb{Z}(\bbh{d}_{\rm sp}-\bbh{d}_0),
\end{equation}
%
where we indicated the measured $\kappa$ power spectra with and without CCD systematics present as $\bbh{d}_{\rm sp}$ and $\bbh{d}_0$ respectively. The $N_\pi\times N_d$ projection matrix $\bb{Z}$ is defined as 

\begin{equation}
\label{eq:7:zeta}
\bb{Z} = (\bb{M}^T\bb{\Psi}\bb{M})^{-1}\bb{M}^T\bb{\Psi},
\end{equation}
%
following the notation of \S~\ref{sec:5:fisher}, in which $\bb{M},\bb{\Psi}$ are feature derivative with respect to cosmology and the inverse covariance matrix respectively. More explicitly, we write the power spectrum residuals as 

\begin{equation}
\label{eq:7:powerccd}
\h{P}_{\kappa\kappa+{\rm sp}}(\ell_b)-\h{P}_{\kappa\kappa}(\ell_b) = \left\vert\frac{\h{\tilde{\kappa}}(\ell_b)+\tilde{\kappa}_{\rm sp}(\ell_b)}{2\pi}\right\vert^2-\h{P}_{\kappa\kappa}(\ell_b) 
\end{equation} 
%
\begin{figure}
\begin{center}
\includegraphics[scale=0.6]{Figures/png/sensors_power.png}
\end{center}
\caption{Power spectral density of the additive $\kappa$ contamination due to the tree ring (blue) and pixel size variation (green) effects. The power spectra were measured from one realization of a $(3.5\,{\rm deg})^2$ field of view obtained repeating the patterns in Figure \ref{fig:7:sensvis} with random $90^\circ$ rotations.}
\label{fig:7:ccdpow}
\end{figure}
%
In equation (\ref{eq:7:powerccd}), the subscript $i$ can either refer to the tree ring or pixel variation effect. Note that, contrary to $\h{\tilde{\kappa}}$, the systematic contribution $\tilde{\kappa}_{\rm sp}$ is not a stochastic quantity, since it is tied to the field of view. The same is true for its angular power spectrum $P_{\kappa\kappa,{\rm sp}}$ (shown in Figure \ref{fig:7:ccdpow}). We express the estimator for the bias in the $\alpha$--th cosmological parameter as 
\begin{equation}
\label{eq:7:biasccd-1}
\h{b}_\alpha = \sum_{\ell_b=\ell_{\rm min}}^{\ell_{\rm max}} Z_{\alpha\ell_b}\left(P_{\kappa\kappa,{\rm sp}}(\ell_b)+\frac{\h{\tilde{\kappa}}(\ell_b)\tilde{\kappa}_{\rm sp}^*(\ell_b)+\h{\tilde{\kappa}}^*(\ell_b)\tilde{\kappa}_{\rm sp}(\ell_b)}{(2\pi)^2}\right)
\end{equation}
%
We assume a diagonal covariance matrix for the $\kappa$ power spectrum, which is calculated according to (\ref{eq:4:powercov-gauss-bin}): 

\begin{equation}
\label{eq:7:diagcov}
C_{\ell_b\ell_{b'}} = \frac{P^2_{\kappa\kappa}(\ell_b)}{N_{\rm modes}(\ell_b)}\delta_{\ell_b\ell_{b'}},
\end{equation}
%
where $N_{\rm modes}(\ell_b)$ is the number of $\pell$ modes that fall inside the Fourier annulus of radius $\ell_b$. The value of $N_{\rm modes}(\ell_b)$ can be read off equation (\ref{eq:4:powercov-gauss-bin}). Using the diagonal assumption, we can write down the expectation value $\bb{b}$ and scatter $\sigma_{\bb{b}}$ of the bias estimator (\ref{eq:7:biasccd-1}) as  

\begin{equation}
\label{eq:7:expb}
b_\alpha = \left\langle\h{b}_\alpha\right\rangle = \sum_{\ell_b=\ell_{\rm min}}^{\ell_{\rm max}} Z_{\alpha\ell_b}P_{\rm sp}(\ell_b)
\end{equation}
%
\begin{equation}
\label{eq:7:sigmab}
\sigma_{b_\alpha} = \sqrt{\left\langle\left(\h{b}_\alpha-b_\alpha\right)^2\right\rangle} = 2\sum_{\ell_b=\ell_{\rm min}}^{\ell_{\rm max}} Z_{\alpha\ell_b}\sqrt{\frac{P_{\rm sp}(\ell_b)P_{\kappa\kappa}(\ell_b)}{N_{\rm modes}(\ell_b)}}
\end{equation}
%
Note that, because of the nature of the bias estimator (\ref{eq:7:biasccd-1}), the parameter bias induced by CCD effects has both a fixed component (\ref{eq:7:expb}) proportional to $P_{\rm sp}$ and a stochastic component with a root mean square error (\ref{eq:7:sigmab}), which scales as $\sqrt{P_{\rm sp}/N_{\rm modes}}$. Depending on the size of the survey, which sets the magnitude of $N_{\rm modes}(\ell_b)$, the fixed and stochastic components of the bias have different relative amplitudes because, while (\ref{eq:7:sigmab}) decreases with the survey area, (\ref{eq:7:expb}) does not.
%
\begin{table}
\begin{center}
\begin{tabular}{c|ccc}
\textbf{Bias component} & $\Omega_m$ & $w_0$ & $\sigma_8$ \\ \hline \hline
\multicolumn{4}{c}{\textbf{Tree rings}} \\ \hline
$\bb{b}$ & $5.05\cdot 10^{-10}$ & $2.79\cdot 10^{-9}$ & $-3.52\cdot 10^{-10}$ \\
$\sigma_{\bb{b}}$ & $6.92\cdot 10^{-8}$ & $1.34\cdot 10^{-7}$ & $1.29\cdot 10^{-7}$ \\ \hline

\multicolumn{4}{c}{\textbf{Pixel size variations}} \\ \hline
$\bb{b}$ & $1.21\cdot 10^{-5}$ & $-2.18\cdot 10^{-5}$ & $-1.79\cdot 10^{-5}$ \\
$\sigma_{\bb{b}}$ & $1.21\cdot 10^{-5}$ & $3.37\cdot 10^{-5}$ & $1.82\cdot 10^{-5}$ \\ \hline
\end{tabular}
\end{center}
\caption{Amplitudes for the fixed (\ref{eq:7:expb}) and stochastic (\ref{eq:7:sigmab}) components of the $(\Omega_m,w_0,\sigma_8)$ bias induced by the tree ring and pixel size variations effects resulting from CCD fabrication imperfections. The spurious contributions to $\kappa$ were measured from a LSST instrument simulation \citep{PetriCCD}, and the forward models necessary to obtain the WL $P_{\kappa\kappa}$ derivatives $\bb{M}$ and covariance matrix $\bb{C}$ were calculated with the analytical code $\ttt{NICAEA}$ \citep{Nicaea,Nicaea17}. The number $N_{\rm modes}$ of $\pell$ modes which appears in equation (\ref{eq:7:sigmab}) is referred to an LSST--like survey. Shape noise contributions for source galaxies placed at $z_s=2$, with a galaxy density of $n_g=15\,{\rm galaxies/arcmin^2}$, are included.}
\label{tab:7:ccdbias}
\end{table}
%
Table \ref{tab:7:ccdbias} shows the values of the bias components $\bb{b},\sigma_{\bb{b}}$ for an LSST--like galaxy survey. Compared with the $1\sigma$ $\Lambda$CDM parameter errors shown in Table \ref{tab:7:spbias}, we can safely conclude that the bias induced by this kind of CCD imperfections is negligible even for a survey with an area as wide as LSST. The bias is several order of magnitude smaller than the parameter uncertainty caused by cosmic variance.  

\section{Photometric redshift errors}
\label{sec:7:photoz}
In this section we study the effect of uncorrected redshift measurement errors on $\Lambda$CDM inferences. Photometric surveys, such as LSST, do not use full spectroscopic information in order to determine the redshift $z_s$ of a source, but use a limited number of frequency bands (LSST uses 5 of them for example) to provide an estimate of $z_s$ instead. This estimate is usually inaccurate \citep{LSSTSciBook}. We model the relation between the photometric and real redshift of a source galaxy as the sum of a fixed bias $b_{\rm ph}$ and a stochastic component of root mean square $\sigma_{\rm ph}$ \citep{PetriPhotoZ,LSSTSciBook}, according to  

\begin{equation}
\label{eq:7:phreal}
z_{\rm ph}(z_s) = z_s + b_{\rm ph}(z_s) + \sigma_{\rm ph}(z_s)\mathcal{N}(0,1)
\end{equation}
%
We chose the functional forms of the fixed and stochastic components following the LSST Science Book \citep{LSSTSciBook}: 

\begin{equation}
\label{eq:7:bph}
b_{\rm ph}(z_s) = 0.003(1+z_s)
\end{equation}
%
\begin{equation}
\label{eq:7:sph}
\sigma_{\rm ph}(z_s) = 0.02(1+z_s)
\end{equation}
%
\begin{figure}
\begin{center}
\includegraphics[scale=0.6]{Figures/png/lsst_galdistr.png}
\end{center}
\caption{Redshift distribution of $10^6$ source galaxies arranged uniformly in a $(3.5\,{\rm deg})^2$ field of view (which corresponds to a density of $n_g=22\,{\rm galaxies/arcmin}^2$). The distribution follows the law $n(z_s)\propto (z_s/z_0)^2\exp(-z_s/z_0)$ with $z_0=0.3$. For the purpose of this study (which makes use of the \ttt{LSST100Parameters} simulation suite, see Appendix) the galaxies have been divided in 5 redshift bins, chosen such that each bin contains the same number of galaxies.}
\label{fig:7:galdistr}
\end{figure}
%
We simulated an LSST--like galaxy survey by drawing the redshift $z_s$ of $N_g=10^6$ source galaxies from the distribution in Figure \ref{fig:7:galdistr}. The galaxies are distributed uniformly in a $(3.5\,{\rm deg})^2$ field of view. Uncorrected photometric redshift errors can bias the constraints on cosmology when employing redshift tomography as a technique to map the WL feature space \citep{HutererWLSys,PetriPhotoZ} more in depth. If we assign the redshift $z_{\rm ph}$ to a galaxy which has a real redshift of $z_s$ during the feature forward modeling process, we must consider the possibility that this forward model is wrong. 
%
\begin{figure}
\begin{center}
\includegraphics[scale=0.4]{Figures/png/photoz_bias_Om-w.png}\includegraphics[scale=0.4]{Figures/png/photoz_constraints_Om-w_lensing_cmb.png}
\end{center}
\caption{Left panel: bias on $(\Omega_m,w_0)$ induced by photometric redshift errors, obtained using the $\kappa$ tomographic power spectrum (red), peak counts (green) and moments (blue). We show the bias values for 20 independent LSST--like survey realizations (crosses) and indicate the mean of $\h{p}^{\rm ph}-\h{p}$ as a square. For reference, we draw the $1\sigma$ (68\% confidence level) ellipses on the bias $\h{p}^{\rm ph}-\h{p}$ assuming its distribution is Gaussian. Right panel: $1\sigma$ confidence ellipses on $(\Omega_m,w_0)$ obtained from tomographic features (color coded in the legend). We the constraints without (thin lines) and with (thick lines) Planck \citep{Planck15} priors included via equation (\ref{eq:5:linapprox-cov-prior}). Feature covariance matrices have been measured from $N_r=16,000$ realizations of the shear catalogs.}
\label{fig:7:photoz}
\end{figure}
%
To study the importance of this effect, we divided the source galaxies in $N_z=5$ redshift bins and we used the \LT\,functionality ray--trace $\pmb{\gamma}$ to the real redshift $z_s$ of each galaxy. This operation produces multiple realizations (see \S~\ref{sec:3:sampling}) for different $\Lambda$CDM cosmologies (we refer to this dataset as to the \ttt{LSST100Parameters} simulations). The shear catalogs are then converted into tomographic images via a binning procedure defined by

\begin{equation}
\label{eq:7:gridding}
\pmb{\gamma}(\pt_p,\bar{z}_b) = \frac{\sum_{g=1}^{N_g}\pmb{\gamma}(\pt_g,z_g)W(\pt_g,\pt_p;z_g,\bar{z}_b)}{\sum_{g=1}^{N_g}W(\pt_g,\pt_p;z_g,\bar{z}_b)}.
\end{equation}  
%
In equation (\ref{eq:7:gridding}), $\pt_g,z_g$ denote the galaxy position and redshift respectively and $\pt_p,\bar{z}_b$ indicate the pixel position on the image and the center of the redshift bin (see Figure \ref{fig:7:galdistr}). We chose the window function $W$ to be a top--hat: 

\begin{equation}
\label{eq:7:that}
W(\pt_g,\pt_p;z_g,\bar{z}_b) = 
\begin{cases}
1 \,\,\,\, {\rm if} \,\,\,\, \pt_g\in\pt_p,z_g\in\bar{z}_b \\
0 \,\,\,\, {\rm otherwise}.
\end{cases}
\end{equation}
%
We then applied the KS inversion procedure (\ref{eq:2:kappa-ks}) to each of the 5 $\pmb{\gamma}(\pt_p,\bar{z}_b)$ images to obtain the convergence field  $\kappa(\pt_p,\bar{z}_b)$ (a smoothing factor of $e^{-\ell^2\theta_G^2/2}$ with $\theta_G=0.5'$ has been applied during the KS inversion for convenience). We extracted image features with the techniques described in Chapter \ref{chp:4}, with the additional tomographic classification of the source galaxies in different redshift bins $\{\bar{z}_b\}$. We define the cross--redshift $\kappa$ power spectrum $P_{\kappa\kappa}(\ell,\bar{z}_b,\bar{z}_{b'})$ as 

\begin{equation}
\label{eq:7:crosspower}
\left\langle\tilde{\kappa}(\pell,\bar{z}_b)\tilde{\kappa}(\pell',\bar{z}_{b'})\right\rangle = (2\pi)^2\delta^D(\pell+\pell')P_{\kappa\kappa}(\ell,\bar{z}_b,\bar{z}_{b'})
\end{equation} 
%
Note that, when introducing tomography, the dimensionality of the feature space $N_d$ defined by $P_{\kappa\kappa}$ increases from $N_d$ to $N_d N_z(N_z-1)/2$. For higher order $\kappa$ features, which are not quadratic in $\kappa$, we join the vectors $\bb{d}$ measured in different bins $\bar{z}_b$, increasing the dimensionality of the feature space from $N_d$ to $N_dN_z$. Dimensionality reduction techniques become especially relevant when considering tomographic features because, given the increased dimensionality, the constraint degradation pitfalls described in \S~\ref{sec:5:degrade}  become important. 

In order to study the effects of photometric redshift errors, we took the simulated shear catalogs in the fiducial cosmology and replaced each redshift $z_s$ with an estimate $z_{\rm ph}$ based on photometry. The estimate was obtained using equation (\ref{eq:7:phreal}). We then performed a $\kappa$ reconstruction with equation (\ref{eq:7:gridding}), we measured the features from the images and we inferred $\Lambda$CDM parameters using equations (\ref{eq:5:linapprox-peak}), (\ref{eq:5:linapprox-cov}). We quantified the bias induced on the inference on a parameter $p$ by uncorrected photometric redshift errors as $\h{p}^{\rm ph}-\h{p}$, where $\h{p}^{\rm ph},\h{p}$ denote parameter estimates from mock observations with and without redshift errors. We show the results in Figure \ref{fig:7:photoz}. The plot shows that photometric redshift errors, if left uncorrected, cause significant bias in the parameters when using polynomial features such as the power spectrum and the moments of $\kappa$. Peak counts, on the other hand, are less affected by these systematics, likely because they probe correlations between shapes galaxy that are very close to each other on the sky. These correlations are affected to a lesser extent by photometric redshift errors, which are spatially uncorrelated. Since the bias for different features appears to point in different feature space directions, the possibility of self--calibration may be considered in the future.

The right panel of Figure \ref{fig:7:photoz} shows constraint forecasts on $\Omega_m$ and $w_0$ coming from WL tomography. The Figure shows that the combination of $\kappa$ power spectrum, moments and peaks can in principle constrain $w_0$ to a percent level.   

\section{Born approximation}
\label{sec:7:born}
In the previous sections we focused on the bias arising from observational systematics. In this section, on the other hand, we study a potential source of error due to the approximate theoretical modeling of WL features, namely the Born approximation. If one truncates the forward model for $\kappa$ to first order in the gravitational potential $\Phi$, equation (\ref{eq:3:kappa-fo-num-2}) is sufficient for the calculation. The Born approximation is faster than exact ray--tracing (based on (\ref{eq:3:jackp1-delta})) because the knowledge of the density contrast $\delta$ is sufficient for the computation. For full ray--tracing, on the other hand, $\Phi$ is necessary to compute the ray deflection angles and hence the solution to the Poisson equation (\ref{eq:3:poisson-psi-2}) is needed. 
%
\begin{table}
\begin{center}
\begin{tabular}{c|c|c|c}
\textbf{Solver} & \textbf{Runtime (1 FOV)} & \textbf{Memory usage} & \textbf{CPU time (1000 FOV)} \\ \hline \hline
Born & 36.0\,s & 0.86\,GB & 10\,hours  \\
Full ray--tracing & 124.8\,s & 1.65\,GB & 35\,hours  \\
Born + $O(\Phi^2)$ & 156.7\,s & 1.52\,GB & 44\,hours \\ \hline
\end{tabular}
\end{center}
\caption{CPU time and memory usage benchmarks for $\kappa$ reconstruction. The test case we considered consists in a discretization with $N_l=42$ uniformly spaced lenses between the observer and the sources at $z_s=2$, each with a resolution of $4096^2$ pixels. The $\kappa$ field is resolved with $2048^2$ light rays. We show both the runtime for producing a single field of view and the CPU hours needed to perform the reconstruction 1000 times, which is the amount of time needed to mock an LSST--like galaxy survey. Run times do not include the Poisson solution calculation, as this can be recycled to produce multiple field of view realizations (see \S~\ref{sec:3:sampling}). The Poisson solution run time is negligible in the account of the total CPU time needed for the production of $N_r\gg N_l$ WL realizations.}
\label{tab:7:benchmarks}
\end{table}
% 
\begin{figure}
\begin{center}
\includegraphics[scale=0.255]{Figures/png/bornBias_convergence_powerSN30_s0_nb100.png}
\includegraphics[scale=0.255]{Figures/png/bornBias_convergence_momentsSN15_s50_nb9.png}
\includegraphics[scale=0.255]{Figures/png/bornBias_convergence_momentsSN30_s50_nb9.png}
\end{center}
\caption{Distribution of parameter estimates for the triplet $(\Omega_m,w_0,\sigma_8)$, obtained with (\ref{eq:5:linapprox-peak}) using a variety of $\kappa$ features which include the power spectrum and higher moments in real space. The observation to fit has been generated with full ray--tracing and the forward model, based on the feature derivatives $\bb{M}$, has been obtained with both the Born approximation (green bars), and exact ray--tracing (for the sake of null testing, blue bars). Forward models and covariance matrices have been estimated from ensembles of 8192 $\kappa$ realizations, and  mock measured features have been generated averaging over 1000 realizations, to mimic the area of an LSST--like survey. The $\bbh{p}_0$ samples were drawn with a bootstrapping procedure. The WL ensembles on which this study is based are taken from the \ttt{DEBatch} simulation suite (see Appendix).}
\label{fig:7:biasfeat}
\end{figure}
%
\begin{figure}
\begin{center}
\includegraphics[scale=0.6]{Figures/png/bornBias_ngal_convergence_powerSN_s0_nb100.png}
\end{center}
\caption{Statistical significance of the bias induced by the Born approximation on the $\Omega_m$(blue), $w_0$(red) and  $\sigma_8$ (green) inferences obtained from the $\kappa$ power spectrum, as a function of the survey galaxy angular density $n_g$. The averaged results refer to an ensemble of 1000 bootstrapped realizations of an LSST--like galaxy survey.}
\label{fig:7:biasbornngal}
\end{figure}
%
Table \ref{tab:7:benchmarks} (taken from \citep{PetriBorn}) shows that, using the \LT\,implementation, one can save as much as a factor of 4 in CPU time when computing $\kappa$ using the Born approximation. These time savings, however, come at a price since the forward model (\ref{eq:5:linapprox}) and the matrix $\bb{M}$ are accurate only at $O(\Phi)$. When using approximate forward models to fit observations via (\ref{eq:5:linapprox-peak}), depending on the particular image feature used, one may induce bias in the inference of parameters. This possibility was studied in \citep{PetriBorn}, from which we take Figure \ref{fig:7:biasfeat}. The plot shows the distribution of parameter estimates $\bbh{p}_0$ obtained with exact and Born approximated forward models. We can clearly see that inaccuracies due to the Born approximation do not lead to significant bias in the constraints obtained from the $\kappa$ power spectrum. This conclusion is valid for an LSST like survey with a galaxy density of $30\,{\rm galaxies/arcmin}^2$ and holds for densities as high as $60\,{\rm galaxies/arcmin}^2$, as suggested by Figure \ref{fig:7:biasbornngal}. Figure \ref{fig:7:biasfeat} also shows that the Born approximation does not predict $\kappa$ moments with sufficient accuracy, because the induced bias in $w_0$ and $\sigma_8$ is significant. The bias persists even when Gaussian shape noise is added to the images: higher $\kappa$ moments are sensitive to non--Gaussian statistical information in the $\kappa$ field, which has a distinct signature even when Gaussian shape noise is introduced. As we conclude in \S~\ref{sec:6:omsi8}, $\kappa$ moments contain significant cosmological information. Because of this, in the analysis of a WL survey with the statistical power of LSST, the Born approximation does not predict $\kappa$ higher moments to sufficient accuracy, and an exact approach based on ray--tracing is needed. 

\section{Other systematic effects}
In this section we briefly overview some of the systematic effects that we did have the chance to investigate in this work, but that might be important for future analysis of WL observations. In \S~\ref{sec:7:ccd} we discussed how CCD imperfections generate spurious contributions to the convergence and we isolated two effects, the tree rings and the variations of pixel sizes, which have negligible impact on parameter inference. There is another effect which influences CCD operations and is worth mentioning: the so called \textit{brighter--fatter} effect \citep{BrightFat}. The response of CCD sensors to the flux of source galaxies is not linear: charge accumulation on the surface of the CCD induces artificial distortions in the images, which have a net effect on the $\kappa$ reconstruction procedure. These artificial deformations are more severe when observing brighter sources.

Another systematic effect worth mentioning has to do with the way one interprets the correlations between the shapes of nearby galaxies: $\kappa$ is inferred with the KS inversion procedure (\ref{eq:2:kappa-ks}) under the assumption that the ellipticity of the image is caused by cosmic shear. Intrinsic galaxy ellipticity is taken into account adding a white noise component to $\kappa$ using equation (\ref{eq:2:shapenoise}). This treatment, however, completely ignores the fact that galaxies are partially aligned by the Large Scale Structure of the universe, and hence their shapes present \textit{intrinsic alignments} (see \citep{IAReview} for a review on the effect). This alignment is usually modeled as an additive contribution $\pmb{\gamma}^I$ to the WL shear but, contrary to shape noise, $\pmb{\gamma}^I$ is spatially correlated. Analytical models for $\pmb{\gamma}^I$ based on the tidal gravitational field have been explored in the literature \citep{IATidal}. The effect of ignoring intrinsic alignment on $\Lambda$CDM inferences using power spectra has also been explored by \citep{IABias} and has been proven to be non negligible for large surveys such as LSST. The effects of intrinsic alignments hence need to be mitigated in order to avoid bias. 

The last effect we mention in this section has to do with baryon effects. The $\kappa$ forward modeling pipeline we made use of relies on Dark Matter only $N$--body simulations, which are relatively straightforward to run thanks to the collision--free nature of Dark Matter particles. In the real Universe, however, baryons with non zero pressure have non negligible effects on small scales. A variety of studies on the effects of baryon physics can be found in the literature. These include investigations of baryon physics on matter power spectra \citep{BaryonsZentner1,BaryonsZentner2}, WL power spectra \citep{BaryonsWhite,BaryonsKnox}, two and three--point shear statistics \citep{BaryonSemboloni} and WL peak counts \citep{BaryonXiuyuan}. Forward modeling pipelines that include baryons add additional computational complexity to the $N$--body simulations, as hydrodynamic approaches need do be adopted in order to model pressure effects correctly. Effects due to AGN and Supernovae feedback are currently under theoretical investigation and pose additional challenges.      

%\bibliography{ref}