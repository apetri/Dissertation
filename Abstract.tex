\chapter*{Abstract} 

The Standard Model of cosmology successfully describes the observable Universe requiring only a small number of free parameters. The model has been validated by a wide range of observable probes such as Supernovae IA, the CMB, Baryonic Acoustic Oscillations and galaxy clusters. Weak Gravitational Lensing (WL) is becoming a popular observational technique to constrain parameters in the Standard Model and is particularly appealing to the scientific community because the tracers it relies on, image distortions, are unbiased probes of density fluctuations in the fabric of the cosmos. Because of the WL sensitivity to the late time evolution of the Universe, in which structures are non--linear, observations cannot be treated as Gaussian random fields and statistical information on cosmology leaks from quadratic correlations into more complicated, higher order, image features. The goal of this dissertation is to analyze the efficiency of some of these higher order features in constraining Standard Model parameters. We approach the investigation from a practical point of view, examining the analytical, computational and numerical accuracy issues that are involved in carrying a complete analysis from observational data to parameter constraints using these higher order statistics. This work is organized as follows:

\begin{itemize}

	\item Chapter \ref{chp:1}: We review the fundamentals of the $\Lambda$CDM Standard Model of cosmology, focusing particularly on the Friedmann picture and on the physics of large scale density fluctuations.
	
	\item Chapter \ref{chp:2}: We give an outline of the Weak Gravitational Lensing effect in the context of cosmology, and we introduce the basic WL observables from an analytical point of view.
	
	\item Chapter \ref{chp:3}: We review the most relevant numerical techniques used in the modeling of Weak Lensing observables, focusing in particular on the algorithms used in ray--tracing simulations. These simulations constitute the base of our modeling efforts.  
	
	\item Chapter \ref{chp:4}: We discuss image feature extraction techniques from WL observations: we treat both quadratic statistics, such as the angular shear--shear power spectrum, and higher order statistics for which analytical treatment is not possible.
	
	\item Chapter \ref{chp:5}: We review the Bayesian formalism behind the inference of $\Lambda$CDM parameters from image features. We pose particular emphasis on physical and numerical effects that can degrade parameter constraints and discuss possible mitigations.
	
	\item Chapter \ref{chp:6}: We apply the previously described techniques to the Canada France Hawaii LenS galaxy survey, showing how the use of higher order image statistics can improve inferences on the $\Lambda$CDM parameters which describe density fluctuations. 
	
	\item Chapter \ref{chp:7}: We discuss some of the issues that can arise in WL analyses of a large scale survey such as the Large Scale Synoptic Survey: we focus on systematic effects caused by sensors imperfections, atmospheric effects, redshift measurements and approximate theoretical modeling.
	
	\item Chapter \ref{chp:8}: We draw our conclusions and discuss possible future developments.

\end{itemize}