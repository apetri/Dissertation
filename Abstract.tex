\chapter*{Abstract} 

Weak Gravitational Lensing (WL) is becoming a popular observational technique to constrain the Physics of the Universe, and is particularly appealing because the tracer it relies on, image distortions, are unbiased probes of density fluctuations in the fabric of the cosmos. Because WL probes the late time evolution of the Universe, in which structures are non--linear, observational data cannot be modeled as Gaussian random fields, and hence physical information on cosmological parameters leaks from the two point correlation function into more complicated, higher order, image features. The goal of this dissertation is to analyze the efficiency of some of these higher order features in constraining cosmology. We approach the problem from a practical point of view, examining the analytical, computational and numerical accuracy issues that are involved in carrying a complete analysis from observational data to parameter constraints. This work is organized as follows:

\begin{itemize}
	\item Chapter \ref{chp:1}: We review the fundamentals of the $\Lambda$CDM cosmological model
	\item Chapter \ref{chp:2}: We review the Weak Gravitational Lensing effect in the context of cosmology
	\item Chapter \ref{chp:3}: We approach Weak Lensing from a numerical point of view, and describe the algorithms used in the simulations on whose this work is based
	\item Chapter \ref{chp:4}: We discuss the image features used to constrain $\Lambda$CDM parameters, with emphasis on higher order statistics
	\item Chapter \ref{chp:5}: We review the parameter inference formalism, which allows to go from observations to parameter constraints. We pose particular focus on physical and numerical effects that can degrade and bias parameter constraints
	\item Chapter \ref{chp:6}: We apply the described techniques to the CFHTLenS galaxy survey
	\item Chapter \ref{chp:7}: We discuss the application to the LSST survey, tackling some of the issues that arise for a large scale survey
	\item Chapter \ref{chp:8}: We draw our conclusions and discuss possible future developments
\end{itemize}