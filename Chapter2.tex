%%%%%%%%%%%%%%%%%%%%%%%%%%%%%%%%%%%%%%%%%%%%%%%%%%%%%%%%%%%%%%%%%%%%%%%%%%%

\chapter{Gravitational Lensing}
\lhead[\fancyplain{}{\thepage}]{\fancyplain{}{\rightmark}}
 \thispagestyle{plain}
\setlength{\parindent}{10mm}

%%%%%%%%%%%%%%%%%%%%%%%%%%%%%%%%%%%%%%%%%%%%%%%%%%%%%%%%%%%%%%%%%%%%%%%%%%%

In this Chapter we review the basics of the Gravitational Lensing (GL) effect. GL is a prediction of General Relativity and states that light rays that travel through space--time inhomogeneities experience trajectory deflections. We start by choosing a convenient Gauge in which solving the Einstein equation to derive the lensing equation. We explore the full solution of the lensing ODE, as well as some convenient approximate forms. We conclude the Chapter by connecting the lensing formalism to Weak Lensing (WL) simulations and observations. 

\section{The lensing equation}

\section{Weak Gravitational Lensing}

\subsection{The Born approximation}   

%\bibliography{ref}