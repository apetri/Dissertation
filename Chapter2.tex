%%%%%%%%%%%%%%%%%%%%%%%%%%%%%%%%%%%%%%%%%%%%%%%%%%%%%%%%%%%%%%%%%%%%%%%%%%%

\chapter{Gravitational Lensing}
\lhead[\fancyplain{}{\thepage}]{\fancyplain{}{\rightmark}}
 \thispagestyle{plain}
\setlength{\parindent}{10mm}

%%%%%%%%%%%%%%%%%%%%%%%%%%%%%%%%%%%%%%%%%%%%%%%%%%%%%%%%%%%%%%%%%%%%%%%%%%%

In this Chapter we review the basics of the Gravitational Lensing (GL) effect. GL is a prediction of General Relativity and states that light rays that travel through space--time inhomogeneities experience trajectory deflections. We start by deriving an equation for the light ray geodesics in a non homogeneous background, following the derivation in \citep{Dodelson-C11}. We then specialize the geodesic solution to the Weak Lensing (WL) case, exploring approximate approaches such as the Born approximation.

\section{Light ray geodesics}

\subsection{Geodesic equation} 

We parametrize a light ray space--time trajectory $x^\mu(\tau)$ in terms of a continuous real parameter $\tau$, which plays the same role as proper time for massive particles. The geodesic equation (\ref{eq:1:geodesic}) can be rephrased as 

\begin{equation}
\label{eq:2:geodesic}
\frac{d^2 x^\mu(\tau)}{d\tau^2} = -\Gamma_{\alpha\beta}^\mu(x^\mu(\tau)) \frac{d x^\alpha(\tau)}{d\tau}\frac{d x^\beta(\tau)}{d\tau}
\end{equation}
%
For the sake of expressing WL observables at first order in the potentials $\Phi,\Psi$, it is sufficient to use the linear expressions (\ref{eq:1:connection}) for the affine connection $\Gamma^\mu_{\alpha\beta}$ at first order. We will make an argument later in the Chapter for higher order post--Newtonian (PN) corrections to be negligible for the scope of this work, following the conclusions of \citep{PNLensing}. We introduce a system of coordinates centered on a fundamental observer on Earth, as illustrated in Figure \ref{fig:2:scheme}

\begin{equation}
\label{eq:2:coord}
x^{\mu} = (ct,\chi,\xp)
\end{equation}
%
Where we indicated the transverse coordinates (with respect of the observer) corresponding to an angle on the sky $\pt$ as $\xp=\chi\pt$. We will adopt the so called flat sky approximation, in which the angles $\pt$ are small. Since photons travel along null geodesics, their 4-momentum $P^\mu=dx^\mu/d\tau=(p^0,\bb{p})$ must satisfy $g_{\mu\nu}P^{\mu}P^{\nu}=0$, which gives the constraint (at first order in $\Psi$)

\begin{equation}
\label{eq:2:nullconstr}
p^0 = c\frac{dt}{d\tau} = p(1-\Psi)
\end{equation} 
%
with $p=\vert\bb{p}\vert$. Using the fact that $d\chi/dt = -c/a$, we can replace the $\tau$ derivatives in (\ref{eq:2:geodesic}) with $\chi$ derivatives using the prescription
\begin{equation}
\label{eq:2:tau2chi}
\frac{d}{d\tau} = \frac{dt}{d\tau}\frac{d\chi}{dt}\frac{d}{d\chi} = -\frac{p(1-\Psi)}{a}\frac{d}{d\chi}
\end{equation}
%
If we focus on the transverse components, the LHS of (\ref{eq:2:geodesic}) looks like
\begin{equation}
\label{eq:2:geodesic-lhs-1}
\frac{d^2\xp}{d\tau^2} = \frac{p}{a}\frac{d}{d\chi}\left(\frac{p}{a}\frac{d\xp}{d\chi}\right)
\end{equation}
%
Following \citep{Dodelson-C11}, we dropped small terms of order $\Psi d\xp$. At dominant order the photon momentum $a$ redshifts as $1/a$, and hence we can pull the product $pa$ out of the differentiation, obtaining 
\begin{equation}
\label{eq:2:geodesic-lhs-2}
\frac{d^2\xp}{d\tau^2} = p^2\frac{d}{d\chi}\left(\frac{1}{a^2}\frac{d\xp}{d\chi}\right)
\end{equation}
%
Now we can focus on the RHS of (\ref{eq:2:geodesic}) which reads, in the transverse components
\begin{equation}
\label{eq:2:geodesic-rhs-1}
\Gamma_{\alpha\beta}^i \frac{d x^\alpha}{d\tau}\frac{d x^\beta}{d\tau} = \frac{p^2}{a^2} \Gamma_{\alpha\beta}^i \frac{d x^\alpha}{d\chi}\frac{d x^\beta}{d\chi}
\end{equation}
%
If we expand the products in (\ref{eq:2:geodesic-rhs-1}) using the affine connection (\ref{eq:1:connection}) at first order in the potentials, we obtain
\begin{equation}
\label{eq:2:geodesic-rhs-2}
\Gamma_{\alpha\beta}^i \frac{d x^\alpha}{d\tau}\frac{d x^\beta}{d\tau} = \frac{p^2}{a^2}\left[\partial_i\Psi-\frac{2aH}{c}\frac{d\xp}{d\chi} -\partial_i\Phi \right]
\end{equation}
%
The complete geodesic equation now becomes 
\begin{equation}
\label{eq:2:geodesic-fo-1}
\frac{d}{d\chi}\left(\frac{1}{a^2}\frac{d\xp}{d\chi}\right) = -\frac{1}{a^2}\left[\nabla_\perp(\Psi-\Phi)-\frac{2aH}{c}\frac{d\xp}{d\chi}\right]
\end{equation}
%
After a few simplifications this assumes the form 

\begin{equation}
\label{eq:2:geodesic-fo-2}
\frac{d^2\xp(\chi)}{d\chi^2} = \nabla_\perp(\Phi(\chi,\xp)-\Psi(\chi,\xp))
\end{equation}
%
\begin{figure}
\begin{center}
\includegraphics[scale=0.75]{Figures/pdf/rayscheme.pdf}
\end{center}
\caption{Coordinate system centered on a fundamental observer on Earth}
\label{fig:2:scheme}
\end{figure}
%
Using the relation (\ref{eq:1:einstein-ij}) between the gravitational potentials, which allows us to substitute $\Psi=-\Phi$, equation (\ref{eq:2:geodesic-fo-2}) becomes 

\begin{equation}
\label{eq:2:geodesic-fo-3}
\frac{d^2\xp(\chi)}{d\chi^2} = 2\nabla_\perp\Phi(\chi,\xp(\chi))
\end{equation}
%
The potential $\Phi$ satisfies the Poisson equation (\ref{eq:1:poisson}), which we can rewrite in our coordinate system (\ref{eq:2:coord}) as

\begin{equation}
\label{eq:2:poisson}
\nabla^2\Phi(\chi,\xp) = -\frac{4\pi Ga(\chi)^2}{c^2}\rho_m(\chi)\delta(\chi,\xp)
\end{equation} 
%
We substituted the time dependency with a $\chi$ dependency using the time--redshift relation (\ref{eq:1:t-z}). 

\subsection{Solution to the geodesic equation}
The geodesic equation (\ref{eq:2:geodesic-fo-3}) is a second order differential equation which can be solved exactly once suitable initial conditions are specified. If we indicate with $\pt$ the angular position of the light ray as it is received by the observer we have $\xp(0)=0$ and $d\xp /d\chi(0)=\pt$. Equation (\ref{eq:2:geodesic-fo-3}) can hence be solved by double integration

\begin{equation}
\label{eq:2:geosol-1}
\xp(\chi,\pt) = \chi\pt + 2\int_0^\chi d\chi'\int_0^{\chi'}d\chi'' \nabla_\perp\Phi(\chi,\xp(\chi'',\pt))
\end{equation} 
%
Note that we can exchange the order of the integration in $\chi'$ and $\chi''$ by taking advantage of the triangular shape of the integration domain, and perform one of the integrations analytically to get

\begin{equation}
\label{eq:2:geosol-2}
\xp(\chi,\pt) = \chi\pt + 2\int_0^\chi d\chi'(\chi-\chi')\nabla_\perp\Phi(\chi',\xp(\chi',\pt))
\end{equation} 
%
We expressed the solution (\ref{eq:2:geosol-2}) to (\ref{eq:2:geodesic-fo-3}) in implicit form, as the RHS contains $\xp(\chi,\pt)$ itself. An implicit expression like (\ref{eq:2:geosol-2}) however, present some advantages: first of all, in order to express the solution at some $\chi_s$, we only need to know $\xp$ for $\chi<\chi_s$, making (\ref{eq:2:geosol-2}) solvable numerically via dynamic programming. Moreover, because the potential $\Phi$ appears in the RHS, in the limit in which $\Phi$ is small, the solution in implicit form suggest a straightforward perturbative approximation in powers of $\Phi$, which will be explored later in the Chapter. In the next section, we connect the light ray geodesic (\ref{eq:2:geosol-2}) to the main WL observables.  


%%%%%%%%%%%%%%%%%%%%%%%%%%%%%%%%%%%%%%%%%%%%%%%%%%%%%%%%%%%%%%%%%%%%%%%%%%%%%%%%%%%%%%%%%%%%%%%%%%%%%%%%%%%%%%%%%%%%%%%%%%%

\section{Weak Gravitational Lensing}

\subsection{Weak Lensing observables}
Gravitational Lensing produces apparent distortions in observed background sources shapes due to light ray deflections. Following the notation in Figure \ref{fig:2:scheme}, a light ray captured by the observer at position $\pt$, in reality originated from a point that on the sky corresponds to an angular position $\pbeta$. Overall shifts in the relation $\pbeta(\pt)$ do not alter the source shape (that we assume distributed on a plane at $\chi=\chi_s$), but simply cause unobservable image displacements. The lowest order image distortions come from the differential ray shift $\partial \beta_i/\partial \theta_j$, which alter the observed ellipticity of the sources. Higher order flexion corrections to source shapes have been investigated in the literature \citep{BornFlexion}, but will not be treated in this work. Elliptical deformations in the shape of background sources are parametrized in terms of deflection Jacobian $\bb{A}$

\begin{equation}
\label{eq:2:dfl-jacobian}
A_{ij}(\chi_s,\pt) = \frac{\partial \beta_i(\chi_s,\pt)}{\partial \theta_j} = 
\begin{pmatrix}
1-\kappa(\pt)-\gamma^1(\pt) & -\gamma^2(\pt)+\omega(\pt) \\
-\gamma^2(\pt)-\omega(\pt) & 1-\kappa(\pt)+\gamma^1(\pt)
\end{pmatrix}
\end{equation}  
%
\begin{figure}
\begin{center}
\includegraphics[scale=0.35]{Figures/eps/distortion.eps}
\end{center}
\caption{Effect of differential distortions $\kappa,\pmb{\gamma},\omega$ on a circular image}
\label{fig:2:distortion}
\end{figure}
%
In this parametrization, $\kappa$ is called the WL convergence, $\pmb{\gamma}=(\gamma^1,\gamma^2)$ is the WL cosmic shear and $\omega$ is the WL rotation. Inverting equation (\ref{eq:2:dfl-jacobian}) we obtain the relations

\begin{equation}
\label{eq:2:dfl-inverse}
\begin{matrix}
\kappa = 1-\frac{1}{2}\Tr \bb{A} & ; & \gamma^1 = \frac{1}{2}(A_{yy} - A_{xx}) \\
\gamma^2 = -\frac{1}{2}(A_{xy}+A_{yx}) & ; & \omega = \frac{1}{2}\Tr (\bb{A}\pmb{\varepsilon})
\end{matrix}
\end{equation}
% 
Figure \ref{fig:2:distortion} shows the physical interpretation of these different types of distortions, for an un--lensed circular image. The convergence has to do with the background source apparent magnifications, and is not directly observable unless the un--lensed size of the source is known in advance. The cosmic shear $\pmb{\gamma}$ encodes the ellipticity of the distortion, while the rotation $\omega$ is the angular tilt in the distorted image shape. Calling $I(\pbeta)$ and $I_{\rm obs}(\pt)$ the emitted and observed source intensity profiles respectively, we can define the observable ellipticity $\pmb{\epsilon}$ of the distorted image in terms of the quadratic moments of the intensity 

\begin{equation}
\label{eq:2:ellipticity-1}
\begin{matrix}
\epsilon^1 = \frac{q_{xx}-q_{yy}}{q_{xx}+q_{yy}} & ; & \epsilon^2 = \frac{2q_{xy}}{q_{xx}+q_{yy}}
\end{matrix}
\end{equation} 
%
Where the quadrupole moments $q_{ij}$ are defined by 
\begin{equation}
\label{eq:2:quadrupole}
q_{ij} = \int d\pt \theta_i\theta_j I_{\rm obs}(\pt) = \int d\pbeta (\bb{A}^{-1}\pbeta)_i(\bb{A}^{-1}\pbeta)_j I(\pbeta)
\end{equation}
%
In the last equality flexion corrections have been ignored. If the un--lensed image is a circle and the image is small enough so that $\kappa,\pmb{\gamma}$ are constant over its profile, the integral can be computed explicitly in terms of the inverse of $\bb{A}$

\begin{equation}
\label{eq:2:ellipticity-2}
\begin{matrix}
\epsilon^1 = \frac{(A^{-1}_{xx})^2-(A^{-1}_{yy})^2}{(A^{-1}_{xx})^2+(A^{-1}_{yy})^2+2(A^{-1}_{xy})^2} & ; & \epsilon^2 = \frac{2A^{-1}_{xy}(A^{-1}_{xx}+A^{-1}_{yy})}{(A^{-1}_{xx})^2+(A^{-1}_{yy})^2+2(A^{-1}_{xy})^2} &
\end{matrix}
\end{equation}
%
If we express the numerator and denominator at lowest order in $\kappa,\pmb{\gamma}$ we get immediately
\begin{equation}
\label{eq:2:ellipticity-3}
\pmb{\epsilon} = \frac{2\pmb{\gamma}}{1-\kappa}
\end{equation}
%
which shows that the ellipticity is proportional to the shear and can hence be used to estimate it. Even if the observable quantity is the reduced shear $\pmb{\gamma}/(1-\kappa)$, for practical purposes in WL when $\kappa\ll 1$, one can approximate $\pmb{\epsilon}\approx 2\pmb{\gamma}$. In the following paragraph we will show how to relate the WL quantities $\kappa,\pmb{\gamma},\omega$ to the gravitational lensing potential $\Phi$. 

\subsection{Ray tracing}
Relating observable WL quantities such as the source ellipticity $\pmb{\epsilon}$ and the cosmic shear $\pmb{\gamma}$ can be done using equation (\ref{eq:2:geosol-2}), which relates light geodesics to the matter density fluctuations through the lensing potential. We can convert the transverse physical coordinates $\xp$ into angles $\pbeta$ dividing by the longitudinal distance

\begin{equation}
\label{eq:2:geosol-beta}
\pbeta(\chi,\pt) = \pt + 2\int_0^\chi d\chi'\left(1-\frac{\chi'}{\chi}\right)\nabla_\perp\Phi(\chi',\chi'\pbeta(\chi',\pt))
\end{equation}
%
And by differentiating (\ref{eq:2:geosol-beta}) with respect to $\pt$ we can obtain a similar relation for the Jacobian $\bb{A}$
\begin{equation}
\label{eq:2:geosol-jac}
A_{ij}(\chi,\pt) = \delta_{ij} + 2\int_0^\chi d\chi'\chi'\left(1-\frac{\chi'}{\chi}\right)T^{\Phi}_{ik}(\chi',\chi'\pbeta(\chi',\pt))A_{kj}(\chi',\pt)
\end{equation}
%
Where we defined the tidal tensor $T^{\Phi}_{ij}=\partial_i\partial_j\Phi$. Like (\ref{eq:2:geosol-2}), equation (\ref{eq:2:geosol-jac}) is an implicit relation that allows to calculate $\bb{A}(\chi_s,\pt)$ at some $\chi_s$ once all $\bb{A}(\chi,\pt)$ are known for $\chi<\chi_s$. A numerical solution to this problem that makes use of the multi--lens--plane algorithm \citep{RayTracingJain,RayTracingHartlap} will be explored in the next Chapter. 
The fact that the potential $\Phi$ appears on the RHS of (\ref{eq:2:geosol-jac}) suggests the possibility of a perturbative expansion of $\bb{A}$ in powers of $\Phi$. Confidence in the validity of the perturbative approach comes from the fact that in the WL limit $\Phi$ is small. If we focus in at most quadratic contributions in $\Phi$, we can write

\begin{equation}
\label{eq:2:jacobian-pert}
A_{ij}(\chi,\pt) = \delta_{ij} + A_{ij}^{(1)}(\chi,\pt) + A_{ij}^{(2)}(\chi,\pt) + O(\Phi^3)
\end{equation}
%
To obtain an expression at linear order in $\Phi$, we can replace Jacobian of the RHS of (\ref{eq:2:geosol-jac}) with the identity and the spatial argument of $\Phi$ with $\chi\pt$. This approach is analogous to the Born approximation commonly used in Quantum Mechanics when computing scattering amplitudes at first order in the interaction potential. We can write

\begin{equation}
\label{eq:2:jacobian-fo}
A_{ij}^{(1)}(\chi,\pt) = 2\int_0^\chi d\chi'\chi'\left(1-\frac{\chi'}{\chi}\right)T^{\Phi}_{ij}(\chi',\chi'\pt)
\end{equation}
%
Equation (\ref{eq:2:jacobian-fo}) essentially says that, at lowest perturbative order, the lensing Jacobian is the projection of the tidal field $T^{\Phi}_{ij}$ on the unperturbed geodesic trajectory $\chi\pt$, weighted with a lensing kernel

\begin{equation}
W(\chi',\chi) = \chi'\left(1-\frac{\chi'}{\chi}\right) 
\end{equation}
%
With the use of (\ref{eq:2:dfl-inverse}) we can express the convergence at first order in the lensing potential
\begin{equation}
\label{eq:2:kappa-1}
\kappa^{(1)}(\chi,\pt) = \int_0^{\chi}d\chi' W(\chi',\chi)\delta_L(\chi',\chi'\pt)
\end{equation}
%
Where we defined the lensing density $\delta_L$ as 
\begin{equation}
\label{eq:2:deltalens}
\delta_L(\chi,\xp) = -\nabla^2_\perp \Phi(\chi,\xp)
\end{equation}
%
Equation (\ref{eq:2:kappa-1}) essentially means that the WL convergence is the integrated column density fluctuations on the line of sight between the observer and the source. Note that because the first order WL quantities are proportional to the integrated tidal field, which is symmetric, the WL rotation $\omega$ vanishes as can be seen in equation (\ref{eq:2:dfl-inverse}). 
Quadratic corrections to the linear relation (\ref{eq:2:jacobian-fo}) between $\bb{A}$ and $\Phi$ come from two different sources in equation (\ref{eq:2:geosol-jac}): one term is generated by replacing the Jacobian in the RHS of (\ref{eq:2:geosol-jac}) with its first order approximation $A_{ij}^{(1)}$, the other comes from the transverse argument of the tidal field $T^{\Phi}_{ij}$, which contains the geodesic perturbations. Using the approximation

\begin{equation}
\label{eq:2:tidalapprox}
T^{\Phi}_{ij}(\chi,\chi\pbeta(\chi,\pt)) = T^{\Phi}_{ij}(\chi,\chi\pt) + \chi\partial_kT^{\Phi}_{ij}(\chi,\chi\pt)[\beta_k^{(1)}(\chi,\pt)-\theta_k] + O(\Phi^3) 
\end{equation}
%
we can write a second order expression for $\bb{A}$ 

\begin{equation}
\label{eq:2:jacobian-so}
\bb{A}^{(2)}(\chi,\pt) = \bb{A}^{(2-{\rm ll})}(\chi,\pt) + \bb{A}^{(2-{\rm gp})}(\chi,\pt) 
\end{equation}
%
\begin{equation}
\label{eq:2:jacobian-so-ll}
A_{ij}^{(2-{\rm ll})}(\chi,\pt) = 4\int_0^\chi d\chi'\int_0^{\chi'}d\chi'' W_2(\chi'',\chi',\chi)T^{\Phi}_{ik}(\chi',\chi'\pt)T^{\Phi}_{kj}(\chi'',\chi''\pt)
\end{equation} 
%
\begin{equation}
\label{eq:2:jacobian-so-gp}
A_{ij}^{(2-{\rm gp})}(\chi,\pt) = 4\int_0^\chi d\chi'\int_0^{\chi'}d\chi'' W_2(\chi'',\chi',\chi)\frac{\chi'}{\chi''}\partial_kT^{\Phi}_{ij}(\chi',\chi'\pt)\partial_k\Phi(\chi'',\chi''\pt)
\end{equation} 
%
With $W_2(t,u,v)=W(t,u)W(u,v)$. The term in (\ref{eq:2:jacobian-so-ll}) originates from the lens--lens coupling between the tidal field at different $\chi$, while the term in (\ref{eq:2:jacobian-so-gp}) has to do with the first order perturbations in the light ray geodesic due to matter density fluctuations. Equations (\ref{eq:2:jacobian-so-ll}),(\ref{eq:2:jacobian-so-gp}) can be easily translated into second order expressions for $\kappa$

\begin{equation}
\label{eq:2:kappa-2-ll}
\kappa^{(2-{\rm ll})}(\chi,\pt) = -2\int_0^\chi d\chi'\int_0^{\chi'}d\chi'' W_2(\chi'',\chi',\chi)\Tr [\bb{T}^{\Phi}(\chi',\chi'\pt)\bb{T}^{\Phi}(\chi'',\chi''\pt)]
\end{equation} 
%
\begin{equation}
\label{eq:2:kappa-2-gp}
\kappa^{(2-{\rm gp})}(\chi,\pt) = 2\int_0^\chi d\chi'\int_0^{\chi'}d\chi'' \frac{\chi'W_2(\chi'',\chi',\chi)}{\chi''}\nabla_\perp \delta_L(\chi',\chi'\pt)\cdot\nabla_\perp\Phi(\chi'',\chi''\pt)
\end{equation} 
%
For completeness we should note that the quadratic terms (\ref{eq:2:kappa-2-ll}),(\ref{eq:2:kappa-2-gp}) are not the only ones that contribute to the WL convergence, as at the beginning of the Chapter we derived an approximate form for the light deflection (\ref{eq:2:geodesic-fo-3}) from the geodesic equation (\ref{eq:2:geodesic}). If one includes the PN corrections to (\ref{eq:2:geodesic-fo-3}), as shown in \citep{PNLensing}, additional quadratic terms $\kappa^{(2-{\rm PN})}$ arise

\begin{equation}
\label{eq:2:kappa-2-pn}
\kappa^{(2-{\rm PN})}(\chi,\pt) = \int_0^\chi d\chi' W(\chi',\chi)\left[\vert\nabla_\perp\Phi(\chi',\chi'\pt)\vert^2+\Phi(\chi',\chi'\pt)\nabla_\perp^2\Phi(\chi',\chi'\pt)\right]
\end{equation}
%
Comparing (\ref{eq:2:kappa-2-pn}) with (\ref{eq:2:kappa-2-ll}),(\ref{eq:2:kappa-2-gp}) we can easily see that the former PN corrections to $\kappa$ are suppressed by a factor of order $\lambda_m H/c$, with $\lambda_m$ being a characteristic coherence scale for the matter density perturbations. \citep{PNLensing} suggest that this suppression factor can be safely estimated to be of the order of $\sim 10^{-2}$ at the location where the lensing kernel $W$ usually peaks, making the PN corrections suppressed by a factor of order of $\sim 10^{-4}$. We will neglect these PN corrections to $\kappa$ throughout this work.  

We can also derive an expression for the dominant contribution to the rotation $\omega$ by taking the antisymmetric part of $\bb{A}$, which comes from lens-lens couplings

\begin{equation}
\label{eq:2:omega-2-ll}
\omega^{(2)}(\chi,\pt) = 2\int_0^\chi d\chi'\int_0^{\chi'}d\chi'' W_2(\chi'',\chi',\chi)\Tr[\bb{T}^{\Phi}(\chi',\chi'\pt)\pmb{\varepsilon}\bb{T}^{\Phi}(\chi'',\chi''\pt)]
\end{equation} 
%
In the remainder of the Chapter we will explore an approximate relation between the WL convergence and shear that proves particularly useful when analyzing survey data.

\subsection{$E/B$ mode shear decomposition}
The convergence $\kappa$ and cosmic shear $\pmb{\gamma}$ can be approximately related to each other if one focuses on their $O(\Phi)$ expressions. This will prove useful in reconstructing the non observable $\kappa$ profile from ellipticity observations which directly probe the shear field. Equation (\ref{eq:2:jacobian-fo}) clearly states that, at linear order in the potential $\Phi$, the differential distortion $\bb{A}$ is the second derivative of the projected potential 

\begin{equation}
\label{eq:2:projection-fo-1}
A^{(1)}_{ij}(\chi,\pt) = \partial_i\partial_j \Phi_2(\chi,\pt) \\ \\
\end{equation}
%
\begin{equation}
\label{eq:2:projection-fo-2}
\Phi_2(\chi,\pt) = \int_0^\chi \frac{d\chi'}{\chi'^2} W(\chi',\chi)\Phi(\chi',\chi'\pt)
\end{equation}
%
From the first order expression for the convergence

\begin{equation}
\label{eq:2:kappa-1-lapl}
\kappa^{(1)}(\chi,\pt) = -\frac{1}{2}\nabla_\perp^2 \Phi_2(\chi,\pt)
\end{equation}
%
We can roll back the laplacian and get an approximate relation between convergence and shear, which is exact at first order

\begin{equation}
\label{eq:2:gamma-1-lapl}
\begin{matrix}
\gamma^1(\chi,\pt) = \nabla_\perp^{-2} (\partial_x^2-\partial_y^2) \kappa(\chi,\pt) + O(\Phi^2) \\ \\
\gamma^2(\chi,\pt) = 2\nabla_\perp^{-2} \partial_x\partial_y \kappa(\chi,\pt) + O(\Phi^2)
\end{matrix}
\end{equation}
%
This relation can be written more compactly in Fourier space using the complex shear $\gamma^c=\gamma^1+i\gamma^2$  

\begin{equation}
\label{eq:2:gamma-c-ks}
\tilde{\gamma}^c_{KS}(\chi,\pell) = \left(\frac{\ell_x^2-\ell_y^2+2i\ell_x\ell_y}{\ell_x^2+\ell_y^2}\right) \tilde{\kappa}(\chi,\pell) = e^{2i\phi_{\pell}} \tilde{\kappa}(\chi,\pell)
\end{equation}
%
Where we introduced the Fourier angle $\phi_{\pell}$ defined by $\cos \phi_{\pell} = \ell_x/\ell$, $\sin\phi_{\pell}=\ell_y/\ell$. Equation (\ref{eq:2:gamma-c-ks}) takes the name of Kaiser-Squires (KS) relation between convergence and shear \citep{KaiserSquires}, and can be inverted to reconstruct the convergence from the shear at first order in $\Phi$

\begin{equation}
\label{eq:2:kappa-ks}
\tilde{\kappa}_{KS}(\chi,\pell) = e^{-2i\phi_{\pell}} \tilde{\gamma}^c(\chi,\pell) = \tilde{\gamma}^E(\chi,\pell) + i\tilde{\gamma}^B(\chi,\pell)
\end{equation}
%
\begin{figure}
\begin{center}
\includegraphics[scale=0.6]{Figures/eps/emode.eps}
\end{center}
\caption{$E$-mode shear pattern for positive and negative $\kappa$ peaks}
\label{fig:2:emode}
\end{figure}
%
Where the shear Fourier $E$ and $B$ modes are defined by
\begin{equation}
\label{eq:2:shear-eb}
\begin{matrix}
\tilde{\gamma}^E(\chi,\pell) =  \tilde{\gamma}^1(\chi,\pell)\cos 2\phi_{\pell} +  \tilde{\gamma}^2(\chi,\pell) \sin 2\phi_{\pell}\\ \\ 
\tilde{\gamma}^B(\chi,\pell) = - \tilde{\gamma}^1(\chi,\pell)\sin 2\phi_{\pell} + \tilde{\gamma}^2(\chi,\pell) \cos 2\phi_{\pell}
\end{matrix}
\end{equation}
%
Note that, because of (\ref{eq:2:gamma-1-lapl}) and (\ref{eq:2:shear-eb}) we have that $\tilde{\gamma}^E=\tilde{\kappa}+O(\Phi^2)$ and $\tilde{\gamma}^B=O(\Phi^2)$. In practice one can estimate the convergence $\kappa$ as the $E$-mode of the shear in the WL limit, and use the detection of a large $B$-mode as an indication of systematic effects \citep{PetriSpShear}. Figure \ref{fig:2:emode} shows the KS reconstructed shear pattern corresponding to positive and negative $\kappa$ peaks. In a real observation, the convergence profile is reconstructed using (\ref{eq:2:kappa-ks}) on the shear field estimated from the observed source ellipticities. Because the non--lensed shapes of most galaxies are not circular, equation (\ref{eq:2:ellipticity-3}) cannot be applied directly as an estimator for the shear, because of non--lensed shape noise. This shape noise is usually modeled as a Gaussian, spatially uncorrelated component in each component of the shear \citep{SongKnox}. As a net effect, shape noise acts as a spatially uncorrelated component $\kappa_{\rm SN}$ on top of the WL $\kappa$ generated by lensing. In this work we will use (see again \citep{SongKnox})

\begin{equation}
\label{eq:2:shapenoise}
\left\langle\h{\kappa}_{\rm SN}(z_s,\pt)\h{\kappa}_{\rm SN}(z_s,\pt')\right\rangle = \frac{(0.15+0.035z_s)^2}{n_g}\delta^D(\pt-\pt')
\end{equation} 
%
where $n_g$ is the angular source density. Note the $1/n_g$ scaling in the shape noise root--mean--square value, which is consistent with the Central Limit Theorem.    

%\bibliography{ref}