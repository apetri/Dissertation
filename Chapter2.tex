%%%%%%%%%%%%%%%%%%%%%%%%%%%%%%%%%%%%%%%%%%%%%%%%%%%%%%%%%%%%%%%%%%%%%%%%%%%

\chapter{Gravitational Lensing}
\lhead[\fancyplain{}{\thepage}]{\fancyplain{}{\rightmark}}
 \thispagestyle{plain}
\setlength{\parindent}{10mm}

%%%%%%%%%%%%%%%%%%%%%%%%%%%%%%%%%%%%%%%%%%%%%%%%%%%%%%%%%%%%%%%%%%%%%%%%%%%

In this Chapter we review the basics of the Gravitational Lensing (GL) effect. GL is a prediction of General Relativity and states that light rays that travel through space--time inhomogeneities experience trajectory deflections. We start by deriving an equation for the light ray geodesics in a non homogeneous background, following the derivation in \citep{Dodelson-C11}. We then specialize the geodesic solution to the Weak Lensing (WL) case, exploring approximate approaches such as the Born approximation. We conclude discussing the short comings of these approximate approaches for modeling of WL observations. 

\section{Light ray geodesics}

\subsection{Geodesic equation}
Scalar perturbations to the FRW metric in (\ref{eq:1:frw}) can be parametrized in the conformal Newtonian gauge by two scalar potentials $\Phi,\Psi$ as follows

\begin{equation}
\label{eq:2:confnewt}
ds^2 = -c^2dt^2(1+2\Psi(\bb{x},t)) + a(t)^2d\bb{x}^2(1+2\Phi(\bb{x},t))
\end{equation}
%
For the scope of the present work we can safely ignore vector and tensor perturbations to the FRW metric as their effects are negligible in WL observations. We parametrize a light ray space--time trajectory $x^\mu(\tau)$ in terms of a continuous real parameter $\tau$, which plays the same role as proper time for massive particles. The geodesic equation reads 

\begin{equation}
\label{eq:2:geodesic}
\frac{d^2 x^\mu(\tau)}{d\tau^2} = -\Gamma_{\alpha\beta}^\mu(x^\mu(\tau)) \frac{d x^\alpha(\tau)}{d\tau}\frac{d x^\beta(\tau)}{d\tau}
\end{equation}
%
Where the affine connection $\Gamma_{\alpha\beta}^\mu$ for the metric (\ref{eq:2:confnewt}) is, at first order in the potentials

\begin{equation}
\label{eq:2:connection}
\begin{matrix}
\Gamma_{00}^0 = \frac{1}{c}\partial_t\Psi & ; & \Gamma_{0i}^0=\Gamma_{i0}^0 = \partial_i\Psi \\  
\Gamma_{ij}^0 = \frac{\delta_{ij}a^2}{c}(H+2H(\Phi-\Psi)+\partial_t\Phi) & ; & \Gamma_{00}^i = \frac{\partial_i \Psi}{a^2} \\
\Gamma^i_{j0} = \Gamma^i_{0j} = \frac{\delta_{ij}}{c}(H+\partial_t\Phi) & ; & \Gamma_{jk}^i = (\delta_{ij}\partial_k+\delta_{ik}\partial_j-\delta_{jk}\partial_i)\Phi\\
\end{matrix}
\end{equation}
%
Where the space--time dependency of the quantities has been omitted for notational simplicity. For the sake of expressing WL observables at first order in the potentials $\Phi,\Psi$, it is sufficient to approximate the affine connection at first order. We will make an argument later in the Chapter for higher order post--Newtonian (PN) corrections to be negligible for the scope of this work, following the conclusions of \citep{PNLensing}. We introduce a system of coordinates centered on a fundamental observer on Earth, as illustrated in Figure \ref{fig:2:scheme}

\begin{equation}
\label{eq:2:coord}
x^{\mu} = (ct,\chi,\xp)
\end{equation}
%
Where we indicated the transverse coordinates (with respect of the observer) corresponding to an angle on the sky $\pt$ as $\xp=\chi\pt$. We will adopt the so called flat sky approximation, in which the angles $\pt$ are small. 

\begin{figure}
\begin{center}
\includegraphics[scale=0.75]{Figures/pdf/rayscheme.pdf}
\end{center}
\caption{Coordinate system centered on a fundamental observer on Earth}
\label{fig:2:scheme}
\end{figure}
%
Since photons travel along null geodesics, their 4-momentum $P^\mu=dx^\mu/d\tau=(p^0,\bb{p})$ must satisfy $g_{\mu\nu}P^{\mu}P^{\nu}=0$, which gives the constraint (at first order in $\Psi$)

\begin{equation}
\label{eq:2:nullconstr}
p^0 = c\frac{dt}{d\tau} = p(1-\Psi)
\end{equation} 
%
with $p=\vert\bb{p}\vert$. Using the fact that $d\chi/dt = -c/a$, we can replace the $\tau$ derivatives in (\ref{eq:2:geodesic}) with $\chi$ derivatives using the prescription
\begin{equation}
\label{eq:2:tau2chi}
\frac{d}{d\tau} = \frac{dt}{d\tau}\frac{d\chi}{dt}\frac{d}{d\chi} = -\frac{p(1-\Psi)}{a}\frac{d}{d\chi}
\end{equation}
%
If we focus on the transverse components, the LHS of (\ref{eq:2:geodesic}) looks like
\begin{equation}
\label{eq:2:geodesic-lhs-1}
\frac{d^2\xp}{d\tau^2} = \frac{p}{a}\frac{d}{d\chi}\left(\frac{p}{a}\frac{d\xp}{d\chi}\right)
\end{equation}
%
Following \citep{Dodelson-C11}, we dropped small terms of order $\Psi d\xp$. At dominant order the photon momentum $a$ redshifts as $1/a$, and hence we can pull the product $pa$ out of the differentiation, obtaining 
\begin{equation}
\label{eq:2:geodesic-lhs-2}
\frac{d^2\xp}{d\tau^2} = p^2\frac{d}{d\chi}\left(\frac{1}{a^2}\frac{d\xp}{d\chi}\right)
\end{equation}
%
Now we can focus on the RHS of (\ref{eq:2:geodesic}) which reads, in the transverse components
\begin{equation}
\label{eq:2:geodesic-rhs-1}
\Gamma_{\alpha\beta}^i \frac{d x^\alpha}{d\tau}\frac{d x^\beta}{d\tau} = \frac{p^2}{a^2} \Gamma_{\alpha\beta}^i \frac{d x^\alpha}{d\chi}\frac{d x^\beta}{d\chi}
\end{equation}
%
If we expand the products in (\ref{eq:2:geodesic-rhs-1}) using the affine connection (\ref{eq:2:connection}) at first order in the potentials, we obtain
\begin{equation}
\label{eq:2:geodesic-rhs-2}
\Gamma_{\alpha\beta}^i \frac{d x^\alpha}{d\tau}\frac{d x^\beta}{d\tau} = \frac{p^2}{a^2}\left[\partial_i\Psi-\frac{2aH}{c}\frac{d\xp}{d\chi} -\partial_i\Phi \right]
\end{equation}
%
The complete geodesic equation now becomes 
\begin{equation}
\label{eq:2:geodesic-fo-1}
\frac{d}{d\chi}\left(\frac{1}{a^2}\frac{d\xp}{d\chi}\right) = -\frac{1}{a^2}\left[\nabla_\perp(\Psi-\Phi)-\frac{2aH}{c}\frac{d\xp}{d\chi}\right]
\end{equation}
%
After a few simplifications this assumes the form 
\begin{equation}
\label{eq:2:geodesic-fo-2}
\frac{d^2\xp(\chi)}{d\chi^2} = \nabla_\perp(\Phi(\chi,\xp)-\Psi(\chi,\xp))
\end{equation}

\subsection{Einstein equation}
The potentials $\Phi,\Psi$, as they appear in the metric (\ref{eq:2:confnewt}), must satisfy Einstein equation. Since we limit our study to scalar perturbations, there are only two independent components of the Einstein equation that we need to consider. Since WL physics is dominated by the late time behavior of density perturbations, we will ignore relativistic particles as contribution to the energy momentum tensor and focus on cold matter only. We parametrize the matter density as 

\begin{equation}
\label{eq:2:matterrho}
\rho_m(\bb{x},t) = \rho_m(t)(1+\delta(\bb{x},t))
\end{equation}
%
where $\rho_m(t)$ is the spatially averaged matter density and $\delta(\bb{x},t)$ is the spatially dependent density contrast. With this notation, the two independent components of the linearized Einstein's equation become

\begin{equation}
\label{eq:2:einstein-00}
\nabla^2\Phi +3\frac{a^2H^2\Psi-a^2H\partial_t\Phi}{c^2} = -4\pi G\rho_m\delta
\end{equation}

\begin{equation}
\label{eq:2:einstein-ij}
\nabla^2(\Phi+\Psi) = 0
\end{equation}
%
A few considerations are in order here. First of all, the terms in (\ref{eq:2:einstein-00}) which contain powers of $aH$ are sub--dominant for the WL case of interest, as the laplacian term will always dominate for modes with wavenumber $k$ well inside the Hubble horizon $kc\gg aH$. We can hence drop these terms from (\ref{eq:2:einstein-00}), which then reduces to a Poisson--like equation. Equation (\ref{eq:2:einstein-ij}) comes from the traceless part of the spatial Einstein equation, and as such it is sourced by anisotropic stresses in the matter components. Because such stresses are proportional to the quadrupole of the distributions in momentum, which are negligible for non relativistic species, they can be safely neglected when studying WL. We will then use (\ref{eq:2:einstein-ij}) to conclude $\Psi=-\Phi$ and express the geodesic equation in terms of the gravitational potential $\Phi$

\begin{equation}
\label{eq:2:geodesic-fo-3}
\frac{d^2\xp(\chi)}{d\chi^2} = 2\nabla_\perp\Phi(\chi,\xp(\chi))
\end{equation}

\begin{equation}
\label{eq:2:poisson}
\nabla^2\Phi(\chi,\xp) = -\frac{4\pi G}{c^2}\rho_m(\chi)\delta(\chi,\xp)
\end{equation}  
%
We substituted the time dependency with $\chi$ dependencies using the distance--redshift relation. 

\subsection{Solution to the geodesic equation}
The geodesic equation (\ref{eq:2:geodesic-fo-3}) is a second order differential equation which can be solved exactly once suitable initial conditions are specified. If we indicate with $\pt$ the angular position of the light ray as it is received by the observer we have $\xp(0)=0$ and $d\xp /d\chi(0)=\pt$. Equation (\ref{eq:2:geodesic-fo-3}) can hence be solved by double integration

\begin{equation}
\label{eq:2:geosol-1}
\xp(\chi,\pt) = \chi\pt + 2\int_0^\chi d\chi'\int_0^{\chi'}d\chi'' \nabla_\perp\Phi(\chi,\xp(\chi'',\pt))
\end{equation} 
%
Note that we can exchange the order of the integration in $\chi'$ and $\chi''$ by taking advantage of the triangular shape of the integration domain, and perform one of the integrations analytically to get

\begin{equation}
\label{eq:2:geosol-2}
\xp(\chi,\pt) = \chi\pt + 2\int_0^\chi d\chi'(\chi-\chi')\nabla_\perp\Phi(\chi',\xp(\chi',\pt))
\end{equation} 
%
We expressed the solution (\ref{eq:2:geosol-2}) to (\ref{eq:2:geodesic-fo-3}) in implicit form, as the RHS contains $\xp(\chi,\pt)$ itself. An implicit expression like (\ref{eq:2:geosol-2}) however, present some advantages: first of all, in order to express the solution at some $\chi_s$, we only need to know $\xp$ for $\chi<\chi_s$, making (\ref{eq:2:geosol-2}) solvable numerically via dynamic programming. Moreover, because the potential $\Phi$ appears in the RHS, in the limit in which $\Phi$ is small, the solution in implicit form suggest a straightforward perturbative approximation in powers of $\Phi$, which will be explored later in the Chapter. In the next section, we connect the light ray geodesic (\ref{eq:2:geosol-2}) to the main WL observables.  


%%%%%%%%%%%%%%%%%%%%%%%%%%%%%%%%%%%%%%%%%%%%%%%%%%%%%%%%%%%%%%%%%%%%%%%%%%%%%%%%%%%%%%%%%%%%%%%%%%%%%%%%%%%%%%%%%%%%%%%%%%%

\section{Weak Gravitational Lensing}

\subsection{Weak Lensing observables}
Gravitational Lensing produces apparent distortions in observed background sources shapes due to light ray deflections. Following the notation in Figure \ref{fig:2:scheme}, a light ray captured by the observer at position $\pt$, in reality originated from a point that on the sky corresponds to an angular position $\pmb{\beta}$. Overall shifts in the relation $\pmb{\beta}(\pt)$ do not alter the source shape (that we assume distributed on a plane at $\chi=\chi_s$), but simply cause unobservable image displacements. The lowest order image distortions come from the differential ray shift $\partial \beta_i/\partial \theta_j$, which alter the observed ellipticity of the sources. Higher order flexion corrections to source shapes have been investigated in the literature \citep{BornFlexion}, but will not be treated in this work. Elliptical deformations in the shape of background sources are parametrized in terms of deflection Jacobian $\bb{A}$

\begin{equation}
\label{eq:2:dfl-jacobian}
A_{ij}(\chi_s,\pt) = \frac{\partial \beta_i(\chi_s,\pt)}{\partial \theta_j} = 
\begin{pmatrix}
1-\kappa(\pt)-\gamma^1(\pt) & -\gamma^2(\pt)+\omega(\pt) \\
-\gamma^2(\pt)-\omega(\pt) & 1-\kappa(\pt)+\gamma^1(\pt)
\end{pmatrix}
\end{equation}  
%
\begin{figure}
\begin{center}
\includegraphics[scale=0.35]{Figures/eps/distortion.eps}
\end{center}
\caption{Effect of differential distortions $\kappa,\pmb{\gamma},\omega$ on a circular image}
\label{fig:2:distortion}
\end{figure}
%
In this parametrization, $\kappa$ is called the WL convergence, $\pmb{\gamma}=(\gamma^1,\gamma^2)$ is the WL cosmic shear and $\omega$ is the WL rotation. Inverting equation (\ref{eq:2:dfl-jacobian}) we obtain the relations

\begin{equation}
\label{eq:2:dfl-inverse}
\begin{matrix}
\kappa = 1-\frac{1}{2}\Tr \bb{A} & ; & \gamma^1 = \frac{1}{2}(A_{yy} - A_{xx}) \\
\gamma^2 = -\frac{1}{2}(A_{xy}+A_{yx}) & ; & \omega = \frac{1}{2}(A_{xy} - A_{yx})
\end{matrix}
\end{equation}
% 
Figure \ref{fig:2:distortion} shows the physical interpretation of these different types of distortions, for an un--lensed circular image. The convergence has to do with the background source apparent magnifications, and is not directly observable unless the un--lensed size of the source is known in advance. The cosmic shear $\pmb{\gamma}$ encodes the ellipticity of the distortion, while the rotation $\omega$ is the angular tilt in the distorted image shape. Calling $I(\pmb{\beta})$ and $I_{\rm obs}(\pt)$ the emitted and observed source intensity profiles respectively, we can define the observable ellipticity $\pmb{\epsilon}$ of the distorted image in terms of the quadratic moments of the intensity 

\begin{equation}
\label{eq:2:ellipticity-1}
\begin{matrix}
\epsilon^1 = \frac{q_{xx}-q_{yy}}{q_{xx}+q_{yy}} & ; & \epsilon^2 = \frac{2q_{xy}}{q_{xx}+q_{yy}}
\end{matrix}
\end{equation} 
%
Where the quadrupole moments $q_{ij}$ are defined by 
\begin{equation}
\label{eq:2:quadrupole}
q_{ij} = \int d\pt \theta_i\theta_j I_{\rm obs}(\pt) = \int d\pmb{\beta} (\bb{A}^{-1}\pmb{\beta})_i(\bb{A}^{-1}\pmb{\beta})_j I(\pmb{\beta})
\end{equation}
%
In the last equality flexion corrections have been ignored. If the un--lensed image is a circle and the image is small enough so that $\kappa,\pmb{\gamma}$ are constant over its profile, the integral can be computed explicitly in terms of the inverse of $\bb{A}$

\begin{equation}
\label{eq:2:ellipticity-2}
\begin{matrix}
\epsilon^1 = \frac{(A^{-1}_{xx})^2-(A^{-1}_{yy})^2}{(A^{-1}_{xx})^2+(A^{-1}_{yy})^2+2(A^{-1}_{xy})^2} & ; & \epsilon^2 = \frac{2A^{-1}_{xy}(A^{-1}_{xx}+A^{-1}_{yy})}{(A^{-1}_{xx})^2+(A^{-1}_{yy})^2+2(A^{-1}_{xy})^2} &
\end{matrix}
\end{equation}
%
If we express the numerator and denominator at lowest order in $\kappa,\pmb{\gamma}$ we get immediately
\begin{equation}
\label{eq:2:ellipticity-3}
\pmb{\epsilon} = \frac{2\pmb{\gamma}}{1-\kappa}
\end{equation}
%
which shows that the ellipticity is proportional to the shear and can hence be used to estimate it. Even if the observable quantity is the reduced shear $\pmb{\gamma}/(1-\kappa)$, for practical purposes in WL when $\kappa\ll 1$, one can approximate $\pmb{\epsilon}\approx 2\pmb{\gamma}$. In the following paragraph we will show how to relate the WL quantities $\kappa,\pmb{\gamma},\omega$ to the gravitational lensing potential $\Phi$. 

\subsection{Ray tracing}
Relating observable WL quantities such as the source ellipticity $\pmb{\epsilon}$ and the cosmic shear $\pmb{\gamma}$ can be done using equation (\ref{eq:2:geosol-2}), which relates light geodesics to the matter density fluctuations through the lensing potential. We can convert the transverse physical coordinates $\xp$ into angles $\pmb{\beta}$ dividing by the longitudinal distance

\begin{equation}
\label{eq:2:geosol-beta}
\pmb{\beta}(\chi,\pt) = \pt + 2\int_0^\chi d\chi'\left(1-\frac{\chi'}{\chi}\right)\nabla_\perp\Phi(\chi',\chi'\pmb{\beta}(\chi',\pt))
\end{equation}
%
And by differentiating (\ref{eq:2:geosol-beta}) with respect to $\pt$ we can obtain a similar relation for the Jacobian $\bb{A}$
\begin{equation}
\label{eq:2:geosol-jac}
A_{ij}(\chi,\pt) = \delta_{ij} + 2\int_0^\chi d\chi'\chi'\left(1-\frac{\chi'}{\chi}\right)\partial_i\partial_k\Phi(\chi',\chi'\pmb{\beta}(\chi',\pt))A_{kj}(\chi',\pt)
\end{equation}
%
Like (\ref{eq:2:geosol-2}), equation (\ref{eq:2:geosol-jac}) is an implicit relation that allows to calculate $\bb{A}(\chi_s,\pt)$ at some $\chi_s$ once all $\bb{A}(\chi,\pt)$ are known for $\chi<\chi_s$. A numerical solution to this problem that makes use of the multi--lens--plane algorithm \citep{RayTracingJain,RayTracingHartlap} will be explored in the next Chapter. 
The fact that the potential $\Phi$ appears on the RHS of (\ref{eq:2:geosol-jac}) suggests the possibility of a perturbative expansion of $\bb{A}$ in powers of $\Phi$. Confidence in the validity of the perturbative approach comes from the fact that in the WL limit $\Phi$ is small. If we focus in at most quadratic contributions in $\Phi$, we can write

\begin{equation}
\label{eq:2:jacobian-pert}
A_{ij}(\chi,\pt) = \delta_{ij} + A_{ij}^{(1)}(\chi,\pt) + A_{ij}^{(2)}(\chi,\pt) + O(\Phi^3)
\end{equation}
%
To obtain an expression at linear order in $\Phi$, we can replace Jacobian of the RHS of (\ref{eq:2:geosol-jac}) with the identity and the spatial argument of $\Phi$ with $\chi\pt$. This approach is analogous to the Born approximation commonly used in Quantum Mechanics when computing scattering amplitudes at first order in the interaction potential. We can write

\begin{equation}
\label{eq:2:jacobian-fo}
A_{ij}^{(1)}(\chi,\pt) = 2\int_0^\chi d\chi'\chi'\left(1-\frac{\chi'}{\chi}\right)\partial_i\partial_j\Phi(\chi',\chi'\pt)
\end{equation}
%
Equation (\ref{eq:2:jacobian-fo}) essentially says that, at lowest perturbative order, the lensing Jacobian is the projection of the tidal field $\partial_i\partial_j\Phi$ on the unperturbed geodesic trajectory $\chi\pt$, weighted with a lensing kernel

\begin{equation}
W(\chi',\chi) = \chi'\left(1-\frac{\chi'}{\chi}\right) 
\end{equation}
%
With the use of (\ref{eq:2:dfl-inverse}) we can express the convergence at first order in the lensing potential
\begin{equation}
\label{eq:2:kappa-1}
\kappa^{(1)}(\chi_s,\pt) = \int_0^{\chi_s}d\chi W(\chi,\chi_s)\delta_L(\chi,\chi\pt)
\end{equation}
%
Where we defined the lensing density $\delta_L$ as 
\begin{equation}
\label{eq:2:deltalens}
\delta_L(\chi,\xp) = -\nabla^2_\perp \Phi(\chi,\xp)
\end{equation}
%
Equation (\ref{eq:2:kappa-1}) essentially means that the WL convergence is the integrated column density fluctuations on the line of sight between the observer and the source. Quadratic corrections to the linear relation (\ref{eq:2:jacobian-fo}) between $\bb{A}$ and $\Phi$ come from two different sources in equation (\ref{eq:2:geosol-jac}): one term is generated by replacing the Jacobian in the RHS of (\ref{eq:2:geosol-jac}) with its first order approximation $A_{ij}^{(1)}$, the other comes from the transverse argument of the tidal field $\partial_i\partial_j\Phi$, which contains the geodesic perturbations. Using the approximation

\begin{equation}
\label{eq:2:tidalapprox}
\partial_i\partial_j\Phi(\chi,\chi\pmb{\beta}(\chi,\pt)) = \partial_i\partial_j\Phi(\chi,\chi\pt) + \partial_i\partial_j\partial_k\Phi(\chi,\chi\pt)\chi[\beta_k^{(1)}(\chi,\pt)-\theta_k] + O(\Phi^3) 
\end{equation}
%
we can write a second order expression for $\bb{A}$ 

\begin{equation}
\label{eq:2:jacobian-so}
\bb{A}^{(2)}(\chi,\pt) = \bb{A}^{(2-ll)}(\chi,\pt) + \bb{A}^{(2-gp)}(\chi,\pt) 
\end{equation}
%
\begin{equation}
\label{eq:2:jacobian-so-ll}
A_{ij}^{(2-ll)}(\chi,\pt) = 4\int_0^\chi d\chi'\int_0^{\chi'}d\chi'' W(\chi'',\chi')W(\chi',\chi)\partial_i\partial_k\Phi(\chi',\chi'\pt)\partial_k\partial_j\Phi(\chi'',\chi''\pt)
\end{equation} 
%
\begin{equation}
\label{eq:2:jacobian-so-gp}
A_{ij}^{(2-gp)}(\chi,\pt) = 4\int_0^\chi d\chi'\int_0^{\chi'}d\chi'' W(\chi'',\chi')W(\chi',\chi)\frac{\chi'}{\chi''}\partial_i\partial_j\partial_k\Phi(\chi',\chi'\pt)\partial_k\Phi(\chi'',\chi''\pt)
\end{equation} 
%


\subsection{$E/B$ mode decomposition of the shear}   

\bibliography{ref}