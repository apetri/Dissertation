%%%%%%%%%%%%%%%%%%%%%%%%%%%%%%%%%%%%%%%%%%%%%%%%%%%%%%%%%%%%%%%%%%%%%%%%%%%

\chapter{An application to real data: the CFHTLenS galaxy survey}
\lhead[\fancyplain{}{\thepage}]{\fancyplain{}{\rightmark}}
 \thispagestyle{plain}
\setlength{\parindent}{10mm}

\label{chp:6}
%%%%%%%%%%%%%%%%%%%%%%%%%%%%%%%%%%%%%%%%%%%%%%%%%%%%%%%%%%%%%%%%%%%%%%%%%%%

\section{Catalogs}

\section{Emulator}

\subsection{Cosmological parameter sampling}
\label{sec:6:sampling}
We describe the procedure we used to sample the $\Lambda$CDM parameter space. We consider a subset of $N_\pi=3$ parameters $\bb{p}=(\Omega_m,w_0,\sigma_8)$ and we seek a way to uniformly sample it so that no parameter is repeated twice. This sampling scheme takes the name of \textit{latin hypercube} \citep{Coyote2}. One way to implement the latin hypercube sample in practice is to set a $N_\pi$--dimensional bounding box that will contain all the sampled points, and normalize it to $[0,1]^{N_\pi}$ for simplicity. We can then set the number $N_M$ of cosmological models we wish to sample and distribute in an uniform latin hypercube scheme. Following \citep{Coyote2,PetriCFHTMink} we define a cost function 

\begin{equation}
\label{eq:6:cost}
\mathcal{C}(\bb{P}) = \frac{2{N_\pi}^{1/2}}{N_M(N_M-1)}\sum_{i<j} \frac{1}{\vert \bb{P}_i-\bb{P}_j \vert}
\end{equation} 
%
where $\bb{P}$ is a $N_M\times N_\pi$ matrix that contains the information on the sampled points and the sum runs over all pair of points. In order to sample the hypercube uniformly, we seek a configuration $\bb{P}$ that minimizes the cost function (\ref{eq:6:cost}) while enforcing the latin hypercube constraint. Because $\mathcal{C}$ is effectively the Coulomb potential energy of $N_M$ unit point charges confined in a box, its minimum leads to a uniform configuration. The simplest latin hypercube arrangement corresponds to the diagonal design $\bb{P}^0$, in which the points are arranged on the diagonal of the hypercube

\begin{equation}
\label{eq:6:diagonal}
\bb{P}^0_{i} = \frac{i}{N_M}\underbrace{(1,1,...,1)}_{N_\pi}
\end{equation}
%  
Of course this trivial arrangement is far from optimal. A possible heuristic method to find out a configuration $\bb{P}$ that minimizes (\ref{eq:6:cost}) is simulated annealing \citep{Skiena}, which however is too computationally expensive for our purposes. We resort on this less accurate, but faster, heuristic scheme instead: 
\begin{enumerate}
\item Start from the diagonal design $\bb{P}^0$
\item Pick a random pair of points $(i,j)$ among the $N_M(N_M-1)/2$ available, pick a random parameter $p$ among the $N_\pi$ available
\item Swap $P_{ip}$ with $P_{jp}$ and vice--versa (the swap preserves the latin hypercube property), recompute the cost function $\mathcal{C}$
\item If the cost decrease, keep the swap, otherwise revert to the previous configuration
\item Re--iterate the procedure starting from point 2. 
\end{enumerate}
%
\begin{figure}
\begin{center}
\includegraphics[scale=0.4]{Figures/eps/cfht_design.eps}
\end{center}
\caption{Distribution of the $(\Omega_m,w_0,\sigma_8)$ sampled triplets in the parameter prior box. We show both the $(\Omega_m,w_0)$ (left) and the $(\Omega_m,\sigma_8)$ (right) projections. The black points correspond to the $N_M=91$ latin hypercube models, and the red cross correspond to the fiducial $\Lambda$CDM parameters shown in Table \ref{tab:1:cosmopar}. The design is the result of $10^5$ iterations of the heuristic procedure described in \S~\ref{sec:6:sampling}.}
\label{fig:6:sampling}
\end{figure}
%
After several iterations, we are left with a latin hypercube design which samples the parameter space approximately uniformly. The last step is to rescale the parameter coordinates from the $[0,1]$ bounds to their originally intended values. The latin hypercube design we use for the present analysis is shown in Figure \ref{fig:6:sampling}. 


\subsection{Interpolation}

\begin{figure}
\begin{center}
\includegraphics[scale=0.4]{Figures/eps/cfht_emulator_accuracy.eps}
\end{center}
\caption{}
\label{fig:6:interpolation}
\end{figure}



\section{$\Lambda$CDM parameter inference}

\subsection{PCA projection}

\subsection{Density fluctuations}

\subsection{Dark Energy}


\bibliography{ref}