%%%%%%%%%%%%%%%%%%%%%%%%%%%%%%%%%%%%%%%%%%%%%%%%%%%%%%%%%%%%%%%%%%%%%%%%%%%

\chapter{Cosmological parameter inference}
\lhead[\fancyplain{}{\thepage}]{\fancyplain{}{\rightmark}}
 \thispagestyle{plain}
\setlength{\parindent}{10mm}
\label{chp:5}

%%%%%%%%%%%%%%%%%%%%%%%%%%%%%%%%%%%%%%%%%%%%%%%%%%%%%%%%%%%%%%%%%%%%%%%%%%%

In this Chapter we give an overview and discuss the cosmological parameter inference techniques used in this work. Inferring $\Lambda$CDM parameters from observations requires the construction of $\kappa$ images from $\pmb{\gamma}$ catalogs and the measurement of a feature $\bb{d}$ from the reconstructed image. When a forward model $\bb{d}(\bb{p})$ that connects features to cosmological parameters $\bb{p}$ is available, an estimate of the parameters $\bbh{p}$ can be derived from an estimate of the feature $\bbh{d}$ in a Bayesian framework. In this Chapter we review the probabilistic framework and we study WL parameter constraining capabilities using the features discussed in Chapter \ref{chp:4}. We also discuss a variety of constraint degradation sources and propose possible remedies.    

\section{Bayesian formalism}
In this paragraph we review the Bayesian probabilistic framework that we base the parameter inference on. We denote with $\bb{d}$ an $N_b$--dimensional image feature and with $\bb{p}$ a $N_\pi$ dimensional tuple of $\Lambda$CDM cosmological parameters (see Table \ref{tab:1:cosmopar}). We also indicate as $\bbh{d}$ an estimate of a feature from a simulated $\kappa$ field of view, as $\dobs$ a measured feature from an actual observation and as $\bbh{p}$ the deriving parameter estimate. We assume the existence of a forward model $\bb{d}(\bb{p})$ which can be obtained using our WL simulation pipeline or, in special cases such as for the $\kappa$ power spectrum, using analytical codes such as NICAEA \citep{Nicaea}. Using Bayes theorem, the likelihood $\lik{\bbh{p}}{\dobs}$ of an estimate $\bbh{p}$ given an observation $\dobs$ is given by

\begin{equation}
\label{eq:5:bayesthm}
\lik{\bbh{p}}{\dobs,\bb{d}(\bb{p})} = \frac{\lik{\dobs}{\bbh{p},\bb{d}(\bb{p})}\Pi(\bbh{p})}{\mathcal{L}(\dobs)}
\end{equation}
%
In the notation of equation (\ref{eq:5:bayesthm}) $\Pi$ encodes prior information on the parameters coming from WL independent probes (such as CMB experiments for example) and $\mathcal{L}(\dobs)$ is the overall observation likelihood, which acts as a $\bb{p}$--independent normalization factor in the parameter likelihood, which we will ignore in the prosecution of this work. We make a Gaussian assumption for the feature likelihood 

\begin{equation}
\label{eq:5:gaussfeatlik}
\lik{\dobs}{\bbh{p},\bb{d}(\bb{p})} = \frac{1}{(2\pi)^{N_b/2}\vert\bb{C}\vert}\exp\left(-\frac{1}{2}(\dobs-\bb{d}(\bb{p}))^T\bb{C}^{-1}(\dobs-\bb{d}(\bb{p}))\right) 
\end{equation}
%
where $\bb{C}$ is a $\bb{p}$--independent feature--feature covariance matrix. The Gaussian assumption for the data likelihood is justified by the Central Limit Theorem applied to feature averaging over a large number of fields of view, which can be as high as $O(10^3)$ for a $\theta_{\rm FOV}=3.5\,{\rm deg}$ sized tiling of a modern galaxy survey such as LSST. The independence of the covariance matrix $\bb{C}$ on the cosmological parameters is not justified in the present work. The drawbacks of such assumption will be reserved for future investigation. 

Once the parameter likelihood is known, parameter confidence intervals can be obtained looking at surfaces with constant $\mathcal{L}$ in $\bb{p}$ space. We define an $N\sigma$ parameter confidence interval as the $\bb{p}$ space region with $\mathcal{L}>\mathcal{L}_N$. The likelihood confidence levels are defined as 

\begin{equation}
\label{eq:5:liklevel}
\int_{\mathcal{L}>\mathcal{L}_N}\lik{\bbh{p}}{\dobs,\bb{d}(\bb{p})} d\bbh{p} = \frac{1}{\sqrt{2\pi}}\int_{-N}^Ne^{-x^2/2}dx
\end{equation}  
%
Note that this definition of $N\sigma$ confidence intervals corresponds to the commonly accepted one when $\mathcal{L}(\bbh{p})$ is Gaussian. If this is the case, calling $\bbh{p}_0$ the location of the likelihood peak, the matrix $\bb{\Sigma}$, defined by

\begin{equation}
\label{eq:5:parcov}
\left(\Sigma^{-1}\right)_{\alpha\beta} = -\left(\frac{\partial^2\log\mathcal{L}(\bbh{p})}{\partial\h{p}_\alpha\partial\h{p}_\beta}\right)_{\bbh{p}=\bbh{p}_0}
\end{equation}
%
is the parameter covariance matrix. We observe that, even if the parameter likelihood is not Gaussian, we can use the peak location $\bbh{p}_0$ and the matrix (\ref{eq:5:parcov}) as an estimate of the parameters and their covariance, although a complete characterization of the parameter space through the confidence intervals defined in (\ref{eq:5:liklevel}) is preferred. The confidence intervals can be calculated by drawing samples from $\mathcal{L}(\bbh{p})$ using Markov Chain Monte Carlo (MCMC) techniques, which are implemented by many convenient software packages (see for example \citep{emcee}). 

\subsection{Fisher matrix approximation}
Parameter inference becomes simpler if the forward model $\bb{d}(\bb{p})$ is linear in the parameters. Linearity can be safely assumed in the limit in which confidence intervals are expected to be localized around the peak of the likelihood, which is the case for large scale WL surveys. We can write 

\begin{equation}
\label{eq:5:linapprox}
\bb{d}(\bb{p}) = \bb{d}_0 + \bb{M}(\bb{p}-\bb{p}_0) + O(\vert\bb{p}-\bb{p}_0\vert^2)
\end{equation}          
%
Assuming a flat prior $\Pi(\bbh{p})$ and plugging (\ref{eq:5:linapprox}) into (\ref{eq:5:gaussfeatlik}) we get, for the $\bb{p}$--dependent part of the likelihood

\begin{equation}
\label{eq:5:linapprox-lik}
-2\log\mathcal{L}(\bb{p}) = \left[\dobs-\bb{d}_0-\bb{M}(\bb{p}-\bb{p}_0)\right]^T\bb{\Psi}\left[\dobs-\bb{d}_0-\bb{M}(\bb{p}-\bb{p}_0)\right]
\end{equation}
%
From (\ref{eq:5:linapprox-lik}) we can immediately get an estimate for the peak of the likelihood $\bbh{p}_0$ and for the parameter covariance $\bb{\Sigma}$ using (\ref{eq:5:parcov})

\begin{equation}
\label{eq:5:linapprox-peak}
\bbh{p}_0 = \bb{p}_0 + \left(\bb{M}^T\bb{\Psi}\bb{M}\right)^{-1}\bb{M}^T\bb{\Psi}\left(\dobs-\bb{d}_0\right)
\end{equation}
%
\begin{equation}
\label{eq:5:linapprox-cov}
\bb{\Sigma} = \bb{F}^{-1} \equiv \left(\bb{M}^T\bb{\Psi}\bb{M}\right)^{-1} 
\end{equation}
%
Equations (\ref{eq:5:linapprox-peak}), (\ref{eq:5:linapprox-cov}) constitute an important result which takes the name of Fisher matrix approximation, and $\bb{F}\equiv\bb{M}^T\bb{\Psi}\bb{M}$ takes the name of Fisher information matrix. In the case where prior information on the parameters is available, the estimates for the likelihood peak and parameter covariance are modified. For a Gaussian prior with distribution

\begin{equation}
\label{eq:5:gaussprior}
\Pi(\bb{p}) = \frac{\vert\bb{F}_\Pi\vert}{(2\pi)^{N_\pi/2}}\exp\left(-\frac{1}{2}(\bb{p}-\bb{p}_\Pi)^T\bb{F}_\Pi(\bb{p}-\bb{p}_\Pi)\right)
\end{equation}
%
we have 

\begin{equation}
\label{eq:5:linapprox-peak-prior}
\bbh{p}_0 = \left(\bb{F}+\bb{F}_\Pi\right)^{-1}\left[\bb{M}^T\bb{\Psi}(\bbh{d}-\bb{d}_0) + \bb{F}_\Pi\bb{p}_\Pi + \bb{F}\bb{p}_0\right]
\end{equation}
%
\begin{equation}
\label{eq:5:linapprox-cov-prior}
\bb{\Sigma} = \left(\bb{F}+\bb{F}_\Pi\right)^{-1} 
\end{equation}

\section{Born approximation induced bias}

\section{Error degradation induced by covariance noise}

\subsection{Covariance matrix estimation}

\subsection{Perturbative approach}

\section{Dimensionality reduction}

\section{Weak Lensing constraining power}

\bibliography{ref}